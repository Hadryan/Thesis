% !TEX root = ../thesis-main.tex
\part{Learning with Weak Supervision}
\label{part2}
%

Human learners often compensate effortlessly for imperfect data. Even animals can learn from conditioning in which they are rewarded inconsistently. However, it seems difficult to build such fault-tolerance into machine learning systems.

An essential aspect of any practical learning algorithm is the need to learn from imperfect data. Few real-world problems operate under perfect conditions. The ML systems may have to learn from inconsistent data provided by different sensors, or may be missing a crucial information.
Nonetheless, even imperfect data can contain a great deal of valid information.


data is a first-class citizen in so many real-world systems


Collecting such training sets by hand is often infeasible due to the time and expense of labeling data; moreover, hand-labeled training sets are static and must be completely relabeled when real-world modeling goals change.

Overview
Modern representation learning techniques like deep neural networks have had a major impact on a wide range of tasks, achieving new state-of-the-art performances on benchmarks using little or no feature engineering. However, these gains are often difficult to translate into real-world settings because they usually require massive hand-labeled training sets. Collecting such training sets by hand is often infeasible due to the time and expense of labeling data; moreover, hand-labeled training sets are static and must be completely relabeled when real-world modeling goals change.

Increasingly popular approaches for addressing this labeled data scarcity include using weak supervision---higher-level approaches to labeling training data that are cheaper and/or more efficient, such as distant or heuristic supervision, constraints, or noisy labels; multi-task learning, to effectively pool limited supervision signal; data augmentation strategies to express class invariances; and introduction of other forms of structured prior knowledge. An overarching goal of such approaches is to use domain knowledge and data resources provided by subject matter experts, but to solicit it in higher-level, lower-fidelity, or more opportunistic ways.


In Part~\ref{part2} of this book, we address one of our research questions:
\begin{resqbox}
\emph{\resq{p1}}
\end{resqbox}


The more neural networks become deep and complex, the more it is crucial for them to be trained on massive amounts of training data.
In many applications, rich annotations are costly to obtain and task-specific training data is now a critical bottleneck. 


\subsection{what is weak supervision}
We aim to study how human can supervise machine learning systems, by labeling training data programmatically instead of labeling by hand. Then, given a vast amount of pragmatically generated labeled data, we discuss how to design neural networks that can go beyond the imperfection in the weakly annotated data.


In the first chapter of this part, Chapter~\ref{chap:4}, we address the following research question:
\begin{resqbox}
\emph{\resq{c2}}
\end{resqbox}

Then, in the second chapter of this part, Chapter~\ref{chap:5} we get down to the following research question:
\begin{resqbox}
\emph{\resq{c3}}
\end{resqbox}



In this paper, we propose a method for training neural networks when we have a large set of data with weak labels and a small amount of data with true labels. In our proposed model, we train two neural networks: a \tnet, the learner and a \cnet, the meta-learner. 
The \tnet is optimized to perform a given task and is trained using a large set of unlabeled data that are weakly annotated. We propose to control the magnitude of the gradient updates to the \tnet using the scores provided by the second \cnet, which is trained on a small amount of supervised data. Thus we avoid that the weight updates computed from noisy labels harm the quality of the \tnet model.

To this end, we propose ``\fwlfulllc'' (\fwl), a semi-supervised student-teacher approach for training deep neural networks using weakly-labeled data. \fwl modulates the parameter updates to a \emph{student} network (trained on the task we care about) on a per-sample basis according to the posterior confidence of its label-quality estimated by a \emph{teacher} (who has access to the high-quality labels).  Both student and teacher are learned from the data. We evaluate \fwl on two tasks in information retrieval and natural language processing where we outperform state-of-the-art alternative semi-supervised methods, indicating that our approach makes better use of strong and weak labels, and leads to better task-dependent data representations.


NO LABEL for chap 4
SOME LABEL for chap 5

\medskip

% \chapter[]{\titleof{c4}}
\label{chap:4}
%
\begin{quote}
In many applications, to overcome the training data scarcity, we can provide supervision signals for learning algorithms not by hand-labeling the data, but by pseudo-labeling the data programmatically. This way, we can generate a much larger training set with a fairly low cost. However, we need to design algorithms that are capable of learning from weakly annotated labels and going beyond the imperfection in pseudo-labels. 
\end{quote}
%
\section{Introduction}
Neural networks are making great progress in many tasks in computer vision~\citep{krizhevsky2012imagenet}, natural language processing~\citep{collobert2008unified}, and information retrieval~\citep{welling2005exponential}. However, these models are data hungry and their performance is strongly correlated with the amount of available labeled data, which is not always readily available and can be expensive to obtain. 

Looking into the research works done in this area, most of them target stable benchmark tasks where standard large-enough datasets exist to train neural networks. However, the labeled data become the scarce commodity when we stray slightly from these standard benchmark tasks toward the realm of real-world applications. In this chapter, we focus on one of our research questions:
\resq{c4}

We aim to study how human can supervise machine learning systems, by labeling training data programmatically instead of labeling by hand. Then, given a vast amount of pragmatically generated labeled data, we discuss how to design neural networks that can go beyond the imperfection in the weakly annotated data. We also study how the ability to learn from noisy signals can lead to better performance when we have intentionally added noise to the training signals in a privacy-preserving training setup.

\medskip
In this chapter, we mainly target the \emph{ranking task}, as one of the core IR problems, where despite the advances of neural network based methods in many other related tasks like reading comprehension, there has been a little progress, mainly due to the lack of a large scale public dataset with query-document pairs labeled by relevance. 

We propose to use a heuristic based ranking method to generate pseudo-labels for a large set of unlabeled query-document pairs to train a neural ranking model given these pseudo-labels as sort of weak annotations.  We tried different architectures, in terms of different objectives and different input representations and study how they learn the ranking task in weak supervision setup.

Interestingly, we observe that using just training data that are annotated by a weak annotator model as the weak annotator, we can outperform that weak annotator on the test data. Based on our analysis, the achieved performance is generally indebted to three main factors: 
%
First, defining an objective function that aims to learn the ranking instead of calibrated scoring to relax the network from fitting to the imperfections in the weakly supervised training data.
%
Second, letting the neural networks learn optimal query/document representations instead of feeding them with a representation based on predefined features. This is a key requirement to maximize the benefits from deep learning models with weak supervision as it enables them to generalize better.
%
Third and last, the weak supervision setting makes it possible to train the network on a massive amount of training data, which is crucial for learning representations.
%

We further thoroughly analyze the behavior of models to understand what they learn, what is the relationship among different models, and how much training data is needed to go beyond the weak supervision signal. We also study if employing deep neural networks may help in different situations.
%
We also examine the scenario of using the network trained on a weak supervision signal as a pre-training step. We demonstrate that, in the ranking problem, the performance of deep neural networks trained on a limited amount of supervised data significantly improves when they are initialized from a model pre-trained on weakly labeled data.

Finally, we study how a neural ranking model that learns from weak/noisy signals can be effectively employed in a setup that noise is intentionally added to the training signal to preserve privacy.


\subsection{Detailed Research Questions}
We break down our main research question in this chapter into three concrete research questions:
\begin{resqbox}
\begin{enumerate}
\item[\textbf{\resqname{c4.1}}] \emph{\resqcontent{c4.1}}
\item[\textbf{\resqname{c4.2}}] \emph{\resqcontent{c4.2}}
\item[\textbf{\resqname{c4.3}}] \emph{\resqcontent{c4.3}}
\end{enumerate}
\end{resqbox}
In the following sections, we will address these research questions.
\section{Weakly Supervised Neural Rankers}
\label{sec:weakly_supervised_neural_rankers}
Despite the promising performance from neural networks on many language understanding tasks, ranking has remained a challenging problem. Besides the inherent difficulty of ``assessing relevance,'' the lack of availability of public large scaled datasets that consist query-document pairs annotated by relevance labels, makes it difficult to advance data hungry models for this task.

Therefore, it is essential to come up with solutions that let us train neural ranking models, where there is no labeled data it is only available at an extremely limited size. 
One the main idea to tackle this problem is to make use of weak human supervision or weakly labeled data, as it is much cheaper to collect or readily available at much larger scale. This section focuses on addressing the following questions:
\resq{c4.1}

We propose to pseudo-label a large set of unlabeled data using an unsupervised method and train a neural ranker using these ``weak'' or ``noisy'' labels. Given this setup, we examine various neural ranking models with different ranking architectures and objectives, i.e., point-wise and pair-wise, as well as different input representations, from encoding query-document pairs into dense\:/\:sparse vectors to learning query\:/\:document embedding representations. 

Our results have broad impact as the proposal to use unsupervised traditional methods as weak supervision signals and is applicable to a variety of IR tasks, such as filtering or classification, without the need for supervised data.  More generally, our approach unifies the classic IR models with currently emerging data-driven approaches in an elegant way.

\subsection{Pseudo-Labeling Unlabeled Data}
\label{sec:pseudo_labeling}
We use the idea of ``Pseudo-Labeling'' and propose to leverage a classic unsupervised IR model to annotate a large amount of unlabeled data and infer weak labels and use this signal to train supervised models as if we had the ground truth labels.
Since the data is generated programmatically, we can generate billions of training samples with almost no cost.\footnote{Weak supervision for training a ranker may refer to using click-through data. Here, we assume that no external information, e.g. search logs, is available.}

We focus on query-dependent ranking as a core IR task. To this aim, we take a well-performing existing unsupervised retrieval model, such as BM25. This model plays the role of ``pseudo-labeler'' in our learning scenario. In more detail, given a target collection and a large set of training queries (without relevance judgments), we make use of the pseudo-labeler to rank/score the documents for each query in the training query set. The goal is to train a ranking model given the scores/ranking generated by the pseudo-labeler as a weak supervision signal.

In the followings, we describe different neural architectures in details and finally investigate their effectiveness when trained on weakly annotated data.  

\subsection{Neural Ranking Architectures}
\label{sec:neural_ranking_arch}
In this section, we introduce three different neural ranking models that are trained based on different ``objectives''. We describe the architecture of the base neural network shared by these models. We further discuss the three different ``input layers'' used in our neural rankers to encode information of given query-document pairs.
\medskip

\label{sec:models}
\begin{figure}[t]
    \centering
    \begin{subfigure}[t]{0.26\columnwidth}
        \centering
        \includegraphics[height=5cm]{03-part-02/chapter-04/figs_and_tables/fig_model_1.png}%
        \caption{\label{fig:m1}\mone model}
    \end{subfigure}%
    ~
    \begin{subfigure}[t]{0.40\columnwidth}
        \centering
        \includegraphics[height=5cm]{03-part-02/chapter-04/figs_and_tables/fig_model_2.png}%
        \caption{\label{fig:m2}\mtwo model}
    \end{subfigure}%
    ~
    \begin{subfigure}[t]{0.37\columnwidth}
        \centering
        \includegraphics[height=5cm]{03-part-02/chapter-04/figs_and_tables/fig_model_3.png}%
        \caption{\label{fig:m3}\mthree model}
    \end{subfigure}%
    \caption{\label{fig:ranking-arch} Different ranking architectures.}
\end{figure}
%

We define three different ranking models, one point-wise and two pair-wise models:

\subsubsection{\label{sec:modelone}\ModelOne} 
This architecture models a point-wise ranking model that learns to predict retrieval scores for query-document pairs. More formally, the goal in this architecture is to learn a \emph{scoring function} $\mathcal{S}(q, d; \theta)$ that determines the retrieval score of document $d$ for query $q$, given a set of model parameters $\theta$.
%
In the training stage, we are given a training set comprising of training samples each a triple $x = (q,d, s_{q,d})$, where $q$ is a query from training query set $Q$, $d$ represents a retrieved document for the query $q$, and $s_{q,d}$ is the relevance score (calculated by a weak supervisor), which is acquired using a retrieval scoring function in our setup.
%
We consider the mean squared error as the loss function for a given batch of training samples:
\begin{equation}
\mathcal{L}(b; \theta) = \frac{1}{|b|} \sum_{i=1}^{|b|}{(\mathcal{S}(\{q, d\}_i; \theta) - s_{\{q, d\}_i})^2}
\end{equation}
where $\{q, d\}_i$ denotes the query and the corresponding retrieved document in the $i^{th}$ training sample, i.e., $x_i$ in the batch $b$.
The conceptual architecture of the model is illustrated in Figure~\ref{fig:m1}.


\subsubsection{\label{sec:modeltwo}\ModelTwo}
In this model, similar to the previous one, the goal is to learn a scoring function $\mathcal{S}(q, d; \theta)$ for a given pair of query $q$ and document $d$ with the set of model parameters $\theta$. 
However, unlike the previous model, we do not aim to learn a calibrated scoring function. 
In this model, as is depicted in Figure~\ref{fig:m2}, we use a pair-wise scenario during training in which we have two point-wise networks that share parameters and we update their parameters to minimize a pair-wise loss.
Each training sample has five elements: $x = (q,d_1, d_2, s_{q,d_1}, s_{q,d_2})$.
During the inference, we treat the trained model as a point-wise scoring function to score query-document pairs.

We have tried different pair-wise loss functions and empirically found that the model learned based on the hinge loss (max-margin loss function) performs better than the others. 
Hinge loss is a linear loss that penalizes examples that violate the margin constraint. It is widely used in various learning to rank algorithms, such as Ranking SVM~\citep{Herbrich:1999}. The hinge loss function for a batch of training samples is defined as follows:
\begin{equation}
\begin{aligned}
\mathcal{L}(b; \theta) = \frac{1}{|b|}
\sum_{i=1}^{|b|}
\max\big\{
& 
0, \varepsilon - \text{sign}
(s_{\{q, d_1\}_i} - s_{\{q, d_2\}_i})
& \\ & 
\left(\mathcal{S}\left(\{q, d_1\}_i; \theta\right) -\mathcal{S}\left(\{q, d_2\}_i; \theta\right)\right)
\big\}
, 
%\nonumber
\end{aligned}     
\end{equation}
where $\varepsilon$ is the parameter determining the margin of hinge loss. We found that as we compress the outputs to the range of $[-1, 1]$, $\varepsilon=1$ works well as the margin for the hinge loss function.

\subsubsection{\label{sec:modelthree}\ModelThree}
The third architecture is based on a pair-wise scenario during both training and inference (Figure~\ref{fig:m3}). This model learns a \emph{ranking function} $\mathcal{R}(q, d_1, d_2; \theta)$ which predicts the probability of document $d_1$ to be ranked higher than $d_2$ given $q$.
Similar to the \modeltwo, each training sample has five elements: $x = (q,d_1, d_2, s_{q,d_1}, s_{q,d_1})$.
For a given batch of training samples, we define our loss function based on cross-entropy as follows:
\begin{align}
\mathcal{L}(b; \theta) = -\frac{1}{|b|}
\sum_{i=1}^{|b|} &
P_{\{q,d_1,d_2\}_i} \log(\mathcal{R}(\{q,d_1,d_2\}_i; \theta)) \\
&
+ (1- P_{\{q,d_1,d_2\}_i})\log(1- \mathcal{R}(\{q,d_1,d_2\}_i; \theta)) \nonumber
\end{align}
where $P_{\{q,d_1,d_2\}_i}$ is the probability of document $d_1$ being ranked higher than $d_2$, based on the scores obtained from training sample $x_i$:
\begin{equation}
P_{\{q,d_1,d_2\}_i} = \frac{s_{\{q,d_1\}_i}}{s_{\{q,d_1\}_i} + s_{\{q,d_2\}_i}}
\end{equation}

A similar loss function has been previously used in RankNet~\citep{Burges:2005}. It is notable that at inference time, we need a scalar score for each document. Therefore, we need to turn the model's pair-wise predictions into a score per document. To do so, for each document, we calculate the average of predictions against all other candidate documents, which has $O(n^2)$ time complexity and is not practical in real-world applications. There are some approximations could be applicable to decrease the time complexity at inference time~\citep{Wauthier:2013}.

\medskip
As shown in Figure~\ref{fig:ranking-arch}, all the described ranking architectures share a neural network module. In all these models, we opted for a simple feed-forward neural network which is composed of: input layer $z_0$, $l-1$ hidden layers, and the output layer $z_l$. The input layer $z_0$ provides a mapping $\psi$ to encode the input query and document(s) into a fixed-length vector.
The exact specification of the input representation feature function $\psi$ is given in the next subsection. 
Each hidden layer $z_i$ is a fully-connected layer that computes the following transformation:
\begin{equation}
    z_i = \alpha(W_i.z_{i-1} + b_i); ~ 1<i<l-1,
\end{equation}
where $W_i$ and $b_i$ respectively denote the weight matrix and the bias term corresponding to the $i^{th}$ hidden layer, and $\alpha(.)$ is the activation function. We use the rectifier linear unit $\textit{ReLU}(x) = \max(0, x)$ as the activation function, which is a common choice in the deep learning literature~\citep{Lecun:2015}. 
The output layer $z_l$ is a fully-connected layer with a single continuous output. The activation function for the output layer depends on the ranking architecture that we use. For the \modelone architecture, we empirically found that a linear activation function works best, while $tanh$ and the sigmoid functions are used for the \modeltwo and \modelthree respectively.

Furthermore, to prevent feature co-adaptation, we use dropout as the regularization technique in all the models. Dropout sets a portion of hidden units to zero during the forward phase when computing the activations which prevents overfitting.


\subsection{Representing Inputs}
\label{sec:feedings}
We explore three definitions of the input layer representation $z_0$ captured by a feature function $\psi$ that maps the input into a fixed-size vector which is further fed into the fully connected layers: 
(i) a conventional dense feature vector representation that contains various statistics describing the input query-document pair, 
(ii) a sparse vector containing bag-of-words representation, and 
(iii) bag-of-embeddings averaged with learned weights. 
These input representations define how much capacity is given to the network to extract discriminative signal from the training data and thus result in different generalization behavior of the networks. 
It is noteworthy that input representation of the networks in the \modelone and \modeltwo is defined for a pair of the query and the document, while the network in the \modelthree needs to be fed by a triple of the query, the first document, and the second document.

\subsubsection{\Feedone (\fone)} 
In this setting, we build a dense feature vector composed of features used by traditional IR methods, e.g., BM25. The goal here is to let the network fit the function described by the BM25 formula when it receives exactly the same inputs. 
In more detail, our input vector is a concatenation ($||$) of the following inputs: total number of documents in the collection (i.e., $N$), average length of documents in the collection (i.e., $avg(l_d)_D$), document length (i.e., $l_d$), frequency of each query term $t_i$ in the document (i.e., $tf(t_i, d)$), and document frequency of each query term (i.e., $df(t_i)$). Therefore, for the point-wise setting, we have the following input vector:
\begin{equation}
\psi(q, d) = [N || avg(l_d)_D || l_d || \{df(t_i) || tf(t_i,d)\}_{1 \leq i \leq k}],
\end{equation}
where $k$ is set to a fixed value ($5$ in our experiments). 
We truncate longer queries and do zero padding for shorter queries. 
For the networks in the \modelthree, we consider a similar function with additional elements: the length of the second document and the frequency of query terms in the second document.


\subsubsection{\Feedtwo (\ftwo)} 
Next, we move away from a fully featurized representation that contains only aggregated statistics and let the network performs feature extraction for us. In particular, we build a bag-of-words representation by extracting term frequency vectors of query ($tfv_q$), document ($tfv_d$), and the collection ($tfv_c$) and feed the network with concatenation of these three vectors. For the point-wise setting, we have the following input vector:
\begin{equation}
\psi(q, d) = [tfv_c || tfv_q || tfv_d]
\end{equation}
For the network in \modelthree, we have a similar input vector with both $tfv_{d_1}$ and $tfv_{d_2}$. Hence, the size of the input layer is $3 \times vocab~size$ in the point-wise setting, and $4 \times vocab~size$ in the pair-wise setting. 
%In this paradigm, we have no document-level statistic, e.g., document frequency, although collection term frequency would be considered as a clue for term weighting. 

\subsubsection{\label{sec:feedthree}\Feedthree (\fthree)}
The major weakness of the previous input representation is that words are treated as discrete units, hence prohibiting the network from performing soft matching between semantically similar words in queries and documents. In this input representation paradigm, we rely on word embeddings to obtain more powerful representations of queries and documents that could bridge the lexical chasm. % of \fthree.
The representation function $\psi$ consists of three components: an embedding function $\mathcal{E}: \mathcal{V} \rightarrow \mathbb{R}^{m}$ (where $\mathcal{V}$ denotes the vocabulary set and $m$ is the embedding dimension), a weighting function $\mathcal{W}: \mathcal{V} \rightarrow \mathbb{R}$, and a compositionality function $\odot: (\mathbb{R}^{m}, \mathbb{R})^n \rightarrow \mathbb{R}^{m}$. More formally, the function $\psi$ for the point-wise setting is defined as:
\begin{equation}
\psi(q, d) = [\odot_{i=1}^{|q|}(\mathcal{E}(t_i^q), \mathcal{W}(t_i^q)) || \odot_{i=1}^{|d|} (\mathcal{E}(t_i^d), \mathcal{W}(t_i^d))],
\end{equation}
where $t_i^q$ and $t_i^d$ denote the $i^{th}$ term in query $q$ and document $d$, respectively. 
For the network of the \modelthree, another similar term is concatenated with the above vector for the second document. The embedding function $\mathcal{E}$ transforms each term to a dense $m$-dimensional float vector as its representation, which is learned during the training phase. The weighting function $\mathcal{W}$ assigns a weight to each term in the vocabulary set, which is supposed to learn term global importance for the retrieval task. The compositionality function $\odot$ projects a set of $n$ embedding and weighting pairs to an $m$-dimensional representation, independent from the value of $n$. The compositionality function is given by:
\begin{equation}
\odot_{i=1}^n(\mathcal{E}(t_i), \mathcal{W}(t_i)) = \sum_{i=1}^n \widehat{\mathcal{W}}(t_i)\cdot \mathcal{E}(t_i),
\end{equation}
which is the weighted element-wise sum of the terms' embedding vectors. $\widehat{\mathcal{W}}$ is the normalized weight that is learned for each term, given as follows:
\begin{equation}
\widehat{\mathcal{W}}(t_i) = \frac{\exp(\mathcal{W}(t_i))}{\sum_{j=1}^n{ \exp(\mathcal{W}(t_j))}}
\end{equation}

\medskip
All combinations of different ranking architectures and different input representations presented in this section can be considered for developing ranking models.

\subsection{Training Neural Rankers with Weak Supervision}
\label{sec:neuralranking_expermients}
In this section,  we discuss the effectiveness of our neural rankers with different learning objectives (Section~\ref{sec:models}) and different input representations (Section~\ref{sec:feedings}), when they are trained with weakly supervised signals.

In the following, we first describe the train and evaluation data, metrics we report, and detailed experimental setup. Then we discuss the results.

\input{03-part-02/chapter-04/figs_and_tables/table_data.tex}
\subsubsection{Collections}
\label{sec:collections}
In our experiments, we used two standard TREC collections: The first collection (called \emph{Robust04}) consists of over 500k news articles from different news agencies, that is available in TREC Disks 4 and 5 (excluding Congressional Records). This collection, which was used in TREC Robust Track 2004, is considered as a homogeneous collection, because of the nature and the quality of documents. The second collection (called \emph{ClueWeb}) that we used is ClueWeb09 Category B, a large-scale web collection with over 50 million English documents, which is considered as a heterogeneous collection. This collection has been used in TREC Web Track, for several years. In our experiments with this collection, we filtered out the spam documents using the Waterloo spam scorer\footnote{\url{http://plg.uwaterloo.ca/~gvcormac/clueweb09spam/}}~\citep{Cormack:2011} with the default threshold $70\%$. The statistics of these collections are reported in Table~\ref{tab:data}. 

\subsubsection{Training query set}
\label{sec:query_set}
To train our neural ranking models, we used the unique queries (only the query string) appearing in the AOL query logs~\citep{Pass:2006}. This query set contains web queries initiated by real users in the AOL search engine that were sampled from a three-month period from March 1, 2006 to May 31, 2006. We filtered out a large volume of navigational queries containing URL substrings (``http'', ``www.'', ``.com'', ``.net'', ``.org'', ``.edu''). We also removed all non-alphanumeric characters from the queries. We made sure that no queries from the training set appear in our evaluation sets. For each dataset, we took queries that have at least ten hits in the target corpus using the pseudo-labeler method. Applying all these processes, we ended up with 6.15 million queries for the Robust04 dataset and 6.87 million queries for the ClueWeb dataset. 
In our experiments, we randomly selected $80\%$ of the training queries as training set and the remaining $20\%$ of the queries were chosen as the validation set for hyper-parameter tuning. As the ``pseudo-labeler'' in our training data, we have used BM25 to score/rank documents in the collections given the queries in the training query set.

\subsubsection{Evaluation query sets} 
We use the following query sets for evaluation that contain human-labeled judgments: a set of 250 queries (TREC topics 301--450 and 601--700) for the Robust04 collection that were previously used in TREC Robust Track 2004. A set of 200 queries (topics 1-200) were used for the experiments on the ClueWeb collection. These queries were used in TREC Web Track 2009--2012. We only used the title of topics as queries.

\subsubsection{Evaluation Metrics}
To evaluate retrieval effectiveness, we report three standard evaluation metrics: mean average precision (MAP) of the top-ranked $1,000$ documents, precision of the top-$20$ retrieved documents (P@20), and normalized discounted cumulative gain (nDCG)~\citep{Jarvelin:2002} calculated for the top-$20$ retrieved documents (nDCG@20). Statistically significant differences of MAP, P@20, and nDCG@20 values are determined using the two-tailed paired t-test with $p\_value<0.05$, with Bonferroni correction.


\subsubsection{Experimental Setup}
All models described in Section~\ref{sec:models} are implemented using TensorFlow~\citep{tang2016:tflearn,tensorflow2015-whitepaper}.
In all experiments, the parameters of the network are optimized employing the Adam optimizer~\citep{Kingma:2014} and using the computed gradient of the loss to perform the back-propagation algorithm.
All model hyper-parameters were tuned on the respective validation set using batched GP bandits with an expected improvement acquisition function~\citep{Desautels:2014}. 
For each model, the size of hidden layers and the number of hidden layers were selected from $[16, 32, 64, 128, 256, 512, 1024]$ and $[1, 2, 3, 4]$, respectively. The initial learning rate and the dropout parameter were selected from $[1\times 10^{-3}, 5\times 10^{-4}, 1\times 10^{-4}, 5\times 10^{-5}, 1\times 10^{-5}]$ and $[0.0, 0.1, 0.2, 0.5]$, respectively. For models with \feedthree, we considered embedding sizes of $[100, 300, 500, 1000]$. As the training data, we take the top-$1,000$ retrieved documents for each query from training query set $Q$, to prepare the training data. In total, we have $|Q|\times 1000$ ($\sim6\times 10^{10}$ examples in our data) point-wise example and $\sim|Q|\times 1000^2$ ($\sim6\times 10^{13}$ examples in our data) pair-wise examples. The batch size in our experiments was selected from  $[128, 256, 512]$.
%
At inference time, for each query, we take the top-$2,000$ retrieved documents using BM25 as candidate documents and re-rank them by the trained models. In our experiments, we use the Indri\footnote{\url{https://www.lemurproject.org/indri.php}} implementation of BM25 with the default parameters (i.e., $k_1 = 1.2$, $b = 0.75$, and $k_3 = 1000$).


\bigskip
Given the setup we explained above, we train and evaluate our neural rankers to address our second research question in this chapter:
\resq{c4.2}

We attempt to break down our experiments and analyses to different parts addressing several subquestions, and provide empirical answers along with the intuition and analysis behind each question:

% !TEX root = main.tex
\begin{table*}[tbp]
\centering
\caption{\label{tbl_main}Performance of the different models on different datasets. \rlap\smash{\ps} or \rlap\smash{\ns} indicates that the improvements or degradations with respect to BM25 are statistically significant, at the 0.05 level using the paired two-tailed t-test.}
%\begin{adjustbox}{max width=\columnwidth}
%\begin{tabular}{@{}l@{~~~}c@{~~}c@{~~}c@{~~~}c@{~~}c@{~~}c@{}}
\begin{adjustbox}{max width=\textwidth}
\begin{tabular}{l c c c c c c}
\toprule
\multirow{2}{*}{\textbf{Method}} &
%\multicolumn{1}{l}{\textbf{Method}} & 
\multicolumn{3}{c}{\textbf{Robust04}} & \multicolumn{3}{c}{\textbf{ClueWeb}}
\\ \cmidrule(lr){2-4} \cmidrule(lr){5-7}
& \textit{MAP} & \textit{P@20} & \textit{nDCG@20}  & \textit{MAP} & \textit{P@20} & \textit{nDCG@20}
\\ \midrule
\textbf{BM25} 
& 0.2503\rlap\smash{\fs} & 0.3569\rlap\smash{\fs} & 0.4102\rlap\smash{\fs}  
& 0.1021\rlap\smash{\fs} & 0.2418\rlap\smash{\fs} & 0.2070\rlap\smash{\fs}
\\ \midrule
\textbf{\mone + \fone} 
& 0.1961\rlap\smash{\ns} & 0.2787\rlap\smash{\ns} & 0.3260\rlap\smash{\ns} 
& 0.0689\rlap\smash{\ns} & 0.1518\rlap\smash{\ns} & 0.1430\rlap\smash{\ns}
\\ 
\textbf{\mone + \ftwo} 
& 0.2141\rlap\smash{\ns} & 0.3180\rlap\smash{\ns} & 0.3604\rlap\smash{\ns} 
& 0.0701\rlap\smash{\ns} & 0.1889\rlap\smash{\ns} & 0.1495\rlap\smash{\ns}
\\ 
\textbf{\mone + \fthree} 
& 0.2423\rlap\smash{\ns} & 0.3501\rlap\smash{\fs} & 0.3999\rlap\smash{\fs} 
& 0.1002\rlap\smash{\fs} & 0.2513\rlap\smash{\fs} & 0.2130\rlap\smash{\fs}
\\ \midrule
\textbf{\mtwo + \fone} 
& 0.1940\rlap\smash{\ns} & 0.2830\rlap\smash{\ns} & 0.3317\rlap\smash{\ns} 
& 0.0622\rlap\smash{\ns} & 0.1516\rlap\smash{\ns} & 0.1383\rlap\smash{\ns}
\\ 
\textbf{\mtwo + \ftwo} 
& 0.2213\rlap\smash{\ns} & 0.3216\rlap\smash{\ns} & 0.3628\rlap\smash{\ns} 
& 0.0776\rlap\smash{\ns} & 0.1989\rlap\smash{\ns} & 0.1816\rlap\smash{\ns}
\\ 
\textbf{\mtwo + \fthree} 
& \textbf{0.2811}\rlap\smash{\ps} & \textbf{0.3773}\rlap\smash{\ps} & \textbf{0.4302}\rlap\smash{\ps} 
& \textbf{0.1306}\rlap\smash{\ps} & \textbf{0.2839}\rlap\smash{\ps} & \textbf{0.2216}\rlap\smash{\ps}
\\ \midrule
\textbf{\mthree + \fone} 
& 0.2192\rlap\smash{\ns} & 0.2966\rlap\smash{\ns} & 0.3278\rlap\smash{\ns} 
& 0.0702\rlap\smash{\ns} & 0.1711\rlap\smash{\ns} & 0.1506\rlap\smash{\ns}
\\ 
\textbf{\mthree + \ftwo} 
& 0.2246\rlap\smash{\ns} & 0.3250\rlap\smash{\ns} & 0.3763\rlap\smash{\ns} 
& 0.0894\rlap\smash{\ns} & 0.2109\rlap\smash{\ns} & 0.1916\rlap\smash{\fs}
\\ 
\textbf{\mthree + \fthree} 
& \textbf{0.2837}\rlap\smash{\ps} & \textbf{0.3802}\rlap\smash{\ps} & \textbf{0.4389}\rlap\smash{\ps} 
& \textbf{0.1387}\rlap\smash{\ps} & \textbf{0.2967}\rlap\smash{\ps} & \textbf{0.2330}\rlap\smash{\ps}
\\ \bottomrule
\end{tabular}
\end{adjustbox}
\end{table*} 
\subsubsection{How do the neural models with different training objectives and input representations compare?}
%
Table~\ref{tbl_main} presents the performance of all model combinations.
Interestingly, combinations of the \modeltwo and the \modelthree with \feedthree outperform BM25 by significant margins in both collections. For instance, the \modelthree with \feedthree that shows the best performance among the other methods, surprisingly, improves BM25 by over $13\%$ and $35\%$ in Robust04 and ClueWeb collections respectively, in terms of MAP. Similar improvements can be observed for the other evaluation metrics.

Regarding the modeling architecture, in the \modeltwo and the \modelthree, compared to the \modelone, we define objective functions that target to learn ranking instead of scoring. This is particularly important in weak supervision, as the scores are imperfect values---using the ranking objective alleviates this issue by forcing the model to learn a preference function rather than reproduce absolute scores.
%
In other words, using the ranking objective instead of learning to predict calibrated scores allows the \modeltwo and the \modelthree to learn to distinguish between examples whose scores are close. This way, some small amount of noise, which is a common problem in weak supervision, would not perturb the ranking as easily~\citep{Zamani:2018:ictir}.

Regarding the input representations, \feedthree leads to better performance compared to the other ones in all models.
Using \feedthree not only provides the network with more information, but also lets the network to learn proper representation capturing the needed elements for the next layers with a better understanding of the interactions between query and documents. 
Providing the network with already engineered features would block it from going beyond the weak supervision signal and limit the ability of the models to learn latent features that are unattainable through feature engineering. 

Note that although the \modelthree is more precise in terms of MAP, the \modeltwo is much faster in the inference time ($O(n)$ compared to $O(n^2)$), which is a desirable property in real-life applications.

% !TEX root = main.tex
\begin{figure*}[t]
    \centering
    \begin{subfigure}[t]{0.33\textwidth}
        \centering
        \includegraphics[height=3.6cm]{03-part-02/chapter-04/figs_and_tables/loss_step_plots/step_m1f1.png}
        \caption{\label{fig:m1f1}\mone-\fone}
    \end{subfigure}%
    ~
    \begin{subfigure}[t]{0.33\textwidth}
        \centering
        \includegraphics[height=3.6cm]{03-part-02/chapter-04/figs_and_tables/loss_step_plots/step_m1f2.png}
        \caption{\label{fig:m1f2}\mone-\ftwo}
    \end{subfigure}%
    ~
    \begin{subfigure}[t]{0.33\textwidth}
        \centering
        \includegraphics[height=3.6cm]{03-part-02/chapter-04/figs_and_tables/loss_step_plots/step_m1f3.png}
        \caption{\label{fig:m1f3}\mone-\fthree}
    \end{subfigure}%
    \\
    \begin{subfigure}[t]{0.33\textwidth}
        \centering
        \includegraphics[height=3.6cm]{03-part-02/chapter-04/figs_and_tables/loss_step_plots/step_m2f1.png}
        \caption{\label{fig:m2f1}\mtwo-\fone}
    \end{subfigure}%
    ~
    \begin{subfigure}[t]{0.33\textwidth}
        \centering
        \includegraphics[height=3.6cm]{03-part-02/chapter-04/figs_and_tables/loss_step_plots/step_m2f2.png}
        \caption{\label{fig:m2f2}\mtwo-\ftwo}
    \end{subfigure}%
    ~
    \begin{subfigure}[t]{0.33\textwidth}
        \centering
        \includegraphics[height=3.6cm]{03-part-02/chapter-04/figs_and_tables/loss_step_plots/step_m2f3.png}
        \caption{\label{fig:m2f3}\mtwo-\fthree}
    \end{subfigure}%
        \\
    \begin{subfigure}[t]{0.33\textwidth}
        \centering
        \includegraphics[height=3.6cm]{03-part-02/chapter-04/figs_and_tables/loss_step_plots/step_m3f1.png}
        \caption{\label{fig:m3f1}\mthree-\fone}
    \end{subfigure}%
    ~
    \begin{subfigure}[t]{0.33\textwidth}
        \centering
        \includegraphics[height=3.6cm]{03-part-02/chapter-04/figs_and_tables/loss_step_plots/step_m3f2.png}
        \caption{\label{fig:m3f2}\mthree-\ftwo}
    \end{subfigure}%
    ~
    \begin{subfigure}[t]{0.33\textwidth}
        \centering
        \includegraphics[height=3.6cm]{03-part-02/chapter-04/figs_and_tables/loss_step_plots/step_m3f3.png}
        \caption{\label{fig:m3f3}\mthree-\fthree}
    \end{subfigure}%
    \caption{Training and validation loss curves for all combinations of different ranking architectures and input representations.}
    \label{fig:step-loss}
\end{figure*}

\subsubsection{Why do \feedone and \feedtwo fail to replicate the performance of BM25?}
%
Although neural networks are capable of approximating arbitrarily complex non-linear functions, we observe that the models with \feedone fail to replicate the BM25 performance, while they are given the same feature inputs as the BM25 components (e.g., TF, IDF, average document length, etc). To ensure that the training converges and there is no overfitting, we have looked into the training and validation loss values of different models during the training time. Figure~\ref{fig:step-loss} illustrates the loss curves for the training and validation sets per training step for different models.
%
As shown, in models with \feedone, the training losses drop quickly to values close to zero while this is not the case for the validation losses, which is an indicator of over-fitting on the training data. 
Although we have tried different regularization techniques, like $l_2$-regularization and dropout with various parameters, there is less chance for generalization when the networks are fed with the fully featurized input. Note that over-fitting would lead to poor performance, especially in weak supervision scenarios as the network learns to model imperfections from weak annotations. 
%
This phenomenon is also the case for models with the \feedtwo, but with less impact. However, in the models with the \feedthree, the networks do not overfit, which helps it to go beyond the weak supervision signals in the training data.

\input{03-part-02/chapter-04/figs_and_tables/plot_models_proximity.tex}
\subsubsection{How are the models related?}
%
To better understand the relationship of different neural models described above, we compare their performance across the query dimension following the approach in~\citep{Mitra:2016}. 
We assume that similar models should perform similarly for the same queries. Hence, we represent each model by a vector, called the performance vector, whose elements correspond to per query performance of the model, in terms of nDCG@20. The closer the performance vectors are, the more similar the models are in terms of the query by query performance. For the sake of visualization, we reduce the vectors dimension by projecting them to a two-dimensional space, using t-Distributed Stochastic Neighbor Embedding (t-SNE)\footnote{\url{https://lvdmaaten.github.io/tsne/}}.

Figure~\ref{fig:modelproximity} illustrates the proximity of different models in the Robust04 collection. Based on this plot, models with similar input representations (same color) have quite close performance vectors, which means that they perform similarly for the same queries. This is not necessarily the case for models with similar architecture (same shape). 
This suggests that the amount and the way that we provide information to the networks are the key factors in the ranking performance. 

We also observe that the \modelone with \feedone is the closest to BM25 which is expected. 
It is also interesting that models with \feedthree are placed far away from other models which shows they perform differently compared to the other input representations.


\subsubsection{How meaningful are the compositionality weights learned in the \feedthree?}
%
In this experiment, we focus on the best performing combination, i.e., the \modelthree with \feedthree. To analyze what the network learns, we look into the weights $\mathcal{W}$ (see Section~\ref{sec:feedthree}) learned by the network. Note that the weighting function $\mathcal{W}$ learns a global weight for each vocabulary term. We notice that in both collections there is a strong linear correlation between the learned weights and the inverse document
%
\begin{figure}[!t]%
    \centering
    \begin{subfigure}[t]{0.45\textwidth}
        \centering
        \includegraphics[height=5cm]{03-part-02/chapter-04/figs_and_tables/plot_composionality_idf_scatter_robust.png}
        \caption{\label{fig:scatter_r}Robust04}{\scriptsize{(Pearson Correlation: 0.8243)}}
    \end{subfigure}%
    ~
    \begin{subfigure}[t]{0.45\textwidth}
        \centering
        \includegraphics[height=5cm]{03-part-02/chapter-04/figs_and_tables/plot_composionality_idf_scatter_clueweb.png}
        \caption{\label{fig:scatter_c}ClueWeb}{\scriptsize{(Pearson Correlation: 0.7014)}}
    \end{subfigure}%
    \caption{\label{fig:scatter}Strong linear correlation between weight learned by the compositionality function in the \feedthree and inverse document frequency.}
\end{figure}
%
Figure~\ref{fig:scatter} illustrates the scatter plots of the learned weight for each vocabulary term and its IDF, in both collections.
This is an interesting observation as we do not provide any global corpus information to the network in training and the network is able to infer such global information by only observing individual training samples.
% This demonstrates the ability of neural networks to automatically extract meaningful features for the task.

% !TEX root = main.tex
\begin{table*}[tbp]
\centering
\caption{\label{tbl_res_m3f3_em}Performance of the \modelthree with variants of the \feedthree on different datasets. \rlap\smash{\ps} indicates that the improvements over all other models are statistically significant, at the 0.05 level using the paired two-tailed t-test, with Bonferroni correction.}
%\begin{adjustbox}{max width=\columnwidth}
%\begin{tabular}{@{}l@{~~~}c@{~~}c@{~~}c@{~~~}c@{~~}c@{~~}c@{}}
\begin{adjustbox}{max width=\textwidth}
\begin{tabular}{l c c c c c c}
\toprule
\multirow{2}{*}{\textbf{Embedding type}} &
%\multicolumn{1}{l}{\textbf{Method}} & 
\multicolumn{3}{c}{\textbf{Robust04}} & \multicolumn{3}{c}{\textbf{ClueWeb}}
\\ \cmidrule(lr){2-4} \cmidrule(lr){5-7}
& \textit{MAP} & \textit{P@20} & \textit{nDCG@20}  & \textit{MAP} & \textit{P@20} & \textit{nDCG@20}
\\ \midrule
\textbf{Pretrained (external) + Uniform weighting} 
& 0.1656\rlap\smash{\fs} & 0.2543\rlap\smash{\fs} & 0.3017\rlap\smash{\fs} 
& 0.0612\rlap\smash{\fs} & 0.1300\rlap\smash{\fs} & 0.1401\rlap\smash{\fs}
\\ 
\textbf{Pretrained (external) + IDF weighting} 
& 0.1711\rlap\smash{\fs} & 0.2755\rlap\smash{\fs} & 0.3104\rlap\smash{\fs} 
& 0.0712\rlap\smash{\fs} & 0.1346\rlap\smash{\fs} & 0.1469\rlap\smash{\fs}
\\ 
\textbf{Pretrained (external) + Weight learning} 
& 0.1880\rlap\smash{\fs} & 0.2890\rlap\smash{\fs} & 0.3413\rlap\smash{\fs} 
& 0.0756\rlap\smash{\fs} & 0.1344\rlap\smash{\fs} & 0.1583\rlap\smash{\fs}
\\ 
\textbf{Pretrained (target) + Uniform weighting} 
& 0.1217\rlap\smash{\fs} & 0.2009\rlap\smash{\fs} & 0.2791\rlap\smash{\fs} 
& 0.0679\rlap\smash{\fs} & 0.1331\rlap\smash{\fs} & 0.1587\rlap\smash{\fs}
\\ 
\textbf{Pretrained (target) + IDF weighting} 
& 0.1402\rlap\smash{\fs} & 0.2230\rlap\smash{\fs} & 0.2876\rlap\smash{\fs} 
& 0.0779\rlap\smash{\fs} & 0.1674\rlap\smash{\fs} & 0.1540\rlap\smash{\fs}
\\ 
\textbf{Pretrained (target) + Weight learning} 
& 0.1477\rlap\smash{\fs} & 0.2266\rlap\smash{\fs} & 0.2804\rlap\smash{\fs} 
& 0.0816\rlap\smash{\fs} & 0.1729\rlap\smash{\fs} & 0.1608\rlap\smash{\fs}
\\
\textbf{Learned + Uniform weighting} 
& 0.2612\rlap\smash{\fs} & 0.3602\rlap\smash{\fs} & 0.4180\rlap\smash{\fs}
& 0.0912\rlap\smash{\fs} & 0.2216\rlap\smash{\fs} & 0.1841\rlap\smash{\fs}
\\
\textbf{Learned + IDF weighting} 
& 0.2676\rlap\smash{\fs} & 0.3619\rlap\smash{\fs} & 0.4200\rlap\smash{\fs} 
& 0.1032\rlap\smash{\fs} & 0.2419\rlap\smash{\fs} & 0.1922\rlap\smash{\fs}
\\ 
\textbf{Learned + Weight learning} 
& \textbf{0.2837}\rlap\smash{\ps} & \textbf{0.3802}\rlap\smash{\ps} & \textbf{0.4389}\rlap\smash{\ps}
& \textbf{0.1387}\rlap\smash{\ps} & \textbf{0.2967}\rlap\smash{\ps} & \textbf{0.2330}\rlap\smash{\ps}
\\ \bottomrule
\end{tabular}
\end{adjustbox}
\end{table*}

\subsubsection{How well do other alternatives for the embedding and weighting functions in the \feedthree perform?}
Considering \feedthree as the input representation, we have examined different alternatives for the embedding function $\mathcal{E}$: (1) employing pre-trained word embeddings learned from an external corpus (we used Google News), (2) employing pre-trained word embeddings learned from the target corpus (using the skip-gram model \cite{Mikolov:2013}), and (3) learning embeddings during the network training as explained in Section~\ref{sec:feedthree}. 
Furthermore, for the compositionality function $\odot$, we tried different alternatives: (1) uniform weighting (simple averaging which is a common approach in compositionality function), (2) using IDF as fixed weights instead of learning the weighting function $\mathcal{W}$, and (3) learning weights during the training as described in Section~\ref{sec:feedthree}.

Table~\ref{tbl_res_m3f3_em} presents the performance of all these combinations on both collections. 
We note that learning both embedding and weighting functions leads to the highest performance in both collections. These improvements are statistically significant.
%
According to the results, regardless of the weighting approach, learning embeddings during training outperforms the models with fixed pre-trained embeddings.
%
This supports the hypothesis that with the \feedthree the neural networks learn an embedding that is based on the interactions of query and documents that tends to be tuned better to the corresponding ranking task.
%
Also, regardless of the embedding method, learning weights helps models to get better performance compared to the fixed weightings, with either IDF or uniform weights. 
%
Although weight learning can significantly affect the performance, it has less impact than learning embeddings.

Note that in the models with pre-trained word embeddings, employing word embeddings trained on the target collection outperforms those trained on the external corpus in the ClueWeb collection; while this is not the case for the Robust04 collection. The reason could be related to the collection size, since the ClueWeb is approximately $100$ times larger than the Robust04.

\begin{figure}[t]
    \centering
    \begin{subfigure}[t]{0.45\textwidth}
        \centering
        \includegraphics[height=5cm]{03-part-02/chapter-04/figs_and_tables/plot_with_pretrained_emb_robust.png}
        \caption{\label{fig:embedding_r}Robust04}
    \end{subfigure}%
    ~
    \begin{subfigure}[t]{0.45\textwidth}
        \centering
        \includegraphics[height=5cm]{03-part-02/chapter-04/figs_and_tables/plot_with_pretrained_emb_clueweb.png}
        \caption{\label{fig:embedding_c}ClueWeb}
    \end{subfigure}%
    \caption{\label{fig:embedding}Performance of the \modelthree with learned embedding, pre-trained embedding, and learned embedding with pre-trained embedding as initialization, with respect to different amount of training data.}
\end{figure}

In addition to the aforementioned experiments, we have also tried initializing the embedding matrix with a pre-trained word embedding trained on the Google News corpus, instead of random initialization.
%
Figure~\ref{fig:embedding} presents the learning curve of the models. According to this figure, the model initialized by a pre-trained embedding performs better than random initialization when a limited amount of training data is available. 
%
When enough training data is fed to the network, initializing with pre-trained embedding and random values converge to the same performance.
An interesting observation here is that in both collections, these two initializations converge when the models exceed the performance of the weak supervision source, which is BM25 in our experiments. 
This suggests that the convergence occurs when accurate representations are learned by the networks, regardless of the initialization.

%In other words, after the network sees enough amount of train data to go beyond the supervision signal, it does not matter that with which initialization the model has started to train.
\input{03-part-02/chapter-04/figs_and_tables/table_svm_results.tex}
\subsubsection{Are deep neural networks a good choice for learning to rank with weak supervision?}
%
To see if there is a real benefit from using a non-linear neural network in different settings, we examined RankSVM~\citep{Joachims:2002} as a strong-performing pair-wise learning to rank method with the linear kernel that is fed with different inputs: \feedone, \feedtwo, and \feedthree. Considering that off-the-shelf RankSVM is not able to learn embedding representations during training, for \feedthree, instead of learning embeddings we use a pre-trained embedding matrix trained on Google News and fixed IDF weights. 

The results are reported in Table~\ref{tbl_svm}. As BM25 is not a linear function, RankSVM with the linear kernel is not able to completely approximate it. However, surprisingly, for both \feedone and \feedtwo, RankSVM works as well as neural networks (see Table~\ref{tbl_main}). 
%
Also, compared to the corresponding experiment in Table~\ref{tbl_res_m3f3_em}, the performance of the neural network with an external pre-trained embedding and IDF weighting is not considerably better than RankSVM. 
This shows that having non-linearity in neural networks does not help that much when we do not have representation learning as part of the model.
%
Note that all of these results are still lower than BM25, which shows that they are not good at learning from weak supervision signals for ranking. 
%

We have also examined the \modelone with a network with a single linear hidden layer, with the \feedthree, which is equivalent to a linear regression model with the ability of representation learning. 
Comparing the results of this experiment with \mone-\fthree in Table~\ref{tbl_main}, we can see that with a single-linear network we are not able to achieve a performance that is as good as a deep neural network with non-linearity.
%
This shows that the most important superiority of deep neural networks over other machine learning methods is their ability to learn an effective representation and take all the interactions between query and document(s) into consideration for approximating an effective ranking/scoring function. 
This can be achieved when we have a deep enough network with non-linear activations.

%\alexi{Mosi@: we should also include supervised learning experiments on smaller subsets of human judgments, if possible. The idea is that in these scenarios where labeled data is scarce, weak supervision is really useful. Having plots, where we use various percentages of labeled examples and observing how weak supervision is increasingly important as the number of labeled examples becomes smaller.}
\input{03-part-02/chapter-04/figs_and_tables/table_semisup_results.tex}
\subsubsection{How useful is learning with weak supervision for supervised ranking?}
%
In this set of experiments, we investigate whether employing weak supervision as a pre-training step helps to improve the performance of supervised ranking, when a small amount of training data is available. Table~\ref{tbl_semisup} shows the performance of the \modelthree with the \feedthree in three situations: (1) when it is only trained on weakly supervised data (similar to the previous experiments), (2) when it is only trained on supervised data, i.e., relevance judgments, and (3) when the parameters of the network is pre-trained using the weakly supervised data and then fine-tuned using relevance judgments.
%
In all the supervised scenarios, we performed 5-fold cross-validation over the queries of each collection and in each step, we used the TREC relevance judgments of the training set as a supervised signal. For each query with $m$ relevant documents, we also randomly sampled $m$ non-relevant documents as negative samples. Binary labels are used in the experiments: $1$ for relevant documents and $0$ for non-relevant ones.

The results in Table~\ref{tbl_semisup} suggest that pre-training the network with a weak supervision signal, significantly improves the performance of supervised ranking.
%
The reason for the poor performance of the supervised model compared to the conventional learning to rank models is that the number of parameters is much larger, hence it needs much more data for training.

In situations when little supervised data is available, it is especially helpful to use unsupervised pre-training which acts as a network pre-conditioning that puts the parameter values in the appropriate range that renders the optimization process more effective for further supervised training~\citep{Rrhan:2010}.

With this experiment, we indicate that the idea of learning from weak supervision signals for neural ranking models, which is presented in this section, not only enables us to learn neural ranking models when no supervised signal is available, but also has substantial positive effects on the supervised ranking models with limited amount of training data. 

In the next section, we study how learning from weak/noisy label can benefit preserving privacy in machine learning, like in situations where additional noise is intentionally added to the training data to guarantee differential privacy.  
\section{Weakly Supervised Learning for Preserving Privacy}
Deep neural networks perform better as the training dataset grows bigger and becomes diverse and representative~\citep{sun2017revisiting} of all possible cases.  In many applications, the datasets that are big and diverse may contain sensitive information from users, for instance, medical histories of patients in a clinical trial, or search logs from users of a search engine.  
It has been shown that a trained model may inadvertently
and implicitly store some of its training data and we can retrieve some of the information about samples in the training data~\citep{Shokri:2015}, either directly by analyzing internal model parameters or indirectly by repeatedly querying the model as a black-box to gather data and do analysis on those data~\citep{Fredrikson:2015}. 

We need to design and use learning algorithms that protect the privacy of users, for instance, by guaranteeing that the output model generalizes away from the specifics of any
individual user. Recently, \citet{Papernot:2017} proposed Private Aggregation of Teacher Ensembles (PATE), a generally applicable approach to providing strong privacy guarantees for training data.  PATE uses a noisy aggregation of the signal that comes from multiple models trained with disjoint datasets, to train a new model that guarantees a certain level of differential privacy.

Almost all deferentially private algorithms add noise to introduce ambiguity. Hence, the training signals become less perfect and employing noise-robust models can support injecting noise, yielding strong privacy guarantees, while having a limited impact on accuracy.

Search and retrieval is one of the applications that needs special attention on preserving the privacy of users' data and many recent advances rely on sensitive and private data such as large-scale query logs, users’ search history, and location information~\citep{Yang:2017}. In this section, we focus on the task of ranking and assessing the relevance.
Here we seek an answer to the third research question of this chapter:
\resq{c4.3}

We present the results of a set of experiments that examine the performance of one of the neural ranking architectures proposed in Section~\ref{sec:weakly_supervised_neural_rankers} when it is employed in PATE, the privacy-preserving framework proposed by~\citet{Papernot:2017}, where the neural ranker is supposed to learn from signals with added noise. 
Since PATE is based on the knowledge distillation framework~\cite{Hinton:2015}, we first train a neural ranker in a mimic-learning setup where a student ranker is trained on the signals from a teacher ranker that is trained using labeled data. Then we use the full privacy preserving pipeline of PATE to train our neural ranker.
%
It is noteworthy that here, we mainly concern about the performance of a neural ranker, when it is employed in the PATE's setup and will not re-discuss the differential privacy of PATE, as this side of the discussion is presented thoroughly in the original paper~\citep{Papernot:2017}.

\subsection{Mimic Learning to Rank}
\label{sec:mimic_learning_to_rank}
Using machine learning-based approaches, sharing the trained model instead of the original data has turned out to be an option for transferring knowledge~\citep{Papernot:2017,Shokri:2015,Abadi:2016}. 
The idea of \emph{mimic learning} is to use a model that is trained based on the signals from the original training data to annotate a large set of unlabeled data and use these labels as training signals for training a new model. 
It has been shown, for many tasks in computer vision and natural language processing, that we can transfer knowledge this way and the newly trained models perform as well as the model trained on the original training data~\citep{Bucilua:2006,Hinton:2015,Romero:2014,Ba:2014}.

We follow the knowledge distillation approach~\cite{Hinton:2015} for training a neural ranker, where we have a teacher network that is trained using labeled data, and a student network that is trained using the signals from the teacher network on a set of unlabeled data. We have two sets of experiments, in the first one, we train the teacher model with full supervision, i.e., on the set of queries with judgments, using 5-fold cross-validation. 
In the second set of experiments, the set of queries with judgments is only used for evaluation and we train the teacher model using the weak supervision setup, i.e., pseudo labels as it is explained in Section~\ref{sec:pseudo_labeling}. 
As the test collection, we use Robust04, which has been introduced in Section~\ref{sec:collections}. In all experiments, we use a separate set of $3$ million queries from the AOL query log, preprocessed as explained in Section~\ref{sec:query_set}.

In all experiments, as the neural rankers, we use \modeltwo,\footnote{Although in Section~\ref{sec:neuralranking_expermients} we showed that the best performing model is the \modelthree, its improvement over the \modeltwo was not statistically significant. Compared to the \modelthree, the \modeltwo is much more efficient at inference time, as it operates in a point-wise setup, unlike the \modelthree which runs in a pair-wise mode. We decided to adapt the \modeltwo for all experiments in this section.} which has been described in Section~\ref{sec:modeltwo}, with \feedthree as the input representation, which has been explained in Section~\ref{sec:feedthree}.
The configuration of teacher and student networks is presented in Table~\ref{tbl:cfg}.
\begin{table}[t]
\centering
\caption{Teacher and student neural networks configurations.}
\begin{tabular}{lcrr} 
\toprule
\bf Parameter & \bf Teacher & \bf Student  \\
\midrule
\textbf{Number of hidden layers} & 3 & 3  \\
\textbf{Size of hidden layers} & 512 & 128 \\
\textbf{Initial learning rate} & 1E-3 & 1E-3 \\
\textbf{Dropout} & 0.2 & 0.1 \\
\textbf{Embedding size} & 500 & 300 \\
\textbf{Batch size} & 512 & 512  \\
\bottomrule
\end{tabular}
\label{tbl:cfg}
\end{table}

\begin{table}[t]
\centering
\caption{\label{tbl_res1}Performance of teacher and student models with different training strategies.}
\vspace{5pt}
\begin{adjustbox}{max width=\textwidth}
\begin{tabular}{l l c c c}
\toprule
\bf Training strategy & \bf model & \textbf{MAP} & \textbf{P@20} & \textbf{nDCG@20} 
\\ \midrule
\multirow{2}{*}{{\textbf{Full supervision}}} & {\textit{Teacher}} 
& 0.1814 & 0.2888 & 0.3419 
\\
& {\textit{Student}} 
& 0.2256 & 0.3111 & 0.3891 
\\ \midrule
\multirow{2}{*}{{\textbf{Weak supervision}}} & {\textit{Teacher}} 
& 0.2716 & 0.3664 & 0.4109 
\\ 
& {\textit{Student}} 
& 0.2701 & 0.3562 & 0.4145 
\\ \bottomrule
\end{tabular}
\end{adjustbox}
\end{table}

Results obtained from these experiments are summarized in Table~\ref{tbl_res1}. As the results suggest, using weak supervision to train the teacher model, the student model performs as good as the teacher model. In case of training the teacher with full supervision (labeled data from Robust04), as the original training data is small, the performance of the teacher model is rather low, which is mostly due to the fact that the big teacher model overfits on the train data and is not able to generalize well. 
However, due to the regularization effect of the knowledge distillation process, the student model, which is trained on the predictions by the teacher model significantly outperforms the teacher model~\citep{Hinton:2015,Romero:2014}.

\subsection{Privacy Preserving Neural Ranker}
In Section~\ref{sec:mimic_learning_to_rank}, we examined the idea of mimic learning to train a neural ranker regardless of the privacy concerns.
It has been shown that there is a risk of privacy problems, both where the adversary is just able to query the model, and where the model parameters are exposed to the adversaries inspection.
For instance, \citet{Fredrikson:2015} show that only by observing the prediction of the machine learning models they can already approximately reconstruct part of the training data (model-inversion attack). \citet{Shokri:2016} also demonstrate that it is possible to infer whether a specific training point is included in the model's training data by observing only the predictions of the model (membership inference attack).

In this section, we adapt the Private Aggregation of Teacher Ensembles (PATE)~\citep{Papernot:2017} to train a privacy-preserving neural ranking model.  PATE is based on the student-teacher framework~\citep{Hinton:2015}, where there are multiple teacher models trained on disjoint subsets of the data and a student model that learns to predict an output that is chosen by noisy voting among all of the teachers. The student model cannot directly access an individual teacher or the underlying data or parameters. They show that PATE improves privacy/utility trade-offs by achieving high accuracy while guaranteeing a certain level of differential privacy.

\begin{figure}[t]
    \centering
    \includegraphics[height=5.5cm]{03-part-02/chapter-04/figs_and_tables/fig_privacy_preserving_ranker_model.pdf}%
    \caption{\label{fig:pp_model} Privacy preserving annotator/model sharing, proposed by~\citet{Papernot:2017}.}
\end{figure}

The general schema of the PATE is illustrated in Figure~\ref{fig:pp_model}. First, the sensitive training data is divided into $n$ partitions. Then, on each partition, an independent neural network model is trained as a teacher.  Once all the teachers are trained, an aggregation step is done using majority voting to generate a single global prediction.  
Laplacian noise is injected into the output of the prediction of each teacher before aggregation. The introduction of this noise is what protects privacy because it obfuscates the vulnerable cases, where teachers disagree. 

The aggregated teacher can be considered as a deferentially private API to which we can submit the input and it then returns the privacy-preserving label. There are some circumstances where due to efficiency reasons the model needs to be deployed on the user device~\cite{Abadi:2016}. To be able to generate a shareable model where the privacy of the training data is preserved, \citet{Papernot:2017} train an additional model called the student model. The student model has access to unlabeled public data during training. The unlabeled public data is annotated using the aggregated teacher to transfer knowledge from teachers to student model in a privacy-preserving fashion. 
This way, if the adversary tries to recover the training data by inspecting the parameters of the student model, in the worst case, the public training instances with privacy-preserving labels from the aggregated teacher are going to be revealed.  The privacy guarantee of this approach is formally proved using the differential privacy framework.

As there is no publicly available large-scale data that we can use in our experiments as the initial sensitive labeled data, we use pseudo labels as it is explained in~\ref{sec:pseudo_labeling}.\footnote{Partitioning the fully supervised training data in our problem leads to very small training sets which are not big enough to train teacher networks.}
In our experiments, we split the training data into three partitions, each contains one million queries annotated by the BM25 method. We train three identical teacher models. Then, we use the aggregated noisy predictions from these teachers to train the student network using the knowledge distillation approach. Configurations of teacher and student networks are similar to the previous experiments, as they are presented in Table~\ref{tbl:cfg}.

We evaluate the performance in two situations: In the first one, the privacy parameter, which determines the amount of noise, is set to zero, and in the second one, the noise parameter is set to $0.05$, which guarantees a low privacy risk~\citep{Papernot:2017}.
%
We report the average performance of the teachers before noise, the performance of noisy and non-noisy aggregated teacher, and the performance of the student networks in two situations.  The results of these experiments are reported in Table~\ref{tbl_res2}.

\begin{table}[t]
\centering
\caption{\label{tbl_res2}Performance of the teachers (average) and student models with noisy and non-noisy aggregation.}
\vspace{5pt}
\begin{adjustbox}{max width=\textwidth}
\begin{tabular}{l c c c}
\toprule
 \textbf{Model} & \textbf{MAP} & \textbf{P@20} & \textbf{nDCG@20} 
\\ \midrule
{\textbf{Teachers (avg)}}
& 0.2566 & 0.3300 & 0.3836
\\ \midrule
\textbf{Non-noisy aggregated teacher} 
& 0.2380 & 0.3055 & 0.3702 
\\
\textbf{Student \small{(non-noisy aggregation)}} 
& 0.2337 & 0.3192 & 0.3717
\\ \midrule 
\textbf{Noisy aggregated teacher} 
& 0.2110 & 0.2868 & 0.3407 
\\
\textbf{Student \small{(noisy aggregation)}} 
& 0.2255 & 0.2984 & 0.3559 
\\ \bottomrule
\end{tabular}
\end{adjustbox}
\end{table}

Results in Table~\ref{tbl_res2} suggest that by using the noisy aggregation of multiple teachers as the supervision signal, we can train a neural ranker with acceptable performance.
%
Compared to the single teacher setup in Section~\ref{sec:mimic_learning_to_rank}, the performance of the student network is not as good as the average performance of teachers, although the student network performs better than the teacher in the noisy aggregation setup. This is more or less the case for a student together with a non-noisy aggregated teacher.
% 
We believe drops in the performance on the student networks compared to the results in Section~\ref{sec:mimic_learning_to_rank} are not just due to partitioning, noise, and aggregation. This is also the effect of the change in the amount of training data for the teachers in our experiments. We expect that in the case of having enough training data in each partition for each teacher, their prediction will be more determined and we will have less disagreement in the aggregation phase and consequently, we will get better signals for training the student model.
\section{Related Work}
In this section, we briefly review the neural ranking models in terms of their general architectures and discuss how the neural models proposed in this chapter are related to the previous works.

Recently, several attempts have been made to study deep neural networks in IR applications, which can be generally partitioned into two categories~\citep{guo2019deep, Onal:2016, Zhang:2016}. 
The first category includes approaches that use the results of trained (deep) neural networks in order to improve the performance in IR applications. Among these, distributed word representations or embeddings~\citep{Mikolov:2013,Pennington:2014} have attracted a lot of attention. Word embedding vectors have been applied to term re-weighting in IR models~\citep{Zheng:2015}, query expansion~\citep{Diaz:2016,Zamani:2016a}, query classification~\citep{Liu:2015,Zamani:2016b},  etc. 
The main shortcoming of most of the approaches in this category is that the objective of the trained neural network differs from the objective of these tasks.  For instance, the word embedding vectors proposed in~\citep{Mikolov:2013,Pennington:2014} are trained based on term proximity in a large corpus, which is different from the objective in most IR tasks. \citet{Zamani:2017} recently proposed relevance-based word embedding models for learning word representations based on the objectives that matter for IR applications.

The second category, which the models proposed in this chapter belong to, consists of the approaches that design and train a (deep) neural network for a specific task, e.g., question answering~\citep{Cohen:2016,Yang:2016}, click models~\citep{Borisov:2016}.
A number of the approaches in this category have been proposed for ranking documents in response to a given query.
These approaches can be generally divided into two groups: \emph{late combination models} and \emph{early combination models} (or representation-focused and interaction-focused models according to~\citep{Guo:2016}). 
The late combination models, following the idea of Siamese networks~\citep{Bromley:1993}, independently learn a representation for each query and candidate document and then calculate the similarity between the two estimated representations via a similarity function. For example, \citet{Huang:2013} proposed DSSM, which is a feed forward neural network with a word hashing phase as the first layer to predict the click probability given a query string and a document title. 
The DSSM model was further improved by incorporating convolutional neural networks~\citep{Shen:2014}.

On the other hand, the early combination models are designed based on the interactions between the query and the candidate document as the input of network. 
For instance, DeepMatch~\citep{Lu:2013} maps each text to a sequence of terms and trains a feed-forward network for computing the matching score. 
The deep relevance matching model for ad-hoc retrieval~\citep{Guo:2016} is another example of an early combination model that feeds a neural network with the  histogram-based features representing interactions between the query and document. 
Early combining enables the model to have an opportunity to capture various interactions between query and document(s), while with late combination approach, the model has only the chance of isolated observation of input elements. Recently, Mitra et al.~\citep{Mitra:2016} proposed to simultaneously learn local and distributional representations, which are early and late combination models respectively,  to capture both exact term matching and semantic term matching.

Until now, all the proposed neural models for ranking are trained on either explicit relevance judgements or clickthrough logs. However, a massive amount of such training data is not always available. 

In this chapter, we propose to train neural ranking models using weak supervision, which is the most natural way to reuse the existing supervised learning models where the imperfect labels are treated as the ground truth.
The basic assumption is that we can cheaply obtain labels (that are of lower quality than human-provided labels) by expressing the prior knowledge we have about the task at hand by specifying a set of heuristics, adapting existing ground truth data for a different but related task (this is often referred to distant supervision\footnote{We do not distinguish between weak and distant supervision as the difference is subtle and both terms are often used interchangeably in the literature.}), extracting supervision signal from external knowledge-bases or ontologies, crowd-sourcing partial annotations that are cheaper to get, etc.
%
Weak supervision is a natural way to benefit from unsupervised data and it has been applied in NLP for various tasks including relation extraction~\citep{Bing:2015,Han:2016}, knowledge-base completion~\citep{Hoffmann:2011}, sentiment analysis~\citep{Severyn:2015:SemEval}, etc.  
There are also similar attempts in IR for automatically constructing test collections~\citep{Asadi:2011} and learning to rank using labeled features, i.e., features that an expert believes they are correlated with relevance~\citep{Diaz:2016:ictir}.
In this chapter, we make use of traditional IR models as the weak supervision signal to generate a large amount of training data and train effective neural ranking models that outperform the baseline methods by a significant margin.


%%%%%%%%%%%%%%%%%%%%%%%%%%%%%%%%%%%%%%%%%%%%%%%%%%%%%%%%%%%%%%%%%%%%%%%%%%%%%%%%%%%%%%%%%%%%%%%%%%%%%%%%%%%%%%%%%%%%%%%%%%%%%%%%%%%%%%%%%%%%%%%%%%%%%%%%%%%%%%%%%%%%%%%%%%%%%%%%%%%%%%%%%%%%%%%%%%%%%%%%%%%%%%%%%%%%%%%%%%%%%%%%%%%%%%%%%%%%%%%%%%%%%%%%%%%%%%%%%%%%%%%%%%%%%%%%%%%%%%%%%%%%%%%%%%%%%%%%%%%%%%%%%%%%%%%%%%%%%%%%%%%%%%%%%%



To circumvent the lack of human-labeled training examples, unsupervised learning methods aim to model the underlying data distribution, thus learning powerful feature representations of the input data, which can be helpful for building more accurate discriminative models especially when little or even no supervised data is available. A large group of unsupervised neural models seeks to exploit the implicit internal structure of the input data, which in turn requires customized formulation of the training objective (loss function), targeted network architectures and often non-trivial training setups. 


For example in NLP, various methods for learning distributed word representations, e.g., word2vec~\citep{Mikolov:2013}, GloVe~\citep{Pennington:2014}, and sentence representations, e.g., paragraph vectors~\citep{Le:2014} and skip-thought~\citep{Kiros:2015} have been shown very useful to pre-train word embeddings that are then used for other tasks such as sentence classification, sentiment analysis, etc. Other generative approaches such as language modeling in NLP, and, more recently, various flavors of auto-encoders~\citep{Baldi:2012} and generative adversarial networks~\citep{Goodfellow:2014} in computer vision have shown a promise in building more accurate models.


A lot of research has been done on the general problem of preserving the privacy of sensitive data in IR applications, where the question is how should we design effective IR systems without damaging users' privacy?  One of the solutions so far is to anonymize the data and try to hide the identity of users~\citep{Carpineto:2013, Zhang:2016}.  As an example, \citet{Zhang:2016} use a differential privacy approach for query log anonymization. However, there is no guarantee that the anonymized data will be as effective as the original data.

Modeling privacy in machine learning is a challenging problem and there has been much research in this area. Preserving the privacy of deep learning models is even more challenging, as there are more parameters to be safeguarded~\citep{Phan:2016}. 
Some work has studied the vulnerability of deep neural network as a service, where the interaction with the model is only via an input-output black box~\citep{Tramer:2016, Fredrikson:2015, Shokri:2016}.
Others have proposed approaches to protect privacy against an adversary with a full knowledge of the training mechanism and access to the model's parameters. For instance, \citet{Abadi:2016} propose a privacy preserving stochastic gradient descent algorithm offering a trade-off between utility and privacy. More recently, \citet{Papernot:2017} propose a semi-supervised method for transferring the knowledge for deep learning from private training data. They propose a setup for learning privacy-preserving student models by transferring knowledge from an ensemble of teachers trained on disjoint subsets of the data for which privacy guarantees are provided.


\section{Conclusion}
In this chapter, we proposed to use unsupervised methods in order to programmatically generate large amounts of training data, as weakly annotated data, to train effective neural ranking models.
We focus on the task of assessing the relevance, i.e. ranking documents given a query that suffers from a large-scale publicly available training set
We examine various neural ranking models with different ranking architectures and objectives, and different input representations. 

We used over six million queries to train our models and evaluated them on Robust04 and ClueWeb 09-Category B collections, in an ad-hoc retrieval setting.  The experiments showed that our best performing model significantly outperforms the BM25 model (our weak supervision signal) by over $13\%$ and $35\%$ MAP improvements in the Robust04 and ClueWeb collections, respectively. 
We also demonstrated that in the case of having a small amount of training data, we can improve the performance of supervised learning by pre-training the network on weakly supervised data.

Based on our results, there are three key ingredients in neural ranking models that lead to good performance with weak supervision:
%
The first is the proper input representation. Providing the network with raw data and letting the network to learn the features that matter, gives the network a chance of learning how to ignore imperfection in the training data.
%
The second ingredient is to target the right goal and define a proper objective function. In the case of having weakly annotated training data, by targeting some explicit labels from the data, we may end up with a model that learned to express the data very well, but is incapable of going beyond it. 
This is especially the case with deep neural networks where there are many parameters and it is easy to learn a model that overfits the data.
%
The third ingredient is providing the network with a considerable amount of training examples. 
As an example, during the experiments we noticed that using the \feedthree, the network needs a lot of examples to learn embeddings that are more effective for retrieval compared to pre-trained embeddings. 
Thanks to weak supervision, we can generate as much training data as we need with almost no cost.

We also study how learning from weak signals can benefit preserving privacy where some noise is intentionally added to the training signal to preserve privacy. We employed Private Aggregation of Teacher Ensembles (PATE)~\citep{Papernot:2017}  to train a privacy-preserving neural ranking model, in which we train several neural rankers on disjoint subsets of the training data and use the noisy aggregated signals from these models on an unlabeled set to train a neural ranker that in a setup that guarantees a certain level of differential privacy.
These experiments lay the groundwork for the idea of sharing a privacy-preserving model instead of sensitive data in IR applications. This suggests researchers from industry share the knowledge learned from actual users' data with the academic community that leads to a better collaboration of all researchers in the field. 

% connection to chapter 5
In this chapter, we mainly focused on the ranking tasks and explored architectural ideas that can implicitly help to have neural networks that are less sensitive to the label noise. 
In the next chapter, we propose more systematic approaches that are tasks and architecture independent and can learn to estimate the quality of the labels and explicitly control the learning process with respect to the estimated qualities.
\chapter{Fidelity Weighted Learning}
\label{chap:5}
\section{Introduction}
The success of deep neural networks to date depends strongly on the availability of labeled data. The more neural networks become deep and complex, the more it is crucial for them to be trained on massive amounts of training data. However, in many applications, labeled data is costly to obtain and task-specific training data is now a critical bottleneck. 

Usually it is much easier to obtain small quantities of high-quality labeled data and large quantities of unlabeled data. The problem of how to best integrate these two different sources of information during training is an active pursuit in the field of semi-supervised learning~\citep{chap:semi06}.
However, for a large class of tasks it is also easy to define one or more so-called ``weak annotators'', additional (albeit noisy) sources of \emph{weak supervision} based on heuristics or ``weaker'', biased classifiers trained on e.g.\ non-expert crowd-sourced data or data from different domains that are related. 
While easy and cheap to generate, it is not immediately clear if and how these additional weakly-labeled data can be used to train a stronger classifier for the task we care about.

More generally, in almost all practical applications machine learning systems have to deal with data \emph{samples of variable quality}. For example, in a large dataset of images only a small fraction of samples may be labeled by experts and the rest may be crowd-sourced using e.g.\ Amazon Mechanical Turk~\citep{Veit:2017}. In addition, in some applications, labels are intentionally perturbed due to privacy issues~\citep{wainwright2012privacy,Papernot:2016, dehghani:2017:neuir}. 

Formally speaking, in our setup, we assume that we are given a large set of unlabeled data samples, a heuristic labeling function called the \emph{\wa}, and a small set of high-quality samples labeled by experts, called the \emph{strong dataset}, consisting of tuples of training samples $x_i$ and their true labels $y_i$, i.e. $\mathcal{D}_s=\{(x_i,y_i)\}$. We consider the latter to be observations from the true target function that we are trying to learn. 
We use the \wa to generate labels for the unlabeled samples. Generated labels are noisy due to the limited accuracy of the \wa. This gives us the \emph{weak dataset} consisting of tuples of training samples $x_i$ and their weak labels $\tilde{y}_i$, i.e. $\mathcal{D}_w=\{(x_i, \tilde{y}_i)\}$.  Note that we can generate a large amount of weak training data $\mathcal{D}_w$ at almost no cost using the \wa. In contrast, we have only a limited amount of observations from the true function, i.e. $|\mathcal{D}_s| \ll |\mathcal{D}_w|$. 

The simplest approach in this setup is to expand the strong training set, $\mathcal{D}_s$, by including the weakly-supervised samples, $\mathcal{D}_w$, which is in fact considering all samples are equally important. Alternatively, one may pretrain on the weak data and then fine-tune on observations from the true function or distribution (which we call strong data). Indeed, it has recently been shown that a small amount of expert-labeled data can be augmented in such a way by a large set of raw data, with labels coming from a heuristic function, to train a more accurate ranking model~\citep{Dehghani:2017:SIGIR, Severyn:2015:SIGIR}.

The downside is that such approaches are oblivious to the amount or source of noise in the labels. Simply speaking, they do not consider the cause of noise in the labels and only focus on the effect. 

In this chapter, we focus on one of our research questions:
% \begin{resqbox}
% \emph{\resq{c5}}
% \end{resqbox}

We argue that treating weakly-labeled samples uniformly (i.e.\ each weak sample contributes equally to the final classifier) ignores potentially valuable information of the label quality. 

Instead, we propose two different approaches that directly model the inaccuracies introduced by the \wa, which can then be used to modulate the training process based on the fidelity (or quality) of each weak sample. In other words, we control the extent to which we make use of this additional source of weak supervision: more for confidently-labeled weak samples close to the true observed data, and less for uncertain samples further away from the observed data.



\subsection{Detailed Research Questions}
We break down our main research question in this chapter into three concrete research questions:
% \begin{resqbox}
% \begin{enumerate}
% \item[\textbf{RQ5.1}] \emph{\resq{c5.1}}
% \item[\textbf{RQ5.2}] \emph{\resq{c5.2}}
% \item[\textbf{RQ5.3}] \emph{\resq{c5.3}}
% \end{enumerate}
% \end{resqbox}
In the following sections, we will address these research questions.

\section{Learning to Learn from Weak Supervision, by Full Supervision}
\label{sec:meta_learning}
Using weak or noisy supervision is a straightforward approach to increase the size of the training data. This is usually done by pre-training the network on a large set of weakly labeled data and fine-tuning it with strong labels~\cite{Dehghani:2017:SIGIR,Severyn:2015:SIGIR}. 
However, these two independent stages do not leverage the full capacity of information from the small set of strong labels.  In particular, in the pre-training stage, we have to learn from labels of variable quality without any control over how these labels contribute to the learning process.

In this section, we address the first research question of this chapter:
\begin{resqbox}
\emph{\resq{c5.1}}
\end{resqbox}

We introduce a semi-\:supervised method that leverages a small amount of data with strong labels to improve the learning from a large amount of data with weak labels. This model in fact, offers learning from \emph{Controlled Weak Supervision} and we refer to it by \emph{\cws} in the rest of this chapter.
%
\cws has three main components:
A \wa, which can be a heuristic model, a weak or biased classifier, or even a human via crowd-sourcing and it is employed to annotate a massive amount of unlabeled data, a \tnet, which uses a large set of weakly annotated instances by the \wa to learn the main task, and a \cnet, which is trained on a small set with strong labels to estimate confidence scores for instances annotated by the \wa. 

The confidence scores estimated by the \cnet define the magnitude of the weight updates applied to the \tnet during training. This way, the \cnet helps the \tnet to avoid the mistakes of her teacher, i.e., \wa, by down-weighting the weight updates from weak labels that do not look reliable according to \cnet.
%
\cws, in fact, employs teacher-student paradigm in which, the \tnet (student) and the \cnet (teacher) are trained jointly in a multi-task fashion and they share parameters of the representation learning layer to share their understanding of the data.

From a meta-learning perspective~\citep{Andrychowicz:2016,Finn2017:ICML,Ravi:2016}, the goal of the \cnet, as the meta-learner, trained jointly with the \tnet, as the learner, is to calibrate the learning rate of the \tnet for each instance in the batch. I.e., the weights $\pmb{w}$ of the \tnet $f_w$ at step $t+1$ are updated as follows:
\begin{equation}
\pmb{w}_{t+1} = \pmb{w}_t - \frac{\eta_t}{b}\sum_{i=1}^b c_{\theta}(x_i, \tilde{y}_i)  \nabla \mathcal{L}(f_{\pmb{w_t}}(x_i), \tilde{y_i})
%+ \nabla \mathcal{R}(\pmb{w_t})
\end{equation}
where $\eta_t$ is the global learning rate, $\mathcal{L}(\cdot)$ is the loss of predicting $\hat{y}=f_w(x_i)$ for an input $x_i$ when the label is $\tilde{y}$; $c_\theta(\cdot)$ is a scoring function learned by the \cnet taking input instance $x_i$ and its noisy label $\tilde{y}_i$. Thus, we can effectively control the contribution to the parameter updates for the \tnet from weakly labeled instances based on how reliable their labels are according to the \cnet (learned on a small supervised data).

Our setup requires running a \wa to label a large amount of unlabeled data, which is done at pre-processing time. For a large class of tasks, it is possible to use a simple heuristic, or implicit human feedback to generate weak labels. This set is then used to train the \tnet.  
In contrast, a small expert-labeled set is used to train the \cnet, which estimates how good the weak annotations are, i.e., controls the effect of weak labels on updating the parameters of the \tnet.

\cws allows learning different types of neural architectures and various tasks, where a meaningful \wa is available. 
Later in this chapter 

we study the performance of \cws by focusing on two applications: sentiment classification and document ranking. 

\subsection{Learning from Controlled Weak Supervision}
\label{sec:method}
In the following, we describe our recipe for semi-\:supervised learning of neural networks, in a scenario where along with a small human-labeled training set a large set of weakly labeled instances is leveraged.

% \subsection{General Architecture}
% \label{sec:generalarchitecture}
\begin{figure}[!t]%
    \makebox[\textwidth][c]{
    \centering
    \begin{subfigure}[t]{0.5\textwidth}
        \centering
        \includegraphics[width=\textwidth]{03-part-02/chapter-05/figs_and_tables/fig_cws_train_v.pdf}
        \caption{\label{fig:train_u}\footnotesize{Full Supervision Mode: Training on batches with strong labels.}}
    \end{subfigure}%
    ~
    \begin{subfigure}[t]{0.5\textwidth}
        \centering
        \includegraphics[width=\textwidth]{03-part-02/chapter-05/figs_and_tables/fig_cws_train_u.pdf}
        \caption{\label{fig:train_v}\footnotesize{Weak Supervision Mode: Training on batches with weak labels.}}
    \end{subfigure}%
    }
    \caption{Learning from controlled weak supervision: Our proposed multi-task network for learning a target task in a semi-supervised fashion, using a large amount of weakly labeled data and a small amount of data with strong labels.
    %
    Faded parts of the network are disabled during the training in the corresponding mode. Red-dotted arrows show gradient propagation. Parameters of the parts of the network in red frames get updated in the backward pass, while parameters of the network in blue frames are fixed during the training. (Best viewed in color.)}
    \label{fig:model}
\end{figure}

In our proposed model we train a multi-task neural network that jointly learns the confidence score of weakly labeled instances and the main task using controlled supervised signals.
%
The high-level representation of the model is shown in Figure~\ref{fig:model}: it comprises a \wa and two neural networks: the \cnet and the \tnet. 

The goal of the \wa is to \emph{provide weak labels} $\tilde{y}_i$ for all the instances $x_i \in U \cup V$. We have this assumption that $\tilde{y}_i$ provided by the \wa are imperfect estimates of strong labels $y_i$, where $y_i$ are available for set $\mathcal{D}_s$, but not for set $\mathcal{D}_w$.

The goal of the \cnet is to \emph{estimate the confidence score} $\tilde{c}_j$ of training instances. It is learned on examples from set $\mathcal{D}_s$, i.e a set of input $x_j$ and its strong label $y_j$ as well its weak label,  $\tilde{y}_j$,  that is annotated by the \wa.
The score $\tilde{c}_j$ is then used to control the effect of weakly labeled instances on updating the parameters of the \tnet in the backward pass of backpropagation.

The \tnet is in charge of \emph{handling the main task} we want to learn, or in other words, approximating the underlying function that predicts the correct labels. 
Given the data instance, $x_i$ and its weak label $\tilde{y}_i$ from the training set $\mathcal{D}_w$, the \tnet aims to predict the label $\hat{y}_i$. 
The \tnet parameter updates are based on noisy labels assigned by the \wa, but the magnitude of the gradient update is based on the output of the \cnet. 

Both networks are trained in a multi-task fashion alternating between the \emph{full supervision} and the \emph{weak supervision} mode.  
In the \emph{full supervision} mode, the parameters of the \cnet get updated using instances from training set $\mathcal{D}_s$.  
As depicted in Figure~\ref{fig:train_v}, each training instance is passed through the representation layer mapping inputs to vectors. These vectors are concatenated with their corresponding weak labels $\tilde{y}_j$ generated by the \wa.
The \cnet, which is a fully connected feedforward network with sigmoid as the output layer, estimates $\tilde{c}_j$ that is the probability of taking data instance $j$ into account for training the \tnet.

In the \emph{weak supervision} mode, the parameters of the \tnet are updated using training set $\mathcal{D}_w$.
As shown in Figure~\ref{fig:train_u}, each training instance is passed through the same representation learning layer and is then processed by the supervision layer which is a part of the \tnet predicting the label for the main task. 
%
We also pass the learned representation of each training instance along with its corresponding label generated by the \wa to the \cnet to estimate the \emph{confidence score} of the training instance, i.e., $\tilde{c}_i$. 
The confidence score is computed for each instance from set $\mathcal{D}_w$. These confidence scores are used to weight the gradient updating \tnet parameters or in other words the step size during back-propagation. 

It is noteworthy that the representation layer is shared between both networks, so besides the regularization effect of layer sharing which leads to better generalization, sharing this layer lays the ground for the \cnet to benefit from the largeness of set $\mathcal{D}_w$ and the \tnet to utilize the quality of set $\mathcal{D}_s$. 

\subsection{Training the Learner and the Meta-Learner}
\label{sec:modeltraining}
Here, we explain how we train \cws in which we jointly update the parameters of \tnet, the learner and the \cnet, the meta-learner. 
Our optimization objective is composed of two terms: (1) the \cnet loss $\mathcal{L}_c$, which captures the quality of the output from the \cnet and (2) the \tnet loss $\mathcal{L}_t$, which expresses the quality for the main task. 

Both networks are trained by alternating between the \emph{weak supervision} and the \emph{full supervision} mode.
%
In the \emph{full supervision} mode, the parameters of the \cnet are updated using training instance drawn from training set $\mathcal{D}_s$. We use cross-entropy loss function for the \cnet to capture the difference between the predicted confidence score of instance $j$, i.e. $\tilde{c}_j$ and the target score $c_j$:
\begin{equation}
% \nonumber
\mathcal{L}_c = \sum_{j\in V} -  c_j \log(\tilde{c}_j) - (1-c_j) \log(1-\tilde{c}_j),
\end{equation}
%given that $c_j$ is not a probability I'm wondering if we can call this loss a cross-entropy --- it only makes sense if $c_j$ is a probability of binary event. Possibly, it's common to use in IR, but i've never seen it used like this on real-valued scores.} 
The target score $c_j$ indicates how similar the strong and the weak labels are, and it is calculated with respect to the main task. 

In the \emph{weak supervision} mode, the parameters of the \tnet are updated using training instances from $\mathcal{D}_w$. We use a weighted loss function, $\mathcal{L}_t$, to capture the difference between the predicted label $\hat{y}_i$ by the \tnet and target label $\tilde{y}_i$:
\begin{equation}
% \nonumber
\mathcal{L}_t = \sum_{i\in U} \tilde{c}_i \mathcal{L}_i,
\end{equation}
where $\mathcal{L}_i$ is the task-specific loss on training instance $i$ and $\tilde{c}_i$ is the confidence score of the weakly annotated instance $i$, estimated by the \cnet.
Note that $\tilde{c}_i$ is treated as a constant during the weak supervision mode and there is no gradient propagation to the \cnet in the backward pass (as depicted in Figure~\ref{fig:train_u}). 

We minimize two loss functions jointly by randomly alternating between full and weak supervision modes (for example, using a 1:10 ratio).
During training and based on the chosen supervision mode, we sample a batch of training instances from $\mathcal{D}_s$ with replacement or from $\mathcal{D}_w$ without replacement (since $\mathcal{D}_w$ can be very large). Since in our setups usually $|\mathcal{D}_w| >> \mathcal{D}_s|$, the training process oversamples the instance from $\mathcal{D}_s$. 

The key point here is that the ``main task'' and ``confidence scoring'' task are always defined to be close tasks and sharing representation will benefit the confidence network as an implicit data augmentation to compensate the small amount of data with strong labels.
Besides, we noticed that updating the representation layer with respect to the loss of the other network acts as a regularization for each of these networks and helps generalization for both target and confidence network since we try to capture all tasks (which are related tasks) and less chance for overfitting.

% We also investigated other possible setups or training scenarios. For instance, we tried updating the parameters of the supervision layer of the \tnet using also data with strong labels. Or instead of using alternating sampling, we tried training the \tnet using controlled weak supervision signals after the \cnet is fully trained.
% As shown in the experiments the architecture and training strategy described above provide the best performance.
\newcommand{\tfunc}{T} %function name for the teacher
\section{\fwlfull}
\label{sec:fidelity_weighted_learning}

In this section, we address the second research question of this chapter:
\resq{c5.2}

We introduce \fwlfull (\fwl), a Bayesian semi-supervised approach that leverages a small amount of data with strong labels to generate a larger training set with \emph{fidelity-weighted weakly-labeled samples}, which can then be used to modulate the learning process based on the quality of each weak sample. By directly modeling the inaccuracies introduced by the \wa in this way, we can control the extent to which we make use of this additional source of weak supervision: more for confidently-labeled weak samples close to the true observed data, and less for uncertain samples further away from the observed data. We use a non-parametric kernel-based method to measure the closeness.

We propose a setting consisting of two main modules. One is called the \std and is in charge of learning a suitable data representation and performing the main prediction task (similar to the \tnet in Section~\ref{sec:meta_learning}), the other is the \tch which modulates the learning process by modeling the inaccuracies in the labels. 

Similar to \cws (introduced in Section~\ref{sec:meta_learning}), \fwl learns to modulate the learning process based on quality of labels. However, unlike \cws, \fwl operates in three different sequential stages and not only it learns to weight the samples based on their quality, but also it learns re-estimate a better labels, i.e. correct the weak labels, during training. 

\begin{figure}[!t]%
    % \makebox[\textwidth][c]{
    \centering
    \begin{subfigure}[t]{0.572\textwidth}
        \centering
        \includegraphics[width=\textwidth]{03-part-02/chapter-05/figs_and_tables/fig_fwl_step_1.pdf}
        \caption{\label{fig:step1}Step 1}
    \end{subfigure}%
    ~
    \begin{subfigure}[t]{0.44\textwidth}
        \centering
        \includegraphics[width=\textwidth]{03-part-02/chapter-05/figs_and_tables/fig_fwl_step_2.pdf}
        \caption{\label{fig:step2}Step 2}
    \end{subfigure}%
    \vfill
    \vspace{30pt}
    \begin{subfigure}[t]{0.748\textwidth}
        \centering
        \includegraphics[width=\textwidth]{03-part-02/chapter-05/figs_and_tables/fig_fwl_step_3.pdf}
        \caption{\label{fig:step3}Step 3}
    \end{subfigure}%
    % }
    \caption{Illustration of \fwlfull: Step 1: Pre-train \std on weak data,  Step 2: Fit \tch to observations from the true function, and Step 3: Fine-tune \std on labels generated by \tch, taking the confidence into account. Red dotted borders and blue solid borders depict components with trainable and non-trainable parameters, respectively.}
    \label{fig:model_fwl}
\end{figure}
\subsection{Recipe of the \fwlfull}
\label{sec:proposed-method}
In this section, we describe the \fwl approach for semi-supervised learning when we have access to weak supervision (e.g., heuristics or weak annotators). 

Our proposed setup comprises a neural network called the \textbf{\std} and a Bayesian function approximator called the \textbf{\tch}. The training process consists of three phases which we summarize in Algorithm~\ref{alg:fwl:main} and Figure~\ref{fig:model_fwl}.

\textbf{Step 1} \emph{Pre-train the \std on $\mathcal{D}_w$ using weak labels generated by the \wa.} (See Figure~\ref{fig:step2}.)

The main goal of this step is to learn a \emph{task dependent} representation of the data as well as pretraining the \std. The \std function is a neural network consisting of two parts. The first part $\psi(\cdot)$ learns the data representation and the second part $\phi(\cdot)$ performs the prediction task (e.g., classification). Therefore the overall function is $\hat{y}=\phi(\psi(x_i))$. The \std is trained on all samples of the weak dataset $\mathcal{D}_w=\{(x_i, \tilde{y}_i)\}_{i=1}^M$. For brevity, in the following, we will refer to both data sample $x_i$ and its representation $\psi(x_i)$ by $x_i$ when it is obvious from the context. 
From self-supervised feature learning point of view, we can say that representation learning in this step is solving a surrogate task of approximating the expert knowledge, for which a noisy supervision signal is provided by the \wa.  


\textbf{Step 2} \emph{Train the \tch on the strong data $(\psi(x_j),y_j) \in \mathcal{D}_s$ represented in terms of the student representation $\psi(\cdot)$ and then use the \tch to generate a soft dataset $\mathcal{D}_{sw}$ consisting of $\langle \textrm{sample}, \textrm{predicted label}, \textrm{ confidence} \rangle$ for \textbf{all} data samples.} (See Figure~\ref{fig:step2}.)

We use a Gaussian process ($\mathcal{GP}$) as the \tch to capture the label uncertainty in terms of the \std representation, estimated w.r.t.\ the strong data. A prior mean and co-variance function is chosen for $\mathcal{GP}$. The learned embedding function $\psi(\cdot)$ in Step 1 is then used to map the data samples to dense vectors as input to the $\mathcal{GP}$. 
We use the learned representation by the \std in the previous step to compensate lack of data in $\mathcal{D}_s$ and the \tch can enjoy the learned knowledge from the large quantity of the weakly annotated data. This way, we also let the \tch  see the data through the lens of the \std.

The $\mathcal{GP}$ is trained on the samples from $\mathcal{D}_s$ to learn the posterior mean $\bm{m}_{\rm post}$ (used to generate soft labels) and posterior co-variance $K_{\rm post}(\cdot,\cdot)$ (which represents label uncertainty).
%\begin{eqnarray*}
%\mathcal{GP}(\bm{m}_{\rm post}, K_{\rm post})&=&\mathcal{GP}(\bm{m}_{\rm %prior}, K_{\rm prior}) | \mathcal{D}_s=\{(\psi(x_j),y_j)\}\\
%\sshrink
%\end{eqnarray*}
We then create the \emph{soft dataset} $\mathcal{D}_{sw}=\{(x_t,\bar{y}_t)\}_{t=1}^{|\mathcal{D}_w \cup \mathcal{D}_s|}$, using the posterior $\mathcal{GP}$, input samples $x_t$ from $\mathcal{D}_w \cup \mathcal{D}_s$, and predicted labels $\bar{y}_t$ with their associated uncertainties as computed by $T(x_t)$ and $\Sigma(x_t)$:
\begin{eqnarray*}
\tfunc(x_t) &=& g(\bm{m}_{\rm post}(x_t))\\
\Sigma(x_t) &=& h(K_{\rm post}(x_t,x_t))
\sshrink
\end{eqnarray*}
The re-generated labels, $\bar{y}_t$, which are called \emph{soft labels}, are equal to strong labels $y_t$, when $x_t \in \mathcal{D}_s$ (with zero uncertainty), and when $x_t \in \mathcal{D}_w$, $\bar{y}_t$ are supposed to be a better labels than the original weak labels $\tilde{y}_t$ (with an uncertainty that is estimated by the $\mathcal{GP}$). 
$g(\cdot)$ transforms the output of $\mathcal{GP}$ to the suitable output space. For example, in classification tasks, $g(\cdot)$ would be the softmax function to produce probabilities that sum up to one. 
For multidimensional-output tasks where a vector of variances is provided by the $\mathcal{GP}$, the vector $K_{\rm post}(x_t,x_t)$ is passed through an aggregating function $h(\cdot)$ to generate a scalar value for the uncertainty of each sample. 
Note that we train $\mathcal{GP}$ only on the strong dataset $\mathcal{D}_s$ but then use it to generate soft labels $\bar{y}_t = \tfunc(x_t)$ and uncertainty $\Sigma(x_t)$ for samples belonging to $\mathcal{D}_{sw}=\mathcal{D}_w\cup \mathcal{D}_s$.

In practice, we furthermore divide the space of data into several regions and assign each region a separate $\mathcal{GP}$ trained on samples from that region. This leads to a better exploration of the data space and makes use of the inherent structure of data. The resulting algorithm, called clustered $\mathcal{GP}$, gave better results compared to a single $\mathcal{GP}$. We describe the detail of the clustered $\mathcal{GP}$ in Section~\ref{sec:CGP}.

By this division of space, we take advantage of the knowledge learned by several teachers, each an expert on its specific region of the data space, which helps in particular when the dimensionality of the input is rather high. As a nice side-effect, this also solves the scalability issues of $\mathcal{GP}$s in that we can increase the number of regions until the number of points in each region is tractable with a single $\mathcal{GP}$, and train these models in parallel. 

\setlength{\textfloatsep}{10pt}
\begin{algorithm}[t!]
\small
% \fontsize{9}{11}\selectfont
\caption{\fwlfull.}%, 
\begin{algorithmic}[1]
\State Train the \std on samples from the weakly-annotated data $D_w$.
\medskip
\State Freeze the representation-learning component $\psi(.)$ of the \std and train \tch on the strong data $\mathcal{D}_s=\{(\psi(x_j),y_j)\}_{j=1}^{N}$. Apply \tch to unlabeled samples $x_t$ to obtain soft dataset $\mathcal{D}_{sw}=\{(x_t,\bar{y}_t)\}_{t=1}^{|\mathcal{D}_w \cup \mathcal{D}_s|}$ where $\bar{y}_t=T(x_t)$ is the soft label and for each instance $x_t$, the uncertainty of its label, $\Sigma(x_t)$, is provided by the \tch.
\medskip
\State Train the \std on samples from $\mathcal{D}_{sw}$ with SGD and modulate the step-size $\eta_t$ according to the per-sample quality estimated using the \tch (Equation~\ref{eqn:eta2}).
\end{algorithmic}
\label{alg:fwl:main}
\end{algorithm}

%
\textbf{Step 3} \emph{Fine-tune the weights of the \std network on the soft dataset, while modulating the magnitude of each parameter update by the corresponding \tch-confidence in its label.} (See Figure~\ref{fig:step2}.)

The \std network of Step 1 is fine-tuned using samples from the soft dataset $\mathcal{D}_{sw}=\{(x_t,\bar{y}_t)\}_{t=1}^{|\mathcal{D}_w \cup \mathcal{D}_s|}$ where $\bar{y}_t=\tfunc(x_t)$.
The corresponding uncertainty $\Sigma(x_t)$ of each sample is mapped to a confidence value according to Equation~\ref{eqn:eta2} below, and this is then used to determine the step size for each iteration of the stochastic gradient descent (SGD). So, intuitively, for data points where we have true labels, the uncertainty of the \tch is almost zero, which means we have high confidence and a large step-size for updating the parameters. However, for data points where the \tch is not confident, we down-weight the training steps of the \std. This means that at these points, we keep the \std function as it was trained on the weak data in Step 1.

More specifically, we update the parameters of the \std by training on $\mathcal{D}_{sw}$ using SGD:
\begin{eqnarray*}
%  \pmb{w}^* &=& \argmin_{\pmb{w} \in \mathcal{W}} \>   \mathcal{L}(\pmb{w}) \\
%  &:=& \frac{1}{T}\sum_{(x_t,\bar{y}_t) \in \mathcal{D}_{sw}}l(\pmb{w}, x_i, \bar{y}_i) + \mathcal{R}(\pmb{w}), \\
  \pmb{w}^* &=& \argmin_{\pmb{w} \in \mathcal{W}} \> \frac{1}{N}\sum_{(x_t,\bar{y}_t) \in \mathcal{D}_{sw}}l(\pmb{w}, x_t, \bar{y}_t) + \mathcal{R}(\pmb{w}), \\
  \pmb{w}_{t+1} &=& \pmb{w}_t - \eta_t(\nabla l(\pmb{w},x_t,\bar{y}_t) + \nabla \mathcal{R}(\pmb{w}))
\end{eqnarray*}
where $l(\cdot)$ is the per-sample loss, $\eta_t$ is the total learning rate, $N$ is the size of the soft dataset $\mathcal{D}_{sw}$, $\pmb{w}$ is the parameters of the \std network, and $\mathcal{R(\cdot)}$ is the regularization term. %Regularization term is the usual regularization used by optimization packages (e.g., weight decay). Therefore, we do not go into its details here.

We define the total learning rate as $\eta_t = \eta_1(t)\eta_2(x_t)$, where $\eta_1(t)$ is the usual learning rate of our chosen optimization algorithm that anneals over training iterations, and $\eta_2(x_t)$ is a function of the label uncertainty $\Sigma(x_t)$ that is computed by the \tch for each data point. Multiplying these two terms gives us the total learning rate. In other words, $\eta_2$ represents the \emph{fidelity} (quality) of the current sample, and is used to multiplicatively modulate $\eta_1$. Note that $\eta_1$ does not necessarily depend on each data point, whereas $\eta_2$ does. We propose
\begin{equation}
 \label{eqn:eta2}
 \eta_2(x_t) = \exp[-\beta \Sigma(x_t)],    
\end{equation}
to exponentially decrease the learning rate for data point $x_t$ if its corresponding soft label $\bar{y}_t$ is unreliable (far from a true sample).
In practice, when using mini-batches, we implement this by multiplying the loss of each sample in the batch by its fidelity score and averaging over these fidelity-weighted losses in the batch when calculating the batch gradient based on that loss. In Equation~\ref{eqn:eta2}, $\beta$ is a positive scalar hyper-parameter. Intuitively, a small $\beta$ results in a \std that listens more carefully to the \tch and copies its knowledge, while a large $\beta$ makes the \std pay less attention to the \tch, staying with its initial weak knowledge. 
%More concretely speaking, as $\beta \to 0$ \std places more trust in the labels $\bar{y}_t$ estimated by the \tch and the \std copies the knowledge of the \tch. On the other hand, as $\beta \to \infty$, \std puts less weight on the extrapolation ability of $\mathcal{GP}$ and the parameters of the \std are not affected by the correcting information from the \tch. 


\subsection{Multi-Teacher \fwl using Clustered GP}
\label{sec:CGP}
In this section, we explain the clustered GP which is an effective way of applying $\mathcal{GP}$where the scale of the data increases.  Clustered GP suggests using several $\mathcal{GP}=\{GP_{c_i}\}$ to explore the entire data space more effectively. Even though inducing points and stochastic methods make $\mathcal{GP}$s more scalable we still observed poor performance when the entire dataset was modeled by a single $\mathcal{GP}$. Therefore, the reason for using multiple $\mathcal{GP}$s is mainly empirical inspired by~\citep{shen2006fast} which is explained in the following:

We used Sparse Gaussian Process implemented in GPflow. The algorithm is scalable in the sense that it is not $O(N^3)$ as the original $\mathcal{GP}$ is, but it introduces inducing points in the data space and defines a variational lower bound for the marginal likelihood. The variational bound can be optimized by stochastic methods, which makes the algorithm applicable in large datasets. However, the tightness of the bound depends on the location of the inducing points, which are found through the optimization process. 

In~\citep{dehghani:2018:ICLR}, it is empirically observed that a single $\mathcal{GP}$ does not give a satisfactory accuracy on left-out test dataset on our tasks/datasets. This can be due to the inability of the algorithm to find good inducing points when the number of inducing points is restricted to just a few. Then we increased the number of inducing points $M$ which trades off the scalability of the algorithm because it scales with $O(NM^2)$. Moreover, apart from scalability which is partly solved by stochastic methods, we argue that the structure of the entire space may not be explored well by a single $\mathcal{GP}$ and its inducing points. This can be due to the observation that our datasets are distributed in a highly sparse way within the high dimensional embedding space. To cure the problem, one can use PCA to reduce input dimensions and give a denser representation, but based on the experiments in~\citep{dehghani:2018:ICLR}, it does not result in a considerable improvement. 
%The results are presented in Tabel~\ref{tbl_cgp}. 
%\input{cgp_res.tex}

We may be able to argue that clustered $\mathcal{GP}$ makes better use of the data structure roughly close to the idea of KISS-GP~\citep{Wilson:2015:KIS:3045118.3045307}.
In inducing point methods, it is normally assumed that $k \ll h$ ($k$ is the number of inducing points and $h$ is the number of training samples) for computational and storage saving. However, we have this intuition that few number of inducing points make the model unable to explore the inherent structure of data~\citep{Wilson:2015:KIS:3045118.3045307}. By employing several GPs, we were able to use a large number of inducing points even when $k>h$ which seemingly better exploits the structure of datasets. Because our work was not aimed to be a close investigation of GP, we considered clustered $\mathcal{GP}$ as the engineering side of the work which is a tool to give us a measure of confidence. Other tools such as a single $\mathcal{GP}$ with inducing points that form a Kronecker or Toeplitz covariance matrix are also conceivable. Therefore, we do not of course claim that we have proposed a new method of inference for GPs.  Here is practical description of clustered $\mathcal{GP}$ algorithm:
{\it Clustered $\mathcal{GP}$}: Let $N$ be the size of the dataset on which we train the \tch. Assume we allocate $K$ teachers to the entire data space. Therefore, each $\mathcal{GP}$ sees a dataset of size $n=N/K$.
Then we use a simple clustering method (e.g., k-means) to find centroids of $K$ clusters $C_1, C_2, \ldots, C_K$ where $C_i$ consists of samples $\{x_{i,1}, x_{i,2},\ldots,x_{i,n}\}$. We take the centroid $c_i$ of cluster $C_i$ as the representative sample for all its content. Note that $c_i$ does not necessarily belong to $\{x_{i,1}, x_{i,2},\ldots,x_{i,n}\}$. We assign each cluster a $\mathcal{GP}$ trained by samples belonging to that cluster. More precisely, cluster $C_i$ is assigned a $\mathcal{GP}$ whose data points are $\{x_{i,1}, x_{i,2},\ldots,x_{i,n}\}$.
Because there is no dependency among different clusters, we train them in parallel to speed-up the procedure. 

The pseudo-code of the clustered $\mathcal{GP}$ is presented in Algorithm~\ref{alg:CGP}. When the main issue is computational resources (when the number of inducing points for each $\mathcal{GP}$ is large), we can first choose the number $n$ which is the maximum size of the dataset on which our resources allow us to train a $\mathcal{GP}$, then find the number of clusters $K=N/n$ accordingly. The rest of the algorithm remains unchanged. 
\begin{algorithm}[t!]
% \small
\caption{Clustered Gaussian processes.}%, 
\begin{algorithmic}[1]
\State Let $N$ be the sample size, $n$ the sample size of each cluster, $K$ the number of clusters, and $c_i$ the center of cluster $i$.
\medskip
\State Run K-means with $K$ clusters over all samples with true labels $\mathcal{D}_s=\{(x_j,y_j)\}_{j=1}^N$.
\begin{equation*}
    {\rm K\mbox{-}means}({x_j}) \rightarrow {c_1, c_2, \ldots, c_K}
\end{equation*}
where $c_i$ represents the center of cluster $C_i$ containing samples $D_s^{c_i}=\{x_{i,1}, x_{i,2}, ... x_{i,n}\}$.
\medskip
\State Assign each of $K$ clusters a Gaussian process and train them in parallel to approximate the label of each sample:
\begin{eqnarray*}
\mathcal{GP}_{c_i}(\bm{m}_{\rm post}^{c_i}, K_{\rm post}^{c_i})&=&\mathcal{GP}(\bm{m}_{\rm prior}, K_{\rm prior}) | D_s^{c_i}=\{(\psi(x_{s,c_i}),y_{s,c_i})\}\\
\tfunc_{c_i}(x_t) &=& g(\bm{m}_{\rm post}^{c_i}(x_t))\\
\Sigma_{c_i}(x_t) &=& h(K_{\rm post}^{c_i}(x_t,x_t))
\end{eqnarray*}, \\
% \tfunc_{{\rm prior},{c_i}}&=&\mathcal{GP}_{c_i}(\bm{0}, K_{\rm prior})\\
% T_{c_i}|\mathcal{D}_s^{c_i}, \tfunc{\rm prior }&=&\mathcal{GP}_{c_i}(\bm{m}_{{\rm prior}, {c_i}}, K_{\rm prior})
where $\mathcal{GP}_{c_i}$ is trained on $\mathcal{D}_s^{c_i}$ containing samples belonging to the cluster $c_i$. Other elements are defined in Section~\ref{sec:proposed-method}
\medskip
\State Use trained teacher $\tfunc_{c_i}(.)$ to evaluate the soft label and uncertainty for samples from $\mathcal{D}_{sw}$ to compute $\eta_2(x_t)$ required for step 3 of Algorithm~\ref{alg:fwl:main}. We use $\tfunc(.)$ as a wrapper for all teachers $\{T_{c_i}\}$.
\end{algorithmic}
\label{alg:CGP}
\end{algorithm}
\shrink



\subsection{FWL on a Toy Example}
\label{sec:toy_exmpale}
\begin{figure}[!t]%
   \makebox[\linewidth][c]{%
    \centering
    \begin{subfigure}[t]{0.5\textwidth}
        \centering
        \includegraphics[width=1\textwidth]{03-part-02/chapter-05/figs_and_tables/fig_toy_ex_plot1.png}
        \vspace{-20pt}
        \caption{\label{fig:toy_plot1}Training \std on 100 examples from the weak function.}
    \end{subfigure}%
    ~
    \begin{subfigure}[t]{0.5\textwidth}
        \centering
        \includegraphics[width=1\textwidth]{03-part-02/chapter-05/figs_and_tables/fig_toy_ex_plot2.png}
        \vspace{-20pt}
        \caption{\label{fig:toy_plot2}Fitting \tch based on 10 observations from the true function.}
    \end{subfigure}%
    }
    \\
    \makebox[\linewidth][c]{%
    \begin{subfigure}[t]{0.5\textwidth}
        \centering
        \includegraphics[width=1\textwidth]{03-part-02/chapter-05/figs_and_tables/fig_toy_ex_plot3.png}
        \vspace{-20pt}
        \caption{\label{fig:toy_plot3}Fine-tuning the \std based on observations from the true function.}
    \end{subfigure}%
    ~
    \begin{subfigure}[t]{0.5\textwidth}
        \centering
        \includegraphics[width=1\textwidth]{03-part-02/chapter-05/figs_and_tables/fig_toy_ex_plot4.png}
        \vspace{-20pt}
        \caption{\label{fig:toy_plot4}Fine-tuning the \std based on label/confidence from \tch.}
    \end{subfigure}%
    }
    \vspace{-5pt}
    \caption{Toy example: The true function we want to learn is $y = \sin(x)$ and the weak function is $y = 2 sinc(x)$.}
    \label{fig:toy}
    \vspace{-5pt}
\end{figure}
To better understand \fwl, we apply \fwl to a one-dimensional toy problem to illustrate the various steps.
%
Let $f_t(x)=\sin(x)$ be the true function (red dotted line in Figure~\ref{fig:toy_plot1}) from which a small set of observations $\mathcal{D}_s=\{(x_j,y_j)\}_{j=1}^{N}$ is provided (red points in Figure~\ref{fig:toy_plot2}). These observation might be noisy, in the same way that labels obtained from a human labeler could be noisy.
%
A \wa function $f_{w}(x)=2sinc(x)$ (magenta line in Figure~\ref{fig:toy_plot1}) is provided, as an approximation to $f_t(\cdot)$.

%
The task is to obtain a good estimate of $f_t(\cdot)$ given the set $\mathcal{D}_s$ of strong observations and the \wa function $f_{w}(\cdot)$.
%
We can easily obtain a large set of observations $\mathcal{D}_w=\{(x_i,\tilde{y}_i)\}_{i=1}^M$ from $f_{w}(\cdot)$ with almost no cost (magenta points in Figure~\ref{fig:toy_plot1}). 

As the \tch, we use standard Gaussian process regression\footnote{\url{http://gpflow.readthedocs.io/en/latest/notebooks/regression.html}} with this kernel:
\begin{equation}
k(x_i,x_j)=k_{\rm RBF}(x_i,x_j)+k_{\rm White}(x_i,x_j),
\end{equation}
where
\begin{flalign*}
    \hspace{6em}
    &&k_{\rm RBF}(x_i,x_j) &= \exp{\left(\frac{\Vert x_i-x_j\Vert^2}{2^2}\right)}, & 
    \\
    &&k_{\rm White}(x_i,x_j) &= constant\_value ~\forall x_i=x_j \text{, and } 0 \text{ otherwise}. & 
\end{flalign*}

We fit only one $\mathcal{GP}$ on all the data points (i.e., no clustering). Also during fine tuning, we set $\beta = 1$.
The \std is a simple feed-forward network with the depth of 3 layers and width of 128 neurons per layer.  We have used $tanh$ as the nonlinearity for the intermediate layers and a linear output layer. As the optimizer, we used Adam~\citep{Kingma:2014} and the initial learning rate has been set to $0.001$.
We randomly sample 100 data points from the \wa and 10 data points from the true function. We introduce a small amount of noise to the observation of the true function to model the noise in the human labeled data. 


We consider two experiments: 
\begin{enumerate}[leftmargin=*]
%save some space
\setlength{\topsep}{0.3pt}
\setlength{\partopsep}{0.3pt}
\setlength{\itemsep}{0.3pt}
\setlength{\parskip}{0.3pt}
\setlength{\parsep}{0.3pt}
    \item A neural network trained on weak data and then fine-tuned on strong data from the true function, which is the most common semi-supervised approach (Figure~\ref{fig:toy_plot3}).
    \item A teacher-student framework working by the proposed \fwl approach.
\end{enumerate} 

As can be seen in Figure~\ref{fig:toy_plot4}, by taking into account label confidence, \fwl gives a better approximation of the true hidden function, compared to the standard fine tuning.  We repeated the above experiment 10 times. The average RMSE with respect to the true function on a set of test points over those 10 experiments for the \std, were as follows:
\begin{enumerate}[leftmargin=*]
%save some space
\setlength{\topsep}{0.3pt}
\setlength{\partopsep}{0.3pt}
\setlength{\itemsep}{0.3pt}
\setlength{\parskip}{0.3pt}
\setlength{\parsep}{0.3pt}
    \item Student is trained on weak data (blue line in Figure~\ref{fig:toy_plot1}): $0.8406$,
    \item Student is trained on weak data then fine tuned on true observations (blue line in Figure~\ref{fig:toy_plot3}): $0.5451$,
    \item Student is trained on weak data, then fine tuned by soft labels and confidence information provided by the teacher (blue line in Figure~\ref{fig:toy_plot4}): $0.4143$ (best).
\end{enumerate}
\section{Applications and Setups}
We apply \cws and \fwl, two approaches introduced in this chapter for learning from vast amount of weakly annotated data, while a small set of labeled data exist, to two different tasks: \emph{document ranking} and \emph{sentiment classification}. 
Whilst these two applications differ considerably, as do the exact operationalization of the propose models to these cases, in both cases the human gold standard data is based on a cognitively complex, or subjective, judgments causing high interrater variation, increasing both the cost of obtaining labels as the need for larger sets of labels.

\begin{table}[tbp]
\caption{\label{tbl_baselines} Descriptions of baseline models.
}
\centering
\fontsize{9}{11}\selectfont{
% \begin{adjustbox}
{\renewcommand{\arraystretch}{1.1}
\begin{tabular}{@{}l@{~~}p{0.16\textwidth}@{~~~}p{0.76\textwidth}@{}}
\toprule
\multicolumn{3}{c}{\textbf{Basic Baselines}}
\\\midrule
\bf 1 & \textbf{WA} & The \wa, i.e., the unsupervised method used for annotating the unlabeled data.
\\
\bf 2 & \textbf{$\text{NN}_{\text{S}}$} & Full Supervision Only, i.e., the \tnet (or the \std) trained only on strong labeled data ($\mathcal{D}_s$).
\\
\bf 3 & \textbf{$\text{NN}_{\text{W}}$} &  Weak Supervision Only, i.e., the \tnet (or the \std)  trained only on weakly labeled data ($\mathcal{D}_w$).
\\ \midrule
\bf 4 & \textbf{$\text{NN}_{\text{W}\text{/S}^+}$}  & Weak Supervision + Oversampled Strong Supervision, i.e., the \tnet (or the \std) trained on samples that are alternately drawn from $\mathcal{D}_w$ without replacement, and $\mathcal{D}_s$ with replacement. Since $|\mathcal{D}_s| \ll |\mathcal{D}_w|$, it oversamples the strong data.
\\
\bf 5 & \textbf{$\text{NN}_{\text{W}} \to \text{NN}_{\text{S}}$}  & Weak Supervision + Fine Tuning, i.e., the \tnet (or the \std) trained on weak dataset $\mathcal{D}_w$ and fine-tuned on strong dataset $\mathcal{D}_s$.
\\
\bf 6 & \textbf{$\text{NN}_{\text{W}} \to \text{NN}^{\text{Sup}}_{\text{S}}$} & Weak Supervision + Supervision Layer fine tuning, i.e., the \tnet (or the \std) trained only on on weak dataset $\mathcal{D}_w$ and the supervision layer is fine-tuned on strong dataset $\mathcal{D}_s$, while the representation learning layer is fixed.
\\
\bf 7 & \textbf{$\text{NN}_{\text{W}} \to \text{NN}^{\text{Rep}}_{\text{S}}$} & Weak Supervision + Representation Learning Layer Fine Tuning, i.e., the \tnet (or the \std) trained only on on weak dataset $\mathcal{D}_w$ and the representation layer is fine-tuned on strong dataset $\mathcal{D}_s$, while the representation learning layer is fixed.
\\\midrule
\multicolumn{3}{c}{\textbf{Controlled Weak Supervision}}
\\\midrule
\bf 8 & \textbf{\cws} & Learning from Controlled Weak Supervision as explained in Section~\ref{sec:meta_learning}.
\\
\bf 9 & \textbf{\cwsnospace$_\text{JT+}$} & Controlled Weak Supervision with Joint Training is the same as \cws (explained in Section~\ref{sec:modeltraining}), except that parameters of the supervision layer in \tnet are also updated using batches from $V$, with regards to the strong labels.
\\
\bf 10 & \textbf{\cwsnospace$_\text{ST}$} & Separate Training, i.e., we consider the \cnet as a separate network, without sharing the representation learning layer, and train it on set $V$. We then train the \tnet on the controlled weak supervision signals.
\\
\bf 11 & \textbf{\cwsnospace$_\text{CT}$} & Circular Training, i.e., we train the \tnet on set $U$. Then the \cnet is trained on data with strong labels, and the \tnet is trained again but on controlled weak supervision signals.
\\
\bf 12 & \textbf{\cwsnospace$_\text{PT}$} & Progressive Training is the mixture of the two previous baselines. Inspired by \cite{Rusu:2016}, we transfer the learned information from the converged \tnet to the \cnet using progressive training. We then train the \tnet again on the controlled weak supervision signals.
\\\midrule
\multicolumn{3}{c}{\textbf{Fidelity Weighted Learning}}
\\\midrule
\bf 13 & \textbf{\fwl} & Fidelity Weighted Learning that is explained in Section~\ref{sec:fidelity_weighted_learning}.
\\
\bf 14 & \textbf{$\text{NN}_{\text{W}^\omega \to \text{NN}_\text{S}}$} & The \std trained on the weak data, but the step-size of each weak sample is weighted by a fixed value $0 \leq \omega \leq 1$, and fine-tuned on strong data. As an approximation for the optimal value for $\omega$, we have used the mean of $\eta_2$ of our model (below).
\\
\bf 15 & \textbf{\fwlnospace$_{unsuprep}$} & The representation in the first step is trained in an unsupervised way\footnote{In the document ranking task, as the representation of documents and queries we use weighted averaging over pretrained embeddings of their words based on their inverse document frequency~\citep{Dehghani:2017:SIGIR}. In the sentiment analysis task, we use skip-thoughts vectors~\citep{kiros2015skip}.} and the student is trained on samples labeled by the \tch using the confidence scores.
\\
\bf 16 & \textbf{\fwlnospace$\backslash\Sigma$} & The \std trained on the weakly labeled data and fine-tuned on samples labeled by the \tch without taking the confidence into account.
\\\bottomrule
\end{tabular}
}
}
%\end{adjustbox}
\end{table}


For both tasks, we evaluate the performance of \cws as well as \fwl compared to some baselines that are described in Table~\ref{tbl_baselines}. 
%We also conducted a series of ablation studies.

\subsection{Document Ranking}
This task is the core information retrieval problem and is challenging as the ranking model needs to learn a representation for long documents and capture the notion of relevance between queries and documents. Furthermore, as it was discussed in Chapter~\ref{chap:4}, the size of publicly available datasets with query-document relevance judgments is unfortunately quite small ($\sim 250$ queries).
%
In our experiments, ranking is cast as a regression task. Given each training sample $x$ as a triple of query $q$, and two documents $d^+$ and $d^-$, the goal is to learn a function $\mathcal{F} : \{<q, d^+, d^->\} \rightarrow \mathbb{R}$, which maps each data sample $x$ to a scalar output value $y$ indicating the probability of $d^+$ being ranked higher than $d^-$ with respect to $q$. 

\begin{figure}{t}
    \centering
            \includegraphics[width=0.35\textwidth]{03-part-02/chapter-05/figs_and_tables/fig_ranker.pdf}
    \caption{The document ranker used as \tch in \cws and \std in \fwl.}
    \label{fig:ranker}
\end{figure}


\subsubsection{The \tnet in \cws and the \std in \fwl}
We employ the pairwise neural ranker architecture explained in Section~\ref{sec:modeltwo} as the \tnet in \cws and \std in \fwl. 

Each training instance $x$ consists of a query $q$, and two documents $d^+$ and $d^-$. The labels, $\tilde{y}$ and $y$, are scalar values indicating the probability of $d^+$ being ranked higher than $d^-$ with respect to $q$.

\mypar{The Representation Learning Layer.}
This layer learns a function $\varepsilon: \mathcal{V} \rightarrow \mathbb{R}^{m}$  (where $\mathcal{V}$ denotes the vocabulary set, and $m$ is the number of embedding dimensions) that maps each word to its embedding which is downstream of the ranking task as well as a weighting function $\omega: \mathcal{V} \rightarrow \mathbb{R}$ which learns the global importance of each word. Then, the learned weights are used to compose word embeddings to generate query/document embeddings. The output of this layer is the concatenation of vectors representing query and two documents.
%
In our experiments, we initialize the embedding function $\varepsilon$ with word2vec embeddings~\cite{Mikolov:2013} pre-trained on Google News and the weighting function $\omega$ with IDF.

\mypar{The Supervision Layer.} 
This layer receives the vector representation of the inputs processed by the representation learning layer and outputs a prediction $\hat{y}_i$.
We opt for a simple fully connected feed-forward network with $l$ hidden layers followed by a sigmoid. We employ the weighted cross entropy loss:
\begin{equation}
% \nonumber
\small
\mathcal{L}_t = \sum_{i\in B_U} \tilde{c}_i [- \tilde{y}_i \log (\hat{y}_i) - (1-\tilde{y}_i) \log(1-\hat{y}_i)],
\end{equation}
where $B_U$ is a batch of instances from $U$, and $\tilde{c}_i$ is the confidence score of the weakly annotated instance $i$, estimated by the \cnet.

The general schema of the \tnet (or \std) is illustrated in Figure~\ref{fig:ranker}. More details are provided in Section~\ref{sec:modeltwo}.

\subsubsection{The \wa}
The \wa in the document ranking task is BM25~\citep{Robertson:2009}, a well-known unsupervised retrieval method. This method heuristically scores a given pair of query-document based on the statistics of their matched terms. In the pairwise document ranking setup, $\tilde{y}_i$ for a given sample $x_j = (q,d^+,d^-)$ is the probability of document $d^+$ being ranked higher than $d^-$: 
$\tilde{y}_i = P_{q,d^+,d^-} = \nicefrac{s_{q,d^+}}{s_{q,d^+} + s_{q,d^-}}$,  where $s_{q,d}$ is the score obtained from the \wa.


\subsubsection{The \cnet in \cws}
The \cnet is a regresses and we use a simple fully connected feed-forward network. To train the \cnet, the target label $c_j$ is calculated using the absolute difference of the strong label and the weak label: $c_j= 1-|y_j - \tilde{y}_j|$, where $y_j$ is calculated similar to $\tilde{y}_i$, but $s_{q,d}$ comes from strong labels provided by humans.


\subsubsection{The \tch in \fwl}
We use Gaussian Process as the \tch in order to generate soft labels. We pass the mean of $\mathcal{GP}$ through the same function $g(.)$ that is applied on the output of the \std network, where the $g(.)$ is sigmoid for the document ranking task.
Since we have one dimensional regression here, $\Sigma(x_t)$ is scalar and $h(.)$ is identity.
In the \tch, linear combinations of different kernels are used. For the document ranking task, we use sparse variational GP regression\footnote{\url{http://gpflow.readthedocs.io/en/latest/notebooks/SGPR_notes.html}}~\citep{Titsias2009variational} with this kernel:
\begin{equation}
k(x_i,x_j)=k_{\rm Matern3/2}(x_i,x_j)+{k_{\rm Linear}}(x_i,x_j)+k_{\rm White}(x_i,x_j)
\end{equation}

where,
\begin{flalign*}
    \hspace{6em}
    &&k_{\rm Matern3/2}(x_i,x_j) &= \left(1+\frac{\sqrt{3}\Vert x_i-x_j\Vert}{l}\right)\exp{\left(-\frac{\sqrt{3}\Vert x_i-x_j\Vert}{l}\right)} & \\
    &&k_{\rm Linear}(x_i,x_j) &= \sigma_0^2+x_i.x_j & \\
    &&k_{\rm White}(x_i,x_j) &= constant\_value, \quad \forall x_1=x_2 \text{ and } 0 \text{ otherwise} & 
\end{flalign*}

We empirically found $l=1$ satisfying value for the length scale of Matern3/2 kernel.
We also set $\sigma_0 = 0$ to obtain a homogeneous linear kernel. 
The constant value of $K_{White}(.,.)$ determines the level of noise in the labels. This is different from the noise in weak labels. This term explains the fact that even in strong labels there might be a trace of noise due to the inaccuracy of human labelers. 

It's noteworthy that we used clustered $\mathcal{GP}$ algorithm, explained in Section~\ref{sec:CGP}, and set the number of clusters to $50$ for this task..  


\subsubsection{Collections}
We use two standard TREC collections for the task of ad-hoc retrieval: The first collection (\emph{Robust04}) consists of 500k news articles from different news agencies as a homogeneous collection. The second collection (\emph{ClueWeb}) is ClueWeb09 Category B, a large-scale web collection with over 50 million English documents, which is considered as a heterogeneous collection. Spam documents were filtered out using the Waterloo spam scorer~\footnote{\url{http://plg.uwaterloo.ca/~gvcormac/clueweb09spam/}}~\citep{Cormack:2011} with the default threshold $70\%$. 

\mypar{Data with strong labels.} 
We take query sets that contain human-labeled judgments: a set of 250 queries (TREC topics 301--450 and 601--700) for the Robust04 collection and a set of 200 queries (topics 1-200) for the experiments on the ClueWeb collection.
For each query, we take all documents judged as relevant plus the same number of documents judged as non-relevant and form pairwise combinations among them.

\mypar{Data with weak labels.}
We create a query set $Q$ using the unique queries appearing in the AOL query logs~\citep{Pass:2006}. 
This query set contains web queries initiated by real users in the AOL search engine that were sampled from a three-month period from March 2006 to May 2006. 
We applied standard pre-processing~\cite{Dehghani:2017:SIGIR,Dehghani2017:CIKM} on the queries: We filtered out a large volume of navigational queries containing URL substrings (``http'', ``www.'', ``.com'', ``.net'', ``.org'', ``.edu''). We also removed all non-alphanumeric characters from the queries. For each dataset, we took queries that have at least ten hits in the target corpus using our \wa method. Applying all these steps, 
We collect 6.15 million queries to train on in Robust04 and 6.87 million queries for ClueWeb.
To prepare the weakly labeled training set $\mathcal{D}_w$, we take the top $1,000$ retrieved documents using BM25 for each query from training query set $Q$, which in total leads to $\sim|Q|\times 10^6$ training samples. 

\subsubsection{Experimental Setup.}
For the evaluation of the whole model, we conducted a 3-fold cross-validation. However, for each dataset, we first tuned all the hyper-parameters of the \tnet in \cws (and \std in the first step of \fwl) in the first step on the set with strong labels using batched GP bandits with an expected improvement acquisition function~\citep{Desautels:2014} and kept the optimal parameters of the \tnet (and \std) fixed for all the other experiments.
The size and number of hidden layers for the \tnet (and \std) is selected from $\{64, 128, 256, 512\}$. The initial learning rate and the dropout parameter were selected from $\{10^{-3}, 10^{-5}\}$ and $\{0.0, 0.2, 0.5\}$, respectively. We considered embedding sizes of $\{300, 500\}$. The batch size in our experiments was set to $128$.  We use ReLU~\citep{Nair:2010} as a non-linear activation function $\alpha$ in \tnet (and \std).  We use the Adam optimizer~\citep{Kingma:2014} for training, and \emph{dropout}~\citep{Srivastava:2014} as a regularization technique.

%
At inference time, for each query, we take the top $2,000$ retrieved documents using BM25 as candidate documents and re-rank them using the trained models. We use the Indri\footnote{\url{https://www.lemurproject.org/indri.php}} implementation of BM25 with default parameters (i.e., $k_1 = 1.2$, $b = 0.75$, and $k_3 = 1,000$).

\subsubsection{Results and Discussions} 
\label{sec:res_and_disc_ranking}
We conducted k-fold cross validation on $\mathcal{D}_s$ (the strong data) and report two standard evaluation metrics for ranking: mean average precision (MAP) of the top-ranked $1,000$
documents and normalized discounted cumulative gain calculated for the top $20$ retrieved documents (nDCG@20). 
Table~\ref{tbl_main} shows the performance on both datasets. As can be seen, \fwl and \cws both provide significant boost on the performance on top of the baseline methods over both datasets.


\begin{table}[tbp]
\renewcommand{\arraystretch}{1.1}
\caption{\label{tbl_main}Performance of \cws and \fwl as well as the main baseline methods, described in Table~\ref{tbl_baselines}, for the document ranking task. \pssmall{i} indicates that the improvements with respect to the baseline $i$ are statistically significant at the 0.05 level using the paired two-tailed t-test with Bonferroni correction.}
\centering
\begin{adjustbox}{max width=\textwidth}
\begin{tabular}{r l l l l l}
\toprule
& \multirow{2}{*}{Method} &
\multicolumn{2}{c}{Robust04} & \multicolumn{2}{c}{ClueWeb}
\\ 
\cmidrule(lr){3-4} \cmidrule(lr){5-6}
& & \small{MAP} & \small{nDCG@20}
& \small{MAP} & \small{nDCG@20}
\\ \midrule
1 & \small{WA$_\text{BM25}$} 
& 0.2503\pssmall{2} & 0.4102\pssmall{2}  
& 0.1021\pssmall{2} & 0.2070\pssmall{2}
\\ \midrule
2 & \small{$\text{NN}_{\text{S}}$} 
& 0.1790 & 0.3519  
& 0.0782 & 0.1730
\\
3 & \small{$\text{NN}_{\text{W}}$} 
& 0.2702\pssmall{12} & 0.4290\pssmall{12}  
& 0.1297\pssmall{12} & 0.2201\pssmall{12}
\\ \midrule
4 & \small{$\text{NN}_{\text{W}\text{/S}^+}$} 
&  0.2763\pssmall{123} & 0.4330\pssmall{123} 
&  0.1354\pssmall{123} & 0.2319\pssmall{123}
\\
5 & \small{$\text{NN}_{\text{W}} \to \text{NN}_{\text{S}}$} 
&  0.2810\pssmall{12346} & 0.4372\pssmall{12346} 
&  0.1346\pssmall{12346} & 0.2317\pssmall{12346}
\\
6 & \small{$\text{NN}_{\text{W}} \to \text{NN}^{\text{Sup}}_{\text{S}}$}
&  0.2711\pssmall{123} & 0.4203\pssmall{123} 
&  0.1002\pssmall{123} & 0.1940\pssmall{123}
\\
7 & \small{$\text{NN}_{\text{W}} \to \text{NN}^{\text{Rep}}_{\text{S}}$}
&  0.2810\pssmall{1234} & 0.4316\pssmall{1234} 
&  0.1286\pssmall{1234} & 0.2240\pssmall{1234}
\\
8 & \small{\cws}
&  0.3017\pssmall{1234567} & 0.4511\pssmall{1234567} 
&  0.1363\pssmall{1234567} & 0.2444\pssmall{1234567}
\\
13 & \small{\fwl}
& \textbf{0.3124}\pssmall{1234567}  & \textbf{0.460}\pssmall{1234567} & \textbf{0.1472}\pssmall{1234567}  & \textbf{0.2453}\pssmall{1234567}
\\\bottomrule
\end{tabular}
\end{adjustbox}
\end{table}



In the ranking task, the \std is designed in particular to be trained on weak annotations~\citep{Dehghani:2017:SIGIR}, hence training the network only on weak supervision, i.e. $\text{NN}_{\text{W}}$ performs better than $\text{NN}_{\text{S}}$. This can be due to the fact that ranking is a complex task requiring many training samples to learn representations that can be used to assess the relevance, while relatively few data with strong labels are available.

Alternating between strong and weak data during training, i.e. $\text{NN}_{\text{W}\text{/S}^+}$ seems to bring little (but statistically significant) improvement. However, we can gain better results by the typical fine-tuning strategies.  Among the fine-tuning experiments, updating all the parameters of the \tnet (or \std), i.e. $\text{NN}_{\text{W}} \to \text{NN}_{\text{S}}$, is the best fine-tuning strategy. Updating only the parameters of the representation layer based on the strong labels, i.e. $\text{NN}_{\text{W}} \to \text{NN}^{\text{Rep}}_{\text{S}}$, works better than updating only parameters of the supervision layer, i.e. $\text{NN}_{\text{W}} \to \text{NN}^{\text{Sup}}_{\text{S}}$. This supports our designed choice of a shared embedding layer in \cws which gets updated on set $V$.

\fwl is the best performing baselines, and \cws achieves 97\% of the performance of \fwl. The main advantage of \cws over \fwl is that it is trained in a single stage process and needs to meet the examples in $\mathcal{D}_w$ (which is a reasonably large set) only one time, while \fwl has three sequential stages during training and it needs to iterate two times over all the examples in $\mathcal{D}_w$. Also employing a Gaussian Process as part of the model in \fwl limits its scalability, while the components of \cws are all neural networks, and this eases the increase in the capacity of the model. 
% With all that, we find \cws a much more efficient model in terms of training cost, with losing little to no performance.


\begin{table}[tbp]
% \renewcommand{\arraystretch}{1.1}
\caption{\label{tbl_variants_rank_cws}Performance of the variants of the \cws on different datasets for document ranking task. Baselines are described in Table~\ref{tbl_baselines}.}
\centering
\begin{adjustbox}{max width=\textwidth}
\begin{tabular}{r l l l l l}
\toprule
& \multirow{2}{*}{Method} &
\multicolumn{2}{c}{Robust04} & \multicolumn{2}{c}{ClueWeb}
\\ 
\cmidrule(lr){3-4} \cmidrule(lr){5-6}
& & \small{MAP} & \small{nDCG@20}
& \small{MAP} & \small{nDCG@20}
\\ \midrule
8 & \small{\cws}
&  \textbf{0.3017} & \textbf{0.4511}
&  0.1363 & \textbf{0.2444}
\\
9 & \small{CWS$_\text{JT}^+$} 
& 0.2786  & 0.4367  
& 0.1310  & 0.2244 
\\ 
10 & \small{CWS$_\text{ST}$} 
&  0.2716  & 0.4237 
&  0.1320  & 0.2213
\\
11 & \small{\cws$_\text{CT}$} 
&  0.2961 & 0.4440 
&  \textbf{0.1378}  & 0.2431 
\\ 
12 & \small{\cws$_\text{PT}$} 
& 0.2784  & 0.4292  
& 0.1314  & 0.2207
\\\bottomrule
\end{tabular}
\end{adjustbox}
\end{table}


As an ablation study on \cws, we tried different training strategies and report the results in Table~\ref{tbl_variants_rank_cws}. As shown, \cws and CWS$_\text{CT}$ perform better than other strategies.
%
\cws$_\text{CT}$ is to let the \cnet to be trained separately, while still being able to enjoy shared learned information from the \tnet. Compared to \cws, \cws$_\text{CT}$ is less efficient as we need two rounds of training on weakly labeled data. 

While it seems reasonable to make use of strong labels for updating \emph{all} parameters of the \tnet, CWS$_\text{JT}^+$ achieves no better results than \cws. We speculate that during training, the direction of the parameter optimization is profoundly affected by the type of supervision signal and while we control the magnitude of the gradients, we do not change their directions. Hence alternating between two sets with different label qualities (different supervision signal types, i.e. weak and strong) confuses the supervision layer of the \tnet. 

In \cws$_\text{ST}$,  the strong dataset, $\mathcal{D}_s$, is too small to train a high-quality \cnet without taking advantage of the vast amount of weakly annotated data in $\mathcal{D}_w$ to learn better representations, so CWS$_\text{ST}$ is not able to improve the performance over $\text{NN}_W$ significantly and also we noticed that this strategy leads to a slow convergence compared to the $\text{NN}_W$. 
Also transferring learned information from \tnet to \cnet via progressive training, i.e., CWS$_\text{PT}$, performs no better than full sharing of the representation learning layer.



\begin{table}[tbp]
% \renewcommand{\arraystretch}{1.1}
\caption{\label{tbl_variants_rank_fwl}Performance of the \fwl against some of the baselines on different datasets for document ranking task. Baselines are described in Table~\ref{tbl_baselines}.}
\centering
\begin{adjustbox}{max width=\textwidth}
\begin{tabular}{r l c c c c}
\toprule
& \multirow{2}{*}{\textbf{Method}} &
\multicolumn{2}{c}{\textbf{Robust04}} & \multicolumn{2}{c}{\textbf{ClueWeb}}
\\ 
\cmidrule(lr){3-4} \cmidrule(lr){5-6}
& & \small{\textit{MAP}} & \bf \small{\textit{nDCG@20}}
&  \small{\textit{MAP}} & \bf \small{\textit{nDCG@20}}
\\ \midrule
\bf 13 & \bf \small{\fwl}
& \textbf{0.3124}  & \textbf{0.4607}
& \textbf{0.1472} & \textbf{0.2453}
\\
\bf 14 & \bf  \small{$\text{NN}_{\text{W}^\omega \to \text{NN}_\text{S}}$}
&  0.2899 & 0.4431
&  0.1320 & 0.2309
\\ 
\bf 15 & \bf \small{\fwl$_{unsuprep}$} 
&  0.2211 & 0.3700
&  0.0831 & 0.1964
\\
\bf 16 & \bf \small{\fwl$\backslash\Sigma$} 
&  0.2980 & 0.4516
&  0.1386 & 0.2340
\\\bottomrule
\end{tabular}
\end{adjustbox}
\end{table}
Table~\ref{tbl_variants_rank_fwl} presents the results of some experiments we have done as ablation studies on \fwl. 

Weighting the gradient updates from weak labels during pretraining and fine-tuning the network with strong labels, i.e. NN$_{\text{W}^\omega \to \text{S}}$ seems to work quite well.
%
Comparing the performance of \fwlnospace$_{unsuprep}$ to \fwl indicates that, first of all learning the representation of the input data downstream of the main task leads to better results compared to a task-independent unsupervised or self-supervised way. Also the dramatic drop in the performance compared to the \fwl, emphasizes the importance of the preretraining the \std on weakly labeled data.

%
We can gain improvement by fine-tuning the NN$_\text{W}$ using labels generated by the \tch without considering their confidence score, i.e. \fwl$\backslash\Sigma$. This means we just augmented the fine-tuning process by generating a fine-tuning set using \tch which is better than $\mathcal{D}_s$ in terms of quantity and $\mathcal{D}_w$ in terms of quality. This baseline is equivalent to setting $\beta = 0$ in Equation~\ref{eqn:eta2}. However, we see a big jump in performance when we use \fwl to include the estimated label quality from the \tch, leading to the best overall results.

\subsection{Sentiment Classification}
In sentiment classification, the goal is to predict the sentiment (e.g., positive, negative, or neutral) of a sentence. Each training sample $x$ consists of a sentence $s$ and its sentiment label $\tilde{y}$.

\begin{figure}[t!]
    \centering
            \includegraphics[width=0.35\textwidth]{03-part-02/chapter-05/figs_and_tables/fig_sentiment.pdf}
    \caption{The sentiment classifier used as \tch in \cws and \std in \fwl.}
    \label{fig:sentiment}
\end{figure}


\subsubsection{The \tnet in \cws and the \std in \fwl}
We use a convolutional model~\citep{Kim:2014} as the \tnet in \cws and the \std in \fwl, which is similar to the state-of-the-art model for Twitter sentiment classification from Semeval 2015 and 2016~\cite{Severyn:2015:SemEval,Deriu2016:SemEval,Deriu:2017,Severyn:2015:SIGIR}.

\mypar{The Representation Learning Layer} 
The representation learning layer in this task consists of an embedding function $\varepsilon: \mathcal{V} \rightarrow \mathbb{R}^{m}$, where $\mathcal{V}$ denotes the vocabulary set and $m$ is the number of embedding dimensions.

This function maps the sentence to a matrix $S \in \mathbb{R}^{m \times |s|}$, where each column represents the embedding of a word at the corresponding position in the sentence. We initialize the embedding matrix with word2vec embeddings~\cite{Mikolov:2013} pretrained on a collection of 50M tweets.

Matrix $S$ is passed through a convolution layer.  In this layer, a set of $f$ filters is applied to a sliding window of length $h$ over $S$ to generate a feature map matrix $O$. Each feature map $o_i$ for a given filter $F$ is generated by $o_i = \sum_{k,j}S[i:i+h]_{k,j} F_{k,j}$, where $S[i:i+h]$ denotes the concatenation of word vectors from position $i$ to $i+h$. The concatenation of all $o_i$ produces a feature vector $o \in \mathbb{R}^{|s|-h+1}$. The vectors $o$ are then aggregated over all $f$ filters into a feature map matrix $O \in \mathbb{R}^{f\times(|s|-h+1)}$.

We also add a bias vector $b \in R^f$ to the result of a convolution.
Each convolutional layer is followed by a non-linear activation function (we use ReLU\cite{Nair:2010}) which is applied element-wise. Afterward, the output is passed to the max pooling layer which operates on columns of the feature map matrix $O$ returning the largest value: $pool(o_i) : \mathbb{R}^{1\times(|s|-h+1)} \rightarrow \mathbb{R}$.

\mypar{The Supervision Layer.} 
This layer is a simple fully connected feed-forward network with $l$ hidden layers, followed by a softmax.  We employ the weighted cross entropy loss:
\begin{equation}
% \nonumber
\mathcal{L}_t = \sum_{i\in B_U} \tilde{c}_i \sum_{k \in K} - \tilde{y}_i^k \log (\hat{y}_i^k),
\end{equation}
where $B_U$ is a batch of instances from $U$, and $\tilde{c}_i$ is the confidence score of the weakly annotated instance $i$, and $K$ is a set of classes. 
The general schema of the \tnet (or \std) is illustrated in Figure~\ref{fig:sentiment}.

\subsubsection{The \wa}
\label{sentiment-WA}
The \wa for the sentiment classification task is a simple lexicon-based method~\citep{Hamdan:2013,Kiritchenko:2014}.
We use SentiWordNet03~\citep{Gaccianella:2010} to assign probabilities (positive, negative and neutral) for each token in set $\mathcal{D}_w$. We use a bag-of-words model for the sentence-level probabilities (i.e.\ just averaging the distributions of the terms), yielding a noisy label $\tilde{y}_i \in \mathbb{R}^{|K|}$, where $|K|=3$ is the number of classes.  We found empirically that using soft labels from the \wa works better than assigning a single hard label.


\subsubsection{The \cnet in \cws}
In this task, the \cnet is also a regresses and we use a simple fully connected feed-forward network. The target label $c_j$ for the \cnet is calculated by using the mean absolute difference of the strong label and the weak label: $c_j= 1-\frac{1}{|K|}\sum_{k\in K}|y_j^k - \tilde{y}_j^k|$, where $y_j$ is the one-hot encoding of the sentence label over all classes.


\subsubsection{The \tch in \fwl}
Similar to the ranking task, we use Gaussian Process as the \tch in order to generate soft labels. We pass the mean of $\mathcal{GP}$ through the same function $g(.)$ that is applied on the output of the \std network, where the $g(.)$ is softmax for the sentiment classification task.
Here in this task, $h(.)$ is an aggregation function that takes variance over several dimensions and outputs a single measure of variance. As a reasonable choice, the aggregating function $h(.)$ in the sentiment classification task (three classes) is \emph{mean} of variances over dimensions. 
In the \tch, linear combinations of different kernels are used. For the sentiment classification task, We use sparse variational GP for multiclass classification\footnote{\url{http://gpflow.readthedocs.io/en/latest/notebooks/multiclass.html}}~\citep{hensman2014scalable} with the following kernel:
\begin{equation}
k(x_i,x_j)=k_{\rm RBF}(x_i,x_j)+{k_{\rm Linear}}(x_i,x_j)+k_{\rm White}(x_i,x_j)
\end{equation}
where,
\begin{flalign*}
    \hspace{6em}
    &&k_{\rm RBF}(x_i,x_j) &= \exp{\left(\frac{\Vert x_i-x_j\Vert^2}{2l^2}\right)} & 
    \\
    &&k_{\rm Linear}(x_i,x_j) &= \sigma_0^2+x_i.x_j & \\
    &&k_{\rm White}(x_i,x_j) &= constant\_value, \quad \forall x_1=x_2 \text{ and } 0 \text{ otherwise} & 
\end{flalign*}

Similar to the ranking task, we set $l=1$ the length scale of RBF kernel, set $\sigma_0 = 0$  for the linear kernel, and set the number of clusters to $30$ in clustered $\mathcal{GP}$ algorithm.


\subsubsection{Collections}
We test our model on the twitter message-level sentiment classification of SemEval-15 Task 10B \citep{rosenthal:2015}. Datasets of SemEval-15 subsume the test sets from previous editions of SemEval, i.e. SemEval-13 and SemEval-14. Each tweet was preprocessed so that URLs and usernames are masked.

\mypar{Data with strong labels.} 
We use train (9,728 tweets) and development (1,654 tweets) data from SemEval-13 for training and SemEval-13-test (3,813 tweets) for validation.
To make your results comparable to the official runs on SemEval we us SemEval-14 (1,853 tweets) and  SemEval-15 (2,390 tweets) as test sets~\citep{rosenthal:2015, Nakov:2016}.

\mypar{Data with weak labels.}
We use a large corpus containing 50M tweets collected during two months for both, training the word embeddings and creating the weakly annotated set $\mathcal{D}_w$ using the lexicon-based method explained in Section~\ref{sentiment-WA}. 

\subsubsection{Experimental Setup.}
Similar to the document ranking task, we tuned hyper-parameters for the \tnet in \cws (and \std in the first step of \fwl) with respect to the strong labels of the validation set using batched GP bandits with an expected improvement acquisition function~\citep{Desautels:2014} and kept the optimal parameters fixed for all the other experiments.  
The size and number of hidden layers for the classifier and is selected from $\{32, 64, 128\}$.
We tested the model with both, $1$ and $2$ convolutional layers. The number of convolutional feature maps and the filter width is selected from $\{200,300\}$ and $\{ 3, 4, 5\}$, respectively. The initial learning rate and the dropout parameter were selected from $\{1E-3, 1E-5\}$ and $\{0.0, 0.2, 0.5\}$, respectively. We considered embedding sizes of $\{100, 200\}$ and the batch size in these experiments was set to $64$. ReLU~\citep{Nair:2010} is used as a non-linear activation function in \tnet (and \std).  Adam optimizer~\citep{Kingma:2014} is used for training, and \emph{dropout}~\citep{Srivastava:2014} as a regularizer.

In the rest of the chapter, we will present the main results of the introduced baseline methods and the proposed models, \cws and \fwl, 


\subsubsection{Results and Discussions} 
\begin{table}[!t]
            \renewcommand{\arraystretch}{1.1}
            \centering
            \caption{\label{tbl_main_sent}Performance ofof \cws and \fwl as well as the main baseline methods,described in Table~\ref{tbl_baselines}, for the sentiment classification task. 
            \pssmall{i} indicates that the improvements with respect to the baseline $i$ are statistically significant at the 0.05 level using the paired two-tailed t-test with Bonferroni correction.}
            \begin{tabular}{r l l l}
            \toprule
            & \textbf{Method} & \textbf{SemEval-14} & \textbf{SemEval-15}
            \\ \midrule
            \bf 1 & \bf \small{WA$_\text{Lexicon}$} 
            & 0.5141 & 0.4471
            \\ \midrule
           \bf  2 & \bf \small{$\text{NN}_{\text{S}}$} 
            & 0.6307\pssmall{1} & 0.5811\pssmall{13}
            \\
            \bf 3 & \bf \small{$\text{NN}_{\text{W}}$} 
            & 0.6719\pssmall{12} & 0.5606\pssmall{1} 
            \\ \midrule
            \bf 4 & \bf \small{$\text{NN}_{\text{W}\text{/S}^+}$} 
            & 0.7032\pssmall{12367} & 0.6319\pssmall{12367}
            \\
            \bf 5 & \bf \small{$\text{NN}_{\text{W}} \to \text{NN}_{\text{S}}$}
            & 0.7080\pssmall{12367} & 0.6441\pssmall{12367}
            \\
           \bf  6 & \bf \small{$\text{NN}_{\text{W}} \to \text{NN}^{\text{Sup}}_{\text{S}}$} 
            & 0.6875\pssmall{123} & 0.6193\pssmall{123}
            \\
            \bf 7 & \bf \small{$\text{NN}_{\text{W}} \to \text{NN}^{\text{Rep}}_{\text{S}}$}
            & 0.6932 \pssmall{123} & 0.6102\pssmall{123}
            \\ \midrule
            \bf 8 & \bf \small{\cws} 
            & 0.7362 \pssmall{1234567} & 0.6626\pssmall{1234567}
            \\
            \bf 13 & \bf \small{\fwl} 
            & \textbf{0.7470} \pssmall{12345678} & \textbf{0.6830}\pssmall{12345678}
            \\ \midrule
            \bf $\ast$ & \bf \small{SemEval$^\text{Best}$} 
            & 0.7162~\citep{Rouvier:2016} & 0.6618~\citep{Deriu2016:SemEval}
            \\\bottomrule
            \end{tabular}
\end{table}
We report Macro-F1, the official SemEval metric, in Table~\ref{tbl_main_sent}. 

Among all the baselines, \fwl is the best performing approach. \cws is also outperforms all the baselines. 

For this task, since the amount of data with strong labels are larger compared to the ranking task, the performance of $\text{NN}_{\text{S}}$ is acceptable. Alternately sampling from weak and strong data, i.e.  $\text{NN}_{\text{W}\text{/S}^+}$ gives better results than either of learning from just weak or just strong labels. However, pretraining on weak labels then fine-tuning both the supervision layer and the representation learning layer on strong labels, further improves the performance.  

Besides the baselines, we also report the best performing systems which are also convolution-based models (\citealt{Rouvier:2016} on SemEval-14; \citealt{Deriu2016:SemEval} on SemEval-15). Both \cws and \fwl outperform these methods.


\begin{table}[!t]
            \renewcommand{\arraystretch}{1.1}
            \centering
            \caption{\label{tbl_variants_sent_cws}Performance of the variants of the \cws on different datasets for the sentiment classification task. Baselines are described in Table~\ref{tbl_baselines}.}
            \begin{tabular}{r l c c}
            \toprule
            & \textbf{Method} & \textbf{SemEval-14 }& \textbf{SemEval-15}
            \\ \midrule
            \bf 8 & \bf \small{\cws} 
            & 0.7362 & 0.6626
            \\
            \bf 9 & \bf \small{\cws$_\text{JT+}$} 
            & 0.7310 & 0.6551
            \\
            \bf 10 & \bf \small{\cws$_\text{ST}$} 
            & 0.7183 & 0.6501
            \\
            \bf 11 & \bf \small{\cws$_\text{CT}$} 
            & \textbf{0.7363} & \textbf{0.6667}
            \\
            \bf 12 & \bf \small{\cws$_\text{PT}$}
            & 0.7009 & 0.6118
            \\\bottomrule
            \end{tabular}
\end{table}
Similar to the ranking task, we have done an ablation study on \cws by trying different strategies for training \cws. The results of these experiments are presented in Table~\ref{tbl_variants_sent_cws}. \cws$_\text{CT}$ archives the highest performance among all the training strategies, however, as we discussed in Section~\ref{sec:res_and_disc_ranking}, it is not as efficient as \cws. 

In sentiment classification, compared to the ranking task, it is easier to estimate the confidence score of instances concerning the amount of available supervised data. Therefore, \cws$_\text{ST}$ improves the performance over $\text{NN}_{\text{S}}$ in Table~\ref{tbl_main_sent}. 


\begin{table}[!t]
            \renewcommand{\arraystretch}{1.1}
            \centering
            \caption{\label{tbl_variants_sent_fwl}Performance of the \fwl against some of the baselines on different datasets for document ranking task. Baselines are described in Table~\ref{tbl_baselines}.}
            \begin{tabular}{r l l l}
            \toprule
            & Method & SemEval-14 & SemEval-15
            \\ \midrule
            13 & \small{\fwl} 
            & \textbf{0.7470} & \textbf{0.6830}
            \\
            14 & \small{$\text{NN}_{\text{W}^\omega \to \text{NN}_\text{S}}$} 
            & 0.7166 & 0.6603
            \\
            15 & \small{\fwl$_{unsuprep}$} 
            & 0.6588  & 0.6954
            \\ 
            16 & \small{\fwl$\backslash\Sigma$} 
            & 0.7202 & 0.6590
            \\\bottomrule
            \end{tabular}
\end{table}
The results of a set of experiments we have done as ablation studies on \fwl is presented Table~\ref{tbl_variants_sent_fwl}. 

Having static weighting on the gradient updates, i.e. NN$_{\text{W}^\omega \to \text{S}}$, leads to the performance that is better than simple fine-tuning, i.e. $\text{NN}_{\text{W}} \to \text{NN}_{\text{S}}$ in Table~\ref{tbl_main_sent} .
%
For this task, similar to the ranking task, learning the representation in an unsupervised task independent fashion, i.e. \fwlnospace$_{unsuprep}$, does not lead to good results compared to the \fwl.
%
Similar to the ranking task, fine-tuning $\text{NN}_{\text{W}}$ based on labels generated by $\mathcal{GP}$ instead of data with strong labels, regardless of the confidence score, i.e. \fwl$\backslash\Sigma$, works better than standard fine-tuning. 

\section{Conclusion}
