\begin{filecontents*}{lambdas_robust.dat}
r	g	s	Rank	Relevancy
0.6224241099	0.1779218276	0.1996540625	1	0.4530612245
0.6426362058	0.1473899680	0.2099738262	2	0.4081632653
0.5525444789	0.1730506908	0.2744048303	3	0.3755102041
0.5411322672	0.1820225045	0.2768452283	4	0.3428571429
0.5210824792	0.1975803060	0.2813372148	5	0.3673469388
0.5118161280	0.1930682979	0.2951155741	6	0.3591836735
0.5831464279	0.1992121152	0.2176414569	7	0.3428571429
0.4088451139	0.2128806819	0.3782742042	8	0.3142857143
0.4517538645	0.2089034512	0.3393426843	9	0.2897959184
0.4098503759	0.2218293882	0.3683202359	10	0.3142857143
0.3878871094	0.2172312660	0.3948816246	11	0.2612244898
0.3730045960	0.2353828070	0.3916125970	12	0.3102040816
0.3066243667	0.2304501728	0.4629254605	13	0.2530612245
0.3558582283	0.2310046328	0.4131371389	14	0.2448979592
0.3763913572	0.2314746506	0.3921339922	15	0.2775510204
0.2622349920	0.2357348530	0.5020301550	16	0.2367346939
0.3602182256	0.2290039194	0.4107778550	17	0.2448979592
0.2444926310	0.2308743994	0.5246329696	18	0.2448979592
0.3687477402	0.2233535585	0.4078987013	19	0.2163265306
0.3472998620	0.2343988725	0.4183012655	20	0.2204081633
0.2819937021	0.2402389853	0.4777673126	21	0.1755102041
0.3298199639	0.2387412628	0.4314387733	22	0.2040816327
0.3149953726	0.2451694944	0.4398351330	23	0.1836734694
0.3065805988	0.2390820320	0.4543373692	24	0.1591836735
0.3111626539	0.2415567440	0.4472806021	25	0.1591836735
0.3060792208	0.2469323954	0.4469883838	26	0.1755102041
0.2993611646	0.2440818421	0.4565569933	27	0.1795918367
0.2359534570	0.2676858803	0.4963606627	28	0.1673469388
0.2981112586	0.2540941109	0.4477946305	29	0.1755102041
0.2410236493	0.2709179730	0.4880583777	30	0.1918367347
0.2638534861	0.2771327601	0.4590137538	31	0.1265306122
0.2950850992	0.2697397862	0.4351751146	32	0.1795918367
0.3143111240	0.2766006603	0.4090882157	33	0.1469387755
0.3242929006	0.2784513925	0.3972557069	34	0.1428571429
0.3292636476	0.2756283377	0.3951080147	35	0.1591836735
0.2962000981	0.2907822424	0.4130176595	36	0.1836734694
0.2975661518	0.3158355545	0.3865982937	37	0.1306122449
0.2086470015	0.3031070198	0.4882459787	38	0.1020408163
0.2865840358	0.2949713301	0.4184446341	39	0.1102040816
0.2704112948	0.3108802520	0.4187084532	40	0.1755102041
0.1847627028	0.3188147630	0.4964225342	41	0.1265306122
0.2094327260	0.3229180700	0.4676492040	42	0.1387755102
0.2765066826	0.3214288539	0.4020644635	43	0.1428571429
0.2599422814	0.3386503803	0.4014073383	44	0.1714285714
0.2581763990	0.3356641822	0.4061594188	45	0.1306122449
0.2611927486	0.3392808290	0.3995264224	46	0.1632653061
0.1901272813	0.3539525186	0.4559202001	47	0.1265306122
0.2249767410	0.3597600859	0.4152631731	48	0.1142857143
0.2440178124	0.3418748254	0.4141073622	49	0.1346938776
0.1865884388	0.3604182412	0.4529933200	50	0.1510204082
0.2111572755	0.3701620098	0.4186807147	51	0.1102040816
0.2398984845	0.3500907404	0.4100107751	52	0.1510204082
0.2263180739	0.3780923340	0.3955895921	53	0.1265306122
0.1896843606	0.3783724923	0.4319431471	54	0.1306122449
0.2040545829	0.3970911484	0.3988542687	55	0.1102040816
0.1891437206	0.3909896854	0.4198665940	56	0.1102040816
0.1776809706	0.4042295063	0.4180895231	57	0.1265306122
0.1927379327	0.4020256137	0.4052364536	58	0.1020408163
0.1969253080	0.4031350958	0.3999395962	59	0.1265306122
0.2009469375	0.3956135046	0.4034395579	60	0.1598360656
0.2040873842	0.4085731057	0.3873395101	61	0.1714285714
0.1883305335	0.4110514251	0.4006180414	62	0.1387755102
0.1671893448	0.4163244050	0.4164862502	63	0.1510204082
0.1815204276	0.4176602168	0.4008193556	64	0.1358024691
0.1377621502	0.4306426117	0.4315952381	65	0.1102040816
0.1110857010	0.4373438557	0.4515704433	66	0.106122449
0.1580784528	0.4392289389	0.4026926083	67	0.1265306122
0.1404100813	0.4422031857	0.4173867330	68	0.1224489796
0.1024015719	0.4469577919	0.4506406362	69	0.1142857143
0.1519074272	0.4584158162	0.3896767566	70	0.1142857143
0.1472505853	0.4890239401	0.3637254746	71	0.106122449
0.1367158478	0.4895998703	0.3736842819	72	0.1224489796
0.1287680883	0.5036763255	0.3675555862	73	0.0734693878
0.1222225550	0.5023617080	0.3754157370	74	0.0816326531
0.1269140085	0.5138313053	0.3592546862	75	0.1020408163
0.1361584233	0.5196033876	0.3442381891	76	0.1102040816
0.1579488086	0.5335885901	0.3084626013	77	0.1265306122
0.1337466104	0.5403031557	0.3259502339	78	0.0857142857
0.1448103875	0.5333709128	0.3218186997	79	0.0979591837
0.1216163739	0.5347422198	0.3436414063	80	0.1102040816
0.1112549642	0.5532337904	0.3355112454	81	0.0979591837
0.1151786101	0.5699385578	0.3148828321	82	0.093877551
0.1195264690	0.5632578156	0.3172157154	83	0.0897959184
0.1091824516	0.5786754122	0.3121421362	84	0.0897959184
0.1191610923	0.5877774152	0.2930614925	85	0.093877551
0.1155240962	0.5794103583	0.3050655455	86	0.1020408163
0.1018571674	0.5849405772	0.3132022554	87	0.1387755102
0.1271077810	0.5958470681	0.2770451509	88	0.1102040816
0.1148793070	0.6022115579	0.2829091351	89	0.0816326531
0.1189715188	0.6090412676	0.2719872136	90	0.1020408163
0.1124031197	0.6297335528	0.2578633275	91	0.0897959184
0.0960319806	0.6393838915	0.2645841279	92	0.0612244898
0.0474686817	0.6364172637	0.3161140546	93	0.1020408163
0.1126303277	0.6358426998	0.2515269725	94	0.1142857143
0.0950936651	0.6530304825	0.2518758524	95	0.0979591837
0.1010769251	0.6409733309	0.2579497440	96	0.0857142857
0.1035328555	0.6624286689	0.2340384756	97	0.0816326531
0.0810918980	0.6836942225	0.2352138795	98	0.0816326531
0.0992808565	0.6727398375	0.2279793060	99	0.0816326531
0.0390425942	0.6880831681	0.2728742377	100	0.106122449
\end{filecontents*}

%
%


\begin{filecontents*}{lambdas_wt10g.dat}
r	g	s	Rank	Relevancy
0.5944499414	0.1639766346	0.241573424	1	0.6464646465
0.5031125469	0.1492254315	0.3476620216	2	0.6020408163
0.4522001636	0.1650201457	0.3827796908	3	0.4693877551
0.2919563871	0.2279825503	0.4800610626	4	0.4081632653
0.3565017411	0.1520367888	0.4914614701	5	0.3979591837
0.3328337699	0.1331667682	0.533999462	6	0.2653061224
0.2887515776	0.1541819648	0.5570664576	7	0.3469387755
0.2180692885	0.1936531383	0.5882775732	8	0.4183673469
0.1519649001	0.1990223174	0.6490127826	9	0.387755102
0.2677724446	0.1340113773	0.5982161781	10	0.306122449
0.2114266434	0.1852944289	0.6032789276	11	0.193877551
0.1973462761	0.2063881214	0.5962656025	12	0.2857142857
0.1835269409	0.248375401	0.5680976581	13	0.3163265306
0.1801629853	0.3582170641	0.4616199506	14	0.2551020408
0.1772115339	0.2993905042	0.5233979619	15	0.3163265306
0.1750202901	0.3652545596	0.4597251503	16	0.2755102041
0.1736895679	0.2940047957	0.5323056365	17	0.2551020408
0.1733937243	0.3005246264	0.5260816493	18	0.3979591837
0.1730656852	0.390270756	0.4366635587	19	0.2857142857
0.1706839208	0.4958213204	0.3334947589	20	0.2448979592
0.1482671154	0.4965871877	0.3551456969	21	0.306122449
0.1653595951	0.475603364	0.3590370409	22	0.2448979592
0.1564484869	0.5975685536	0.2459829595	23	0.1836734694
0.1694270885	0.5608529141	0.2697199974	24	0.2755102041
0.1518554672	0.5992033104	0.2489412224	25	0.2142857143
0.1501563152	0.663782192	0.1860614929	26	0.1632653061
0.1158509598	0.5156355125	0.3685135277	27	0.2040816327
0.1477135399	0.655369879	0.1969165811	28	0.1428571429
0.1403665666	0.4867727642	0.3728606692	29	0.2346938776
0.1248981079	0.6033091459	0.2717927462	30	0.1836734694
0.1225225621	0.6189259979	0.25855144	31	0.1734693878
0.086617717	0.6931411632	0.2202411198	32	0.1224489796
0.109639783	0.8891298275	0.0012303895	33	0.1428571429
0.1059290796	0.7884022185	0.1056687019	34	0.193877551
0.1054133543	0.7645399135	0.1300467322	35	0.1632653061
0.1030489512	0.6104793259	0.2864717229	36	0.1734693878
0.1030489512	0.6104793259	0.2864717229	37	0.2448979592
0.10090606	0.895968753	0.003125187	38	0.193877551
0.1213246119	0.7206649208	0.1580104674	39	0.1020408163
0.1195544818	0.837578834	0.0428666841	40	0.1428571429
0.102298453	0.6278687839	0.2698327631	41	0.193877551
0.1152625707	0.7987521811	0.0859852482	42	0.1734693878
0.1145269718	0.7483309368	0.1371420914	43	0.2346938776
0.1110374997	0.675673424	0.2132890763	44	0.1836734694
0.1072151377	0.7454184713	0.147366391	45	0.1428571429
0.1050519274	0.6431103032	0.2518377694	46	0.2244897959
0.1037165607	0.8935691297	0.0027143095	47	0.1020408163
0.1024729216	0.7673437945	0.130183284	48	0.112244898
0.0976787158	0.7427348264	0.1595864578	49	0.2040816327
0.1090554418	0.5619866712	0.328957887	50	0.2040816327
0.1071555191	0.6562923193	0.2365521616	51	0.1428571429
0.1071253868	0.8857410241	0.0071335891	52	0.1428571429
0.1062531087	0.8859194833	0.0078274079	53	0.2040816327
0.1041029108	0.8630158669	0.0328812223	54	0.2346938776
0.1037399193	0.7351675998	0.1610924809	55	0.1734693878
0.1023209606	0.8778115624	0.019867477	56	0.1632653061
0.1012405087	0.8762344726	0.0225250186	57	0.1632653061
0.0984182407	0.8212896945	0.0802920648	58	0.0918367347
0.0892254315	0.4631125469	0.4476620216	59	0.112244898
0.0966674052	0.880050206	0.0232823888	60	0.1428571429
0.0944399782	0.7061168764	0.1994431454	61	0.193877551
0.093657759	0.7234523257	0.1828899153	62	0.0816326531
0.093657759	0.7234523257	0.1828899153	63	0.1530612245
0.0907494616	0.8072398896	0.1020106488	64	0.1836734694
0.074351592	0.5114620713	0.4141863366	65	0.0714285714
0.0989954826	0.887537232	0.0134672855	66	0.1632653061
0.0976532201	0.7748607154	0.1274860645	67	0.2040816327
0.096997791	0.8486963936	0.0543058154	68	0.1326530612
0.090499773	0.9012625245	0.0082377026	69	0.1428571429
0.0904219822	0.8628879429	0.0466900749	70	0.1326530612
0.0876549798	0.8259569778	0.0863880424	71	0.1020408163
0.0844246594	0.7843029937	0.1312723469	72	0.1020408163
0.0831524903	0.7169792451	0.1998682646	73	0.1020408163
0.0825852384	0.9146879701	0.0027267915	74	0.193877551
0.0807844969	0.6636290007	0.2555865024	75	0.1530612245
0.0788245322	0.7502788202	0.1708966477	76	0.1428571429
0.076960232	0.9003617111	0.0226780569	77	0.112244898
0.0747399125	0.7914911628	0.1337689248	78	0.1530612245
0.0707844969	0.6716290007	0.2575865024	79	0.1530612245
0.0733751004	0.7240723828	0.2025525168	80	0.1326530612
0.0588944014	0.9075535383	0.0335520604	81	0.1020408163
0.0703661746	0.757300132	0.1723336934	82	0.1632653061
0.0679864276	0.6894026342	0.2426109382	83	0.1326530612
0.0673497468	0.6964562768	0.2361939764	84	0.1020408163
0.0661494604	0.9096111086	0.024239	85	0.0918367347
0.0608860126	0.9025409688	0.0365730185	86	0.0918367347
0.0608860126	0.9005409688	0.0385730185	87	0.112244898
0.0608860126	0.8985409688	0.0405730185	88	0.112244898
0.0607885485	0.9071310051	0.0320804	89	0.1020408163
0.0591006316	0.9068296985	0.0340697	90	0.0918367347
0.0558679011	0.9077162252	0.036416	91	0.0918367347
0.0582240585	0.9037215314	0.0380544	92	0.0816326531
0.0561723794	0.8828195488	0.0610080718	93	0.1326530612
0.0559569654	0.8017070209	0.1423360137	94	0.0918367347
0.0533121488	0.8863208563	0.0603669949	95	0.1020408163
0.0547755285	0.8267466996	0.1184777718	96	0.1836734694
0.050503	0.8879059718	0.0615908331	97	0.1020408163
0.052142618	0.8975203132	0.050337	98	0.0918367347
0.050753	0.8971429735	0.052104	99	0.1326530612
0.050753	0.8951429735	0.054104	100	0.1224489796
\end{filecontents*}
%
%


\begin{filecontents*}{lambdas_gov2.dat}
r	g	s	Rank	Relevancy
0.6240429338	0.1215850376	0.2543720286	1	0.8
0.6494116194	0.116543642	0.2340447386	2	0.9066666667
0.5783448829	0.1143307512	0.3073243659	3	0.68
0.6111375806	0.110363857	0.2784985624	4	0.72
0.6260825245	0.1124197085	0.261497767	5	0.64
0.5599872275	0.1082256682	0.3317871043	6	0.6666666667
0.5446212787	0.1190321386	0.3363465827	7	0.6666666667
0.5290221053	0.1642044	0.3067734947	8	0.5866666667
0.5624805788	0.1629857516	0.2745336696	9	0.7466666667
0.5565917586	0.1526511689	0.3007570725	10	0.6
0.5012604813	0.107274793	0.3914647257	11	0.7733333333
0.4901711183	0.1193441897	0.390484692	12	0.6266666667
0.4831683329	0.1184912387	0.3983404284	13	0.5466666667
0.4548808328	0.1841653416	0.3609538256	14	0.56
0.4448544231	0.1917282645	0.3634173124	15	0.64
0.4324067617	0.204043379	0.3635498593	16	0.6133333333
0.4213757803	0.1964128142	0.3822114055	17	0.68
0.4057269595	0.2257084868	0.3685645537	18	0.5333333333
0.3935940206	0.2395838497	0.3668221297	19	0.5866666667
0.3895879037	0.2435604172	0.3668516791	20	0.48
0.3884865208	0.2079975499	0.4035159293	21	0.64
0.3717917851	0.2133534138	0.4148548011	22	0.4133333333
0.3645486572	0.2668496468	0.368601696	23	0.4266666667
0.3490163773	0.2144800671	0.4365035556	24	0.5466666667
0.3386545526	0.2718250983	0.3895203491	25	0.56
0.3333142445	0.2762514784	0.3904342771	26	0.5466666667
0.3230485148	0.222687504	0.4542639812	27	0.4933333333
0.3224405242	0.219453848	0.4581056278	28	0.5733333333
0.3076330955	0.2519579916	0.4404089129	29	0.5066666667
0.3052704081	0.2795964922	0.4151330997	30	0.52
0.2981478151	0.2431673254	0.4586848595	31	0.4933333333
0.2913760942	0.2628450987	0.4457788071	32	0.44
0.2810583633	0.2799589808	0.4389826559	33	0.5066666667
0.2631866707	0.2401186067	0.4966947226	34	0.52
0.2454065981	0.2872395146	0.4673538873	35	0.56
0.2011464291	0.2392816098	0.5595719611	36	0.3866666667
0.1731699691	0.2996328951	0.5271971358	37	0.5333333333
0.1730660435	0.2529975206	0.5739364359	38	0.6666666667
0.1627781482	0.311873599	0.5253482528	39	0.5333333333
0.1459018494	0.3503070292	0.5037911214	40	0.4666666667
0.1436318138	0.334350479	0.5220177072	41	0.52
0.1375495421	0.4946033571	0.3678471008	42	0.4933333333
0.135452397	0.474835616	0.389711987	43	0.5333333333
0.1248057082	0.4497779514	0.4254163404	44	0.5066666667
0.1241668022	0.4435797725	0.4322534253	45	0.4266666667
0.1239215339	0.4355842656	0.4404942005	46	0.52
0.1244501125	0.4789034082	0.3966464793	47	0.52
0.1212520189	0.4099371966	0.4688107845	48	0.5466666667
0.1171634634	0.3882547738	0.4945817628	49	0.4133333333
0.1624695319	0.4171991089	0.4203313592	50	0.3866666667
0.1822230056	0.4046671526	0.4131098418	51	0.5066666667
0.217599659	0.371937064	0.410463277	52	0.3866666667
0.1938648765	0.3825139715	0.423621152	53	0.4
0.1003843085	0.3663869389	0.5332287526	54	0.4133333333
0.1100940565	0.4236216443	0.4662842992	55	0.28
0.1070990342	0.4112598387	0.4816411271	56	0.3866666667
0.0884481322	0.4110508737	0.5005009941	57	0.4133333333
0.0978005956	0.4098234204	0.492375984	58	0.4666666667
0.0881022594	0.33125851	0.5806392306	59	0.44
0.0910044627	0.4127665135	0.4962290238	60	0.4933333333
0.0920641372	0.3788757418	0.529060121	61	0.36
0.1017324028	0.3776740445	0.5205935527	62	0.3866666667
0.1030624895	0.3645340165	0.532403494	63	0.5733333333
0.0839594462	0.3507392598	0.565301294	64	0.4133333333
0.0893775211	0.3533380093	0.5572844696	65	0.32
0.0977353957	0.3616091183	0.540655486	66	0.3866666667
0.1044969578	0.4526371226	0.4428659196	67	0.2933333333
0.0957215194	0.4052024632	0.4990760174	68	0.4266666667
0.1015359918	0.4207085433	0.4777554649	69	0.4933333333
0.0742286648	0.5506943766	0.3750769586	70	0.4266666667
0.0804430889	0.4387348421	0.480822069	71	0.48
0.0717483124	0.3757572904	0.5524943972	72	0.52
0.0665695034	0.3370906151	0.5963398815	73	0.48
0.0523617119	0.4862797501	0.461358538	74	0.4133333333
0.0328926397	0.4562067306	0.5109006297	75	0.3866666667
0.0366886917	0.4686637271	0.4946475812	76	0.3733333333
0.0318689764	0.5178197566	0.450311267	77	0.44
0.0411597391	0.5314515056	0.4273887553	78	0.3066666667
0.0390001779	0.5049351276	0.4560646945	79	0.44
0.0387862001	0.5217786841	0.4394351158	80	0.44
0.0389243166	0.5363244649	0.4247512185	81	0.4133333333
0.0384996822	0.5753679266	0.3861323912	82	0.5066666667
0.0438130406	0.5245213476	0.4316656118	83	0.4
0.0448836228	0.5450246491	0.4100917281	84	0.32
0.0491279547	0.5759519554	0.3749200899	85	0.48
0.0470794607	0.5982121771	0.3547083622	86	0.4133333333
0.0560536164	0.621020159	0.3229262246	87	0.3333333333
0.061749195	0.60982985	0.328420955	88	0.4133333333
0.0587346237	0.6593585591	0.2819068172	89	0.44
0.0640313412	0.7674664323	0.1685022265	90	0.4266666667
0.0614565394	0.6809256894	0.2576177712	91	0.48
0.0712540903	0.6660105067	0.262735403	92	0.3466666667
0.069950477	0.7263508241	0.2036986989	93	0.4666666667
0.0495688685	0.7695673367	0.1808637948	94	0.28
0.0465794938	0.8054820035	0.1479385027	95	0.3466666667
0.0560818524	0.8360577571	0.1078603905	96	0.32
0.0424826209	0.8521777045	0.1053396746	97	0.3333333333
0.0487525904	0.8270298648	0.1242175448	98	0.3733333333
0.0532006111	0.8015227921	0.1452765968	99	0.32
0.0601366311	0.8027947191	0.1370686498	100	0.3866666667
\end{filecontents*}

%
%


\begin{filecontents*}{poison_pill_374.dat}
FBnum	RM1	RM2	RM3	RM4	DMM	MEDMM	SMM	RMM	SWLM	RSWLM
0	0.3111	0.3111	0.3111	0.3111	0.3111	0.3111	0.3111	0.3111	0.3111	0.3111
1	0.3213	0.3271	0.3404	0.3401	0.3202	0.3209	0.3305	0.3403	0.3350	0.3403
2	0.3312	0.3317	0.3553	0.3432	0.3278	0.3343	0.335	0.3472	0.3433	0.3569
3	0.3412	0.3413	0.3573	0.3423	0.3258	0.3426	0.345	0.3579	0.345	0.3614
4	0.3415	0.3435	0.3663	0.3521	0.3331	0.3523	0.3446	0.3629	0.3577	0.361
5	0.3416	0.3474	0.367	0.3593	0.3333	0.3601	0.355	0.3532	0.3642	0.3637
6	0.3527	0.3599	0.3691	0.3603	0.3384	0.3604	0.3548	0.3602	0.3729	0.3734
7	0.3404	0.3457	0.3558	0.3508	0.3328	0.3411	0.3511	0.3504	0.3722	0.372
8	0.3469	0.3477	0.3659	0.3509	0.3335	0.3503	0.3445	0.3537	0.378	0.3741
9	0.354	0.3554	0.3837	0.3647	0.335	0.3618	0.3655	0.3733	0.3792	0.3765
10	0.3599	0.364	0.3847	0.3742	0.3401	0.3605	0.3647	0.3813	0.3819	0.381
\end{filecontents*}

\begin{filecontents*}{poison_pill_lambdas.dat}
r	g	s	Rank
0	0	0	0
0.6039723417	0.0832124153	0.312815243	1
0.6139700877	0.0873340375	0.2986958748	2
0.6024729488	0.1338591861	0.2636678651	3
0.6019298903	0.0809957112	0.3170743985	4
0.5424375548	0.1720420172	0.285520428	5
0.6021874506	0.0852533346	0.3125592148	6
0.1636283719	0.1303344453	0.7060371828	7
0.6027343047	0.1006031758	0.2966625195	8
0.6021624409	0.1127874811	0.285050078	9
0.6020473736	0.1833130873	0.2146395391	10
\end{filecontents*}


\begin{filecontents*}{divergence_robust04.dat}
qnum	qnump	relnum	RM3	RM4	DMM	MEDMM	SMM	RMM	SWLM	RSWLM
%19	0.19	0	0	0	0	0	0	0	0	0
13	0.13	0.1	0.5482763082	0.5295415441	0.580231146	0.4961943631	0.5601747558	0.4437263295	0.5601747558	0.4437263295
21	0.21	0.2	0.3526459451	0.3139493165	0.4247525539	0.4091586117	0.3697867312	0.2852892626	0.3397867312	0.2052892626
13	0.13	0.3	0.2568973807	0.2648744295	0.3507605682	0.3073875692	0.2942537185	0.1926366746	0.213383534	0.1326366746
10	0.1	0.4	0.2068945659	0.1983427124	0.2453318636	0.1940917938	0.2147962981	0.1777585838	0.169873114	0.1377585838
5	0.05	0.5	0.1555468429	0.144353691	0.1505631198	0.1455482095	0.1430094145	0.1133591349	0.072628612	0.0733591349
5	0.05	0.6	0.1332493192	0.1391561178	0.1673952484	0.1284836151	0.1469603736	0.1055820954	0.0463837594	0.0458209541
5	0.05	0.7	0.1109948742	0.1073875692	0.1377722647	0.1048448979	0.0963746846	0.083615481	0.0359813981	0.0333615481
6	0.06	0.8	0.0827129794	0.0740993817	0.0959011953	0.0726366746	0.0508677746	0.0484979484	0.0098823252	0.0084979484
2	0.02	0.9	0.0123873934	0.0130054195	0.0249082067	0.0067758588	0.0094644475	0.0087103696	0.0021507977	0.0027036964
1	0.01	1	0	0	0	0	0	0	0	0
\end{filecontents*}

\begin{filecontents*}{divergence_wt10g.dat}
qnum	qnump	relnum	RM3	RM4	DMM	MEDMM	SMM	RMM	SWLM	RSWLM
%33	0.132	0	0	0	0	0	0	0	0	0
38	0.152	0.1	0.6383427124	0.6453318636	0.7440917938	0.6847962981	0.6777585838	0.6098739114	0.6777585838	0.6098739114
40	0.16	0.2	0.604353691	0.5805631198	0.7055482095	0.6330094145	0.6133591349	0.522628612	0.5633591349	0.4191561178
32	0.128	0.3	0.5491561178	0.5073952484	0.6384836151	0.5569603736	0.5055820954	0.4663837594	0.418209541	0.333875692
25	0.1	0.4	0.5073875692	0.4477722647	0.5948448979	0.5263746846	0.4083615481	0.4059813981	0.2833615481	0.1840993817
11	0.044	0.5	0.3182763082	0.4095415441	0.520231146	0.4061943631	0.3601747558	0.3737263295	0.1901747558	0.1630054195
21	0.084	0.6	0.2826459451	0.2039493165	0.4047525539	0.3391586117	0.2897867312	0.2852892626	0.1497867312	0.1373875692
17	0.068	0.7	0.2068973807	0.1348744295	0.3507605682	0.2573875692	0.2242537185	0.1926366746	0.083383534	0.0890917938
15	0.06	0.8	0.0742537185	0.0926366746	0.133383534	0.0926366746	0.1326366746	0.0808677746	0.0412763082	0.0308541544
11	0.044	0.9	0.0247962981	0.0477585838	0.089873114	0.0377585838	0.0367758588	0.0104644475	0.0012645945	0.0013993165
7	0.028	1	0	0	0	0	0	0	0	0
\end{filecontents*}



\begin{filecontents*}{divergence_gov2.dat}
qnum	qnump	relnum	RM3	RM4	DMM	MEDMM	SMM	RMM	SWLM	RSWLM
%15	0.1006711409	0	0	0	0	0	0	0	0	0
18	0.1208053691	0.1	0.4182215774	0.4452107286	0.4539706588	0.4546751631	0.4076374488	0.3797527764	0.4076374488	0.3797527764
17	0.1140939597	0.2	0.374232556	0.3804419848	0.4054270745	0.3928882795	0.3532379999	0.342507477	0.3732379999	0.2990349828
14	0.0939597315	0.3	0.361440043	0.3172741134	0.3883624801	0.3468392386	0.3254609604	0.3162626244	0.268088406	0.223754557
14	0.0939597315	0.4	0.2572664342	0.2876511297	0.3447237629	0.2862535496	0.2582404131	0.2558602631	0.1832404131	0.1339782467
10	0.067114094	0.5	0.2081551732	0.2494204091	0.310110011	0.1860732281	0.1500536208	0.2536051945	0.0900536208	0.0828842845
11	0.0738255034	0.6	0.1325248101	0.1438281815	0.2546314189	0.1190374767	0.0796655962	0.2351681276	0.0696655962	0.0672664342
12	0.0805369128	0.7	0.1067762457	0.1247532945	0.1506394332	0.0572664342	0.0541325835	0.1825155396	0.043262399	0.0489706588
14	0.0939597315	0.8	0.0293042369	0.0825155396	0.103262399	0.056245611	0.0325155396	0.0607466396	0.0111551732	0.0107330194
10	0.067114094	0.9	0.0013584948	0.0476374488	0.039751979	0.0176374488	0.0166547238	0.0303433125	0.0011434595	0.0012781815
14	0.0939597315	1	0	0	0	0	0	0	0	0
\end{filecontents*}



\begin{filecontents*}{sensitivity.dat}
FbDocNum	SWLM_R	RSWLM_R	SWLM_W	RSWLM_W	SWLM_G	RSWLM_G
0	0.2611	0.2555	0.2058	0.2148	0.314	0.3108
1	0.2641	0.2672	0.207	0.2201	0.3205	0.3267
2	0.2722	0.273	0.2096	0.2296	0.325	0.3299
3	0.2798	0.2755	0.2104	0.2301	0.328	0.3359
4	0.2839	0.2786	0.2165	0.2313	0.3308	0.3387
5	0.2876	0.2829	0.2188	0.2329	0.3357	0.3437
6	0.2889	0.2864	0.2245	0.2364	0.3403	0.3458
7	0.2893	0.2915	0.2298	0.2391	0.3414	0.3477
8	0.2902	0.2943	0.2306	0.243	0.3415	0.3482
9	0.291	0.2945	0.2317	0.2478	0.3422	0.3487
10	0.2918	0.2928	0.2318	0.2497	0.3421	0.3492
20	0.2876	0.2924	0.2437	0.2506	0.3423	0.351
30	0.2856	0.2865	0.2462	0.2501	0.3383	0.3502
40	0.2842	0.2868	0.2317	0.2491	0.3379	0.3493
50	0.2835	0.2855	0.2253	0.2458	0.3351	0.348
60	0.2833	0.2855	0.2201	0.245	0.3333	0.3478
70	0.2821	0.2847	0.2198	0.2451	0.3295	0.3471
80	0.2793	0.2847	0.2173	0.2443	0.3291	0.3462
90	0.2779	0.2847	0.2152	0.2441	0.3263	0.3462
100	0.2755	0.2845	0.2136	0.2441	0.3258	0.3461
200	0.2721	0.2831	0.2102	0.2421	0.3232	0.3438
300	0.2707	0.2826	0.2096	0.2418	0.3217	0.3427
\end{filecontents*}
\usepackage[palatino]{quotchap}
\usepackage{array}
\usepackage{todonotes}
\usetikzlibrary{decorations.text}
\usepackage{multirow}
\usepackage{amsmath}
\usepackage{amssymb}
\usepackage{mathtools}
\usepackage{booktabs}
\usepackage{pgfplots}
\usepackage{graphics}
\usepackage{adjustbox}
\usepackage{csquotes}
\usetikzlibrary{pgfplots.groupplots}
\usepackage{mathrsfs} 
\usepackage{nicefrac}
\pgfplotsset{compat=1.11}
\usepgfplotslibrary{ternary}
\usepackage{tikz}
\usetikzlibrary{calc,spy,shapes}
\usepackage{soul}
\usepackage{afterpage}
\usepackage{bibentry}
\usepackage{dirtytalk}
\usepackage{wrapfig}
\usepackage{enumitem}
\usepackage{colortbl}
\usepackage{listings}
\usepackage{color}
\usepackage{graphicx}
\usepackage{microtype}
\usepackage{multicol}
\usepackage{courier}
\usepackage{xstring} %switch-case
\usetikzlibrary{arrows,decorations.pathmorphing,fit,positioning,patterns}
\usepackage{xspace} % insert space when needed.
\usepackage[square,comma,numbers,sort&compress,sectionbib]{natbib}
\usepackage{algorithm}
\usepackage{algpseudocode}
\usepackage{tkz-graph}
\usepackage{tabularx} % Tables spanning the \linewidth or \textwidth
\usepackage{tikz-qtree}
\usepackage{xargs} 
\usepackage{caption}
\usepackage{subcaption}
\usepackage{microtype} % fixes the line braking to not exceed the line width

% \captionsetup{font=footnotesize}
% \newcommand{\floatfontsize}{\fontsize{10}{8}\selectfovnt}
% \newcommand{\subfloatfontsize}{\fontsize{7}{5}\selectfovnt}
\newcommand{\floatfontsize}{footnotesize}
\newcommand{\subfloatfontsize}{scriptsize}

\captionsetup[figure]{font=\floatfontsize, labelfont=\floatfontsize,textfont=\floatfontsize}

\captionsetup[subfigure]{font=\subfloatfontsize, labelfont=\subfloatfontsize,textfont=\subfloatfontsize}

\captionsetup[table]{font=\floatfontsize, labelfont=\floatfontsize,textfont=\floatfontsize}

\captionsetup[subtable]{font=\subfloatfontsize, labelfont=\subfloatfontsize,textfont=\subfloatfontsize}
 
% \makeatletter
% \DeclareCaptionLabelFormat{numberless}{\ALG@name#1}
% \captionsetup[algorithm]{labelformat=numberless} 
% \makeatother


% \usepackage[subtle]{savetrees}
%saves some space (stealing space very smoothly!):
% \usepackage{setspace}
% \setstretch{0.98}

\newcommand{\algrule}[1][.2pt]{\par\vskip.5\baselineskip\hrule height #1\par\vskip.5\baselineskip}
\makeatother

\makeatletter
\algnewcommand{\LineComment}[1]{\Statex \hskip\ALG@thistlm $\blacktriangleright$ #1}
\makeatother

\usetikzlibrary{arrows.meta}
\usetikzlibrary{positioning,automata}
\usetikzlibrary{calc,spy,shapes}
\usetikzlibrary{arrows,decorations.pathmorphing,fit,positioning,patterns}
\pgfplotsset{compat=newest}
\usepackage{bm}
\DeclareMathOperator*{\expectation}{\mathbb{E}}
\newcommand{\subsup}[3]{{#1}_{\mkern-4mu #2}^{\mkern-4mu #3}}


\renewcommand*\Call[2]{\textproc{#1}(#2)}
\newcommand{\tc}{\cellcolor{lightgray}}
\newcommand{\shrink}{\vspace{-1.5ex}}
\newcommand{\sshrink}{\vspace{-.75ex}}
\renewcommand{\shrink}{}
\renewcommand{\sshrink}{}





\usepackage{framed} 
\usepackage[T1]{fontenc}
\usepackage{libertine}


\renewenvironment{leftbar}[2][\hsize]
{
    \def\FrameCommand
    {
        {\color{#2}\vrule width 3pt}
        \hspace{0pt}
    }
    \MakeFramed{\hsize#1\advance\hsize-\width\FrameRestore}
}
{\endMakeFramed}

% TeX Gyre Pagella Math
\usepackage{mathpazo}

\setlength{\parskip}{5pt}%
\setlength{\parindent}{0pt}%

% \usepackage{setspace}
% \setstretch{1.1}

\newtheorem{mydef}{\textbf{Definition}}%[chapter]

\newcommand{\tracrnet}{TraCRNet\xspace} 

%\usepackage{mathabx}
% instead of \usepackage{mathabx}
% Setup the mathb font (from mathabx.sty)
\DeclareFontFamily{U}{mathb}{\hyphenchar\font45}
\DeclareFontShape{U}{mathb}{m}{n}{
<-6> mathb5 <6-7> mathb6 <7-8> mathb7
<8-9> mathb8 <9-10> mathb9
<10-12> mathb10 <12-> mathb12
}{}
\DeclareSymbolFont{mathb}{U}{mathb}{m}{n}

\DeclareMathSymbol{\smalltriangleup} {2}{mathb}{"98}% name to be checked
\DeclareMathSymbol{\smalltriangledown} {2}{mathb}{"99}% name to be checked
\DeclareMathSymbol{\smalltriangleleft} {2}{mathb}{"9A}% name to be checked
\DeclareMathSymbol{\smalltriangleright}{2}{mathb}{"9B}% name to be checked
\DeclareMathSymbol{\blacktriangleup} {2}{mathb}{"9C}% name to be checked
\DeclareMathSymbol{\blacktriangledown} {2}{mathb}{"9D}% name to be checked
\DeclareMathSymbol{\blacktriangleleft} {2}{mathb}{"9E}% name to be checked
\DeclareMathSymbol{\blacktriangleright}{2}{mathb}{"9F}% name to be checked


\makeatletter
\def\@part[#1]#2{%
    \ifnum \c@secnumdepth >-2\relax
      \refstepcounter{part}%
      \addcontentsline{toc}{part}{\thepart\hspace{1em}#1}%
    \else
      \addcontentsline{toc}{part}{#1}%
    \fi
    \markboth{}{}%
  \reset@font
  \parindent \z@ 
  \vspace*{10\p@}%
  \hbox{%
    \vbox{%
      \hsize=7mm%
      \begin{tabular}{@{}p{7mm}@{}}
        \makebox[7mm]{\scshape\strut\small\partname}\\
        \makebox[7mm]{\cellcolor{gray}\Huge\color{white}\bfseries\strut\thepart\rule[-4cm]{0pt}{4cm}}%
      \end{tabular}%
      \makebox(0,0){\put(-10,-100){\fbox{\phantom{\rule[-4cm]{7mm}{4cm}}}}}
      }%
    \kern-2pt
    \vbox to 0pt{%
       \tabular[t]{@{}p{1cm}p{\dimexpr\hsize-2.1cm}@{}}\hline
        %   & \fontsize{25}{25}\selectfovnt\itshape\bfseries\rule{0pt}{1.5\ht\strutbox}#1\endtabular}%
        %   & \Huge\itshape\rule{0pt}{1.5\ht\strutbox}#1\endtabular}%
         & \Huge\rule{0pt}{1.5\ht\strutbox}#1\endtabular}%
    }%
  \cleardoublepage
%  \vskip 100\p@
}
% \renewcommand*\thepart{\arabic{part}}
\renewcommand\thepart{\Roman{part}}
\makeatother

\makeatletter
\let\size@chapter\LARGE
\makeatother

\newcommand{\mypar}[1]{\medskip\noindent\textit{#1}~}
\newcommand{\fix}{\marginpar{FIX}}
\newcommand{\new}{\marginpar{NEW}}
\newcommand{\rpm}{\raisebox{.2ex}{$\scriptstyle\pm$}}
 
\definecolor{codegreen}{rgb}{0,0.6,0}
\definecolor{codegray}{rgb}{0.5,0.5,0.5}
\definecolor{codepurple}{rgb}{0.58,0,0.82}
\definecolor{backcolour}{rgb}{0.95,0.95,0.92}

\lstdefinestyle{mystyle}{
    backgroundcolor=\color{backcolour},   
    commentstyle=\color{codegreen},
    keywordstyle=\color{magenta},
    numberstyle=\tiny\color{codegray},
    stringstyle=\color{codepurple},
    basicstyle=\footnotesize,
    breakatwhitespace=false,         
    breaklines=true,                 
    captionpos=b,                    
    keepspaces=true,                 
    numbers=left,                    
    numbersep=5pt,                  
    showspaces=false,                
    showstringspaces=false,
    showtabs=false,                  
    tabsize=2
}
 
\lstset{style=mystyle}

\renewcommand{\texttt}[1]{{\fontsize{10}{11}\fontfamily{pcr}\selectfont{#1}}}
% \renewcommand{\url}[1]{{\fontfamily{pcr}\selectfont{\url{#1}}}}
\renewcommand{\todo}[1]{\textcolor{red}{{\bf [TODO: }{\em #1}{\bf ]}}}


% Part one's stuff
\newcommand{\swlm}{significant words language model\xspace}
\newcommand{\Swlm}{Significant words language model\xspace}
\newcommand{\SWLM}{Significant Words Language Model\xspace}
\newcommand{\swlms}{{\swlm}s\xspace}
\newcommand{\Swlms}{{\Swlm}s\xspace}
\newcommand{\SWLMs}{{\SWLM}s\xspace}
\newcommand{\acswlm}{SWLM\xspace}

\newcommand{\rswlm}{regularized significant words language model\xspace}
\newcommand{\Rswlm}{Regularized significant words language model\xspace}
\newcommand{\RSWLM}{Regularized Significant Words Language Model\xspace}
\newcommand{\rswlms}{{\rswlm}s\xspace}
\newcommand{\Rswlms}{{\Rswlm}s\xspace}
\newcommand{\RSWLMs}{{\RSWLM}s\xspace}
\newcommand{\acrswlm}{RSWLM\xspace}

\newcommand{\hswlm}{hierarchical significant words language model\xspace}
\newcommand{\Hswlm}{Hierarchical significant words language model\xspace}
\newcommand{\HSWLM}{Hierarchical Significant Words Language Model\xspace}

\newcommand{\hswlms}{{\hswlm}s\xspace}
\newcommand{\Hswlms}{{\Hswlm}s\xspace}
\newcommand{\HSWLMs}{{\HSWLM}s\xspace}
\newcommand{\achswlm}{HSWLM\xspace}
\newcommand{\ssp}{Strong Separation Principle\xspace}
\newcommand{\acssp}{SSP\xspace}


% Part two's stuff
\newcommand{\Modelone}{Score model\xspace}
\newcommand{\Modeltwo}{Rank model\xspace}
\newcommand{\Modelthree}{Rank\-Prob model\xspace}

\newcommand{\modelone}{\textit{score} model\xspace}
\newcommand{\modeltwo}{\textit{rank} model\xspace}
\newcommand{\modelthree}{\textit{rank\-prob} model\xspace}

\newcommand{\mone}{Score\xspace}
\newcommand{\mtwo}{Rank\xspace}
\newcommand{\mthree}{RankProb\xspace}


\newcommand{\Feedone}{Dense vector representation\xspace}
\newcommand{\Feedtwo}{Sparse vector representation\xspace}
\newcommand{\Feedthree}{Embedding vector representation\xspace}

\newcommand{\feedone}{dense vector representation\xspace}
\newcommand{\feedtwo}{sparse vector representation\xspace}
\newcommand{\feedthree}{embedding vector representation\xspace}

\newcommand{\fone}{Dense\xspace}
\newcommand{\ftwo}{Sparse\xspace}
\newcommand{\fthree}{Embed\xspace}


\newcommand{\pssmall}[1]{\ensuremath{\operatorname{^{\blacktriangleup {#1}}}}}

\newcommand{\tnet}{target network\xspace}
\newcommand{\tnets}{target networks\xspace}
\newcommand{\cnet}{confidence network\xspace}

\newcommand{\cws}{CWS\xspace}

\newcommand{\fwl}{FWL\xspace}
\newcommand{\fwlnospace}{FWL}
\newcommand{\fwlfull}{Fidelity-Weighted Learning\xspace}
\newcommand{\fwlfulllc}{fidelity-weighted learning\xspace}

\newcommand{\std}{student\xspace}
\newcommand{\tch}{teacher\xspace}
\newcommand{\repnet}{representation learner\xspace}
\newcommand{\wa}{weak annotator\xspace}
\newcommand{\was}{weak annotators\xspace}


\newcommand{\ps}{$^\blacktriangleup$}
\newcommand{\ns}{$^\smalltriangledown$}
\newcommand{\fs}{\phantom{$^\blacktriangleup$}}


\DeclareMathOperator*{\argmax}{arg\,max}
\DeclareMathOperator*{\argmin}{arg\,min}

%% this next section (till \makeatother) makes sure that blank pages	
%% are actually completely blank, cause they're not usually	
\makeatletter	
\def\cleardoublepage{\clearpage\if@twoside \ifodd\c@page\else	
	\hbox{}	
	\vspace*{\fill}	
	\thispagestyle{empty}	
	\newpage	
	\if@twocolumn\hbox{}\newpage\fi\fi\fi}	
\makeatother	


%% make the quotation appear next to the chapter number	
% \renewcommand\chapterheadstartvskip{\vspace*{-5\baselineskip}}	

% %% now change the section heading to have a line beneath it	
% %% load up the fancy title-style package	
% \usepackage[calcwidth]{titlesec}	
% \titleformat{\section}[hang]{}	
% {\bf\fontsize{18}{16}\selectfovnt\thesection}{12pt}{\bf\fontsize{18}{16}\selectfovnt}	

% \titleformat{\subsection}[hang]{}	
% {\bf\no\thesubsection}{12pt}{\bf\fontsize{12}{10}\selectfovnt }	

% \titleformat{\subsubsection}[hang]{}	
% {}{12pt}{\bfseries}

