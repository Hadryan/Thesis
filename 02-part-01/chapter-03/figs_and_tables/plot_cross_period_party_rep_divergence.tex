\begin{figure}[!t]
\centering
\resizebox{0.9\linewidth}{!}{%
\begin{tikzpicture}
\pgfkeys{
     /pgf/number format/precision=2, 
    /pgf/number format/fixed zerofill=true,
    /pgf/number format/fixed
}
\begin{axis}[
    width= 12cm, %\textwidth,
    height=5cm, %5cm,
    ybar,%=0pt,
    ymajorgrids,
    minor tick num=1,
    bar width= 6.5pt,
    enlarge y limits=0.25,
    symbolic x coords={
    VVD,
    PvdA,
    CDA,
    PVV,
    SP,
    D66,
    CU,
    GL,
    Opposition,
    Government
},
    x tick label style={rotate=45,anchor=east},
    xtick=data,
    % ymin=0.0, 
    % ymax=0.5,
    ytick = {0.00,0.10,0.20,0.30,0.40,0.50},
    ylabel={},
    %legend style={at={(0.9,0.95),font=\fontsize{5}{6}},
    anchor=north,
    ylabel={JS-Divergence},
%    legend style={at={(0.65,-0.2),font=\fontsize{5}{6}},
    legend style={at={(0.965,0.95),font=\fontsize{5}{6}\selectfont},
    legend columns=1
    },
    nodes near coords,
    every node near coord/.append style={font=\fontsize{6}{7}\selectfont, rotate=90, anchor=west},
	tick label style={font=\fontsize{7}{9}\selectfont},
    label style = {font=\fontsize{8}{10}\selectfont},
    legend style={font=\fontsize{7}{9}\selectfont},
    ]
\definecolor{b}{HTML}{3399FF}
\definecolor{g}{HTML}{5C7C19}
\addplot[fill=b, draw=b] coordinates
{
(VVD,0.3422)
(PvdA,0.3917)
(CDA,0.3889)
(PVV,0.2422)
(SP,0.3050)
(D66,0.2979)
(CU,0.4336)
(GL,0.4471)
(Opposition,0.1562)
(Government,0.2471)
};

\addplot[fill=g, draw=g, pattern color = g, pattern = north west lines] coordinates
{
(VVD,0.1639)
(PvdA,0.1631)
(CDA,0.1666)
(PVV,0.1641)
(SP,0.1918)
(D66,0.1890)
(CU,0.1757)
(GL,0.2759)
(Opposition,0.0512)
(Government,0.0759)
};

\legend{SLM,\acswlm}

\end{axis}
\end{tikzpicture}
}
\caption{
Average of JS-Divergence of standard language models and {\acswlm}s for parliamentary entities in three different periods.\label{fig:cross_period_party_rep_divergence}}
 \end{figure}
