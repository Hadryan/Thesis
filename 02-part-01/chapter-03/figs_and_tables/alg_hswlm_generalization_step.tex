% \begin{figure}[!t]
% \centering
\begin{algorithm}[t!]
% \captionsetup{labelformat=empty}
\begin{algorithmic}[1]
\Procedure{Generalization}{}
\State Stack $\gets$ all entities in breadth first order
\While{Stack is not empty}
  \State $e \gets$ Stack.pop()
  \State $l \gets e$.Height()
  \While{$l > 0$}
    \State $D \gets e$.\Call{getDecedents}{$l$}
    \State \Call{Parsimonize}{$e$,$D$}
    \State $l \gets l-1$  
  \EndWhile
\EndWhile
\EndProcedure
\algrule
 \LineComment{Function \Call{getDecedents}{$l$} gives all the decedents of entity $e$ with $l$ edges distance from it. Function \textsc{Parsimonize}($e$,$B$)  parsimonizes $\theta_e$ given background models in $B$ (Algorithm~\ref{alg:model_parsimonization}).}
\end{algorithmic}
% \captionof{algorithm}{Generalization Stage}
\caption{\label{alg:generalization_stage}Procedure of Generalization.}
\end{algorithm}
% \caption{\label{alg:generalization_stage}Procedure of Generalization. $e$. Function \Call{getDecedents}{$l$} gives all the decedents of entity $e$ with $l$ edges distance from it. Function \textsc{Parsimonize}($e$,$B$)  parsimonizes $\theta_e$ given background models in $B$.}
% \end{figure}