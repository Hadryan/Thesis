\usepackage{xstring}
\newcommand{\resqcontent}[1]{%
    \IfEqCase{#1}{%
    % -------------MAIN I-------------
    {main}{How can we improve the learning process when the data and/or the supervision signal is noisy or limited, in the context of language understanding tasks?}
    % -------------PART I-------------
    {p1}{How to use the structure of the data as prior knowledge to learn robust and effective representations when the data is noisy or highly variable?}
    % -------------chapter 2-------------
    {c2}{How to learn robust representations that are affected by neither undiscerning general, nor noisy accidental features, given the structural relations in the data?}%
    %
    {c2.1}{How to estimate a representation for a set of entities that captures all, and only, the essential shared commonalities of these entities?}
    %
    {c2.2}{How does \swlms captures the mutual notion of relevance for a set of feedback documents and prevent noisy terms by controlling the contribution of each of documents in the feedback model?}%
    %
    {c2.3}{How well can \swlms profile groups of entities and how effective these profiles are in content customization tasks?}%
    % -------------chapter 3-------------
    {c3}{How to learn separable representations for hierarchically structured data that are less sensitive to structural changes and more transferable across time?}%
    %
    {c3.1}{What makes separability of representations a desirable property for classifiers?}%
    %
    {c3.2}{How can we estimate horizontally and vertically separable representations for the hierarchically structured entities?}%
    %
    {c3.3}{How separability of representations can improve their transferability?}%
    % -------------PART II-------------
    {p2}{How to design models that can learn from imperfect examples?}
    % -------------chapter 4-------------
    {c4}{How to train neural networks using pragmatically generated labels, as the weak supervision signal, that will exhibit superior generalization capabilities?}%
    %
    {c4.1}{Can labels from an unsupervised heuristic-based model be used as pragmatically generated weak supervision signal to train an effective neural network?}
    %
    {c4.2}{What setup in terms of input representation and learning objective is most suitable for a neural ranker when training on pragmatically generated labeled data?}
    %
    {c4.3}{How learning from weak supervision signals can help preserving privacy while training neural networks on sensitive data?}
    % -------------chapter 5-------------
    {c5}{How we can best leverage the capacity of information in a given large set of weakly annotated samples and small set of samples with high quality labels to train a neural network?}%
    %
    {c5.1}{When learning from samples of variable quality, can we meta learn an adjustment for the magnitude of the parameter updates in backpropagation based on the merit of labels?}
    %
    {c5.2}{When learning from samples of variable quality, can we reannotate these samples and provide (hopefully) better labels, associated with a fidelity score to regulate the learning rate?}
    % -------------PART III-------------
    {p3}{How can inductive biases help improving the generalization of self-attentive feed-forward sequence models?}
    % -------------chapter 4-------------
    {c6}{How can we combine the recurrent inductive bias of RNNs with the parallelizability and global receptive field of Transformer?}%
    %
    {c6.1}{How do Universal Transformers implement both concurrency and recurrency for sequence modeling?}
    %
    {c6.2}{How effective are Universal Transformers at complex reasoning tasks with limited data, algorithmic tasks that need generalization over observed training samples, and finally real world language understanding tasks?}
    % -------------chapter 6-------------
    }[\PackageError{rq}{Undefined option to rq: #1}{}]%
}%

\newcommand{\resqname}[1]{%
    \IfEqCase{#1}{%
    % -------------MAIN I-------------
    {main}{RQ-Main}
    % -------------PART I-------------
    {p1}{RQ-1}
    % -------------chapter 2-------------
    {c2}{RQ-1.1}
    %
    {c2.1}{RQ-1.1.1}
    %
    {c2.2}{RQ-1.1.2}
    %
    {c2.3}{RQ-1.1.3}
    % -------------chapter 3-------------
    {c3}{RQ-1.2}
    %
    {c3.1}{RQ-1.2.1}
    %
    {c3.2}{RQ-1.2.2}
    %
    {c3.3}{RQ-1.2.3}
    % -------------PART II-------------
    {p2}{RQ-2}
    % -------------chapter 4-------------
    {c4}{RQ-2.1}%
    %
    {c4.1}{RQ-2.1.1}
    %
    {c4.2}{RQ-2.1.2}
    %
    {c4.3}{RQ-2.1.3}
    % -------------chapter 5-------------
    {c5}{RQ-2.2}%
    %
    {c5.1}{RQ-2.2.1}
    %
    {c5.2}{RQ-2.2.2}
    % -------------PART III-------------
    {p3}{RQ-3}
    % -------------chapter 4-------------
    {c6}{RQ-3.1}%
    %
    {c6.1}{RQ-3.1.1}
    %
    {c6.2}{RQ-3.1.2}
    % -------------chapter 6-------------
    }[\PackageError{rq}{Undefined option to rq: #1}{}]%
}%

\newcommand{\resq}[1]{%
    \begin{resqbox}
        \begin{enumerate}
            \item[\textbf{\resqname{#1}}] \emph{\resqcontent{#1}}
        \end{enumerate}
    \end{resqbox}
}%






