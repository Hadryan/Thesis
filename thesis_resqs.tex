\newcommand{\resqcontent}[1]{%
    \IfEqCase{#1}{%
    % -------------MAIN I-------------
    {main}{How can we improve the learning process for language understanding tasks, when the supervision signal, i.e., data and/or labels, is noisy or limited?}
    % -------------PART I-------------
    {p1}{How to use the structure of the data as prior knowledge to learn robust and effective representations of entities and concepts, when the data is noisy or highly variable?}
    % -------------chapter 2-------------
    {c2}{How to learn robust representations for entities and abstract concepts that are affected by neither undiscerning general, nor noisy accidental features, given the structural relations in the data?}%
    %
    {c2.1}{How to estimate a representation for a set of entities that captures all, and only, the essential shared commonalities of these entities?}
    %
    {c2.2}{How do \swlms capture the mutual notion of relevance for a set of feedback documents and prevent noisy terms by controlling the contribution of each of the documents in the feedback model?}%
    %
    {c2.3}{How well can \swlms profile groups of entities and how effective are these profiles in content customization tasks?}%
    % -------------chapter 3-------------
    {c3}{How to learn separable representations for hierarchically structured entities that are less sensitive to structural changes in the data and more transferable across time?}%
    %
    {c3.1}{What makes separability of representations a desirable property for classifiers?}%
    %
    {c3.2}{How can we estimate horizontally and vertically separable representations for hierarchically structured entities?}%
    %
    {c3.3}{How can separability of representations for hierarchical entities improve their transferability?}%
    % -------------PART II-------------
    {p2}{How to design learning algorithms that can learn from weakly annotated samples, while generalizing over the imperfection in their labels?}
    % -------------chapter 4-------------
    {c4}{How can we train neural networks using pragmatically generated pseudo-labels as a weak supervision signal, in a way that they exhibit superior generalization capabilities?}%
    %
    {c4.1}{Can labels from an unsupervised heuristic-based model be used as pragmatically generated weak supervision signal to train an effective neural network?}
    %
    {c4.2}{What setup in terms of input representation and learning objective is most suitable for a neural ranker when training on pragmatically generated labeled data?}
    %
    {c4.3}{How can learning from weak supervision signals help to preserve privacy while training neural networks on sensitive data?}
    % -------------chapter 5-------------
    {c5}{Given a large set of weakly annotated samples and a small set of samples with high-quality labels, how can we best leverage the capacity of information in these sets to train a neural network?}%
    %
    {c5.1}{When learning from samples of variable quality, can we meta learn an adjustment for the magnitude of the parameter updates in backpropagation based on the merit of labels?}
    %
    {c5.2}{When learning from samples of variable quality, can we reannotate these samples and provide (hopefully) better labels, associated with a fidelity score to regulate the learning rate?}
    % -------------PART III-------------
    {p3}{How can inductive biases help to improve the data-efficiency and generalization of learning algorithms?}
    % -------------chapter 6-------------
    {c6}{How can inductive biases improve the generalization of self-attentive feed-forward sequence models?}
    %
    {c6.1}{How do Universal Transformers combine the recurrent inductive bias of RNNs with the parallelizability and global receptive field of Transformer?}
    %
    {c6.2}{How effective are Universal Transformers at complex reasoning tasks with limited data, algorithmic tasks that need generalization over observed training samples, and finally real-world language understanding tasks?}
    }[\PackageError{rq}{Undefined option to rq: #1}{}]%
}%

\newcommand{\resqname}[1]{%
    \IfEqCase{#1}{%
    % -------------MAIN I-------------
    {main}{RQ-Main}
    % -------------PART I-------------
    {p1}{RQ-1}
    % -------------chapter 2-------------
    {c2}{RQ-1.1}
    %
    {c2.1}{RQ-1.1.1}
    %
    {c2.2}{RQ-1.1.2}
    %
    {c2.3}{RQ-1.1.3}
    % -------------chapter 3-------------
    {c3}{RQ-1.2}
    %
    {c3.1}{RQ-1.2.1}
    %
    {c3.2}{RQ-1.2.2}
    %
    {c3.3}{RQ-1.2.3}
    % -------------PART II-------------
    {p2}{RQ-2}
    % -------------chapter 4-------------
    {c4}{RQ-2.1}%
    %
    {c4.1}{RQ-2.1.1}
    %
    {c4.2}{RQ-2.1.2}
    %
    {c4.3}{RQ-2.1.3}
    % -------------chapter 5-------------
    {c5}{RQ-2.2}%
    %
    {c5.1}{RQ-2.2.1}
    %
    {c5.2}{RQ-2.2.2}
    % -------------PART III-------------
    {p3}{RQ-3}
    % -------------chapter 6-------------
    {c6}{RQ-3.1}%
    %
    {c6.1}{RQ-3.1.1}
    %
    {c6.2}{RQ-3.1.2}
    }[\PackageError{rq}{Undefined option to rq: #1}{}]%
}%

\newcommand{\resq}[1]{%
    \begin{resqbox}
        \begin{enumerate}
            \item[\textbf{\resqname{#1}}] \emph{\resqcontent{#1}}
        \end{enumerate}
    \end{resqbox}
}%