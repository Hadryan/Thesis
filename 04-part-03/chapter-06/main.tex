\chapter{\titleof{c6}}
\label{chap:6}
%
\begin{quote}
Inductive biases are great ways for encoding modeling assumptions and improve data efficiency. For many sequence modeling tasks, the recurrent inductive bias plays a crucial role for generalizing beyond the observed data, while self-attentive feed-forward sequence models forego the recurrence toward parallelizability. We can, however, introduce a form of recurrent inductive bias to these models to improve their generalization while keeping the parallelization in the computations. 
\end{quote}
%
\section{Introduction}
Convolutional and fully-intentional feed-forward architectures like the Transformer have recently emerged as viable alternatives to recurrent neural networks (RNNs) for a range of sequence modeling tasks, notably machine translation~\citep{JonasFaceNet2017,transformer}. These parallel-in-time architectures address a significant shortcoming of RNNs, namely their inherently sequential computation which prevents parallelization across elements of the input sequence, whilst still addressing the vanishing gradients problem as the sequence length gets longer~\citep{vanishing-exploding-gradient}.
The Transformer model in particular relies entirely on a self-attention mechanism \citep{decomposableAttnModel,lin2017structured} to compute a series of context-informed vector-space representations of the symbols in its input and output, which are then used to predict distributions over subsequent symbols as the model predicts the output sequence symbol-by-symbol. Not only is this mechanism straightforward to parallelize, but as each symbol's representation is also directly informed by all other symbols' representations, this results in an effectively global receptive field across the whole sequence. This stands in contrast to e.g.\ convolutional architectures which typically only have a limited receptive field.

Notably, however, the Transformer with its fixed stack of distinct layers foregoes RNNs' inductive bias towards learning iterative or recursive transformations. Our experiments indicate that this inductive bias may be crucial for several algorithmic and language understanding tasks of varying complexity: in contrast to models such as the Neural Turing Machine~\citep{ntm14}, the Neural GPU~\citep{neural_gpu} or Stack RNNs~\citep{stack_rnn}, the Transformer does not generalize well to input lengths not encountered during training. 

In this chapter, we address the following research question:
\resq{c6}

\begin{figure}
 \centering
 \includegraphics[width=\textwidth]{04-part-03/chapter-06/figs_and_tables/fig_universal-transformer-as-rnn.pdf}
 \caption{The Universal Transformer repeatedly refines a series of vector representations for each position of the sequence in parallel, by combining information from different positions using self-attention (see Eqn~\ref{MultiheadSelfAttention}) and applying a recurrent transition function (see Eqn~\ref{RecurrentTransition}) across all time steps $1 \leq t \leq T$. We show this process over two recurrent time-steps. Arrows denote dependencies between operations. Initially, $h^0$ is initialized with the embedding for each symbol in the sequence. $h^t_i$ represents the representation for input symbol $1 \leq i \leq m$ at recurrent time-step $t$. With dynamic halting, $T$ is dynamically determined for each position (Section~\ref{sec:dynamic-halting}).}
 \label{fig:rec-state}
\end{figure}


We introduce the \emph{Universal Transformer (UT)}, a parallel-in-time recurrent self-attentive sequence model which can be cast as a generalization of the Transformer model, yielding increased theoretical capabilities and improved results on a wide range of challenging sequence-to-sequence tasks. UTs combine the parallelizability and global receptive field of feed-forward sequence models like the Transformer with the recurrent inductive bias of RNNs, which seems to be better suited to a range of algorithmic and natural language understanding sequence-to-sequence problems. As the name implies, and in contrast to the standard Transformer, under certain assumptions UTs can be shown to be Turing-complete (or ``computationally universal'', as shown in Section~\ref{sec:related}).

In each recurrent step, the Universal Transformer iteratively refines its representations for all symbols in the sequence in parallel using a self-attention mechanism~\citep{decomposableAttnModel,lin2017structured}, followed by a transformation (shared across all positions and time-steps) consisting of a depth-wise separable convolution \citep{xception2016,slicenet} or a position-wise fully-connected layer (see Fig~\ref{fig:rec-state}). We also add a dynamic per-position halting mechanism \citep{graves2016adaptive}, allowing the model to choose the required number of refinement steps \emph{for each symbol} dynamically, and show for the first time that such a conditional computation mechanism can in fact improve accuracy on several smaller, structured algorithmic and linguistic inference tasks (although it marginally degraded results on MT). 

Our strong experimental results show that UTs outperform Transformers and LSTMs across a wide range of tasks. The added recurrence yields improved results in machine translation where UTs outperform the standard Transformer. In experiments on several algorithmic tasks and the bAbI language understanding task, UTs also consistently and significantly improve over LSTMs and the standard Transformer. Furthermore, on the challenging LAMBADA text understanding data set UTs with dynamic halting achieve a new state of the art.

\subsection{Detailed Research Questions}
We break down our main research question in this chapter into two concrete research questions:
\begin{resqbox}
\begin{enumerate}
\item[\textbf{\resqname{c6.1}}] \emph{\resqcontent{c6.1}}
\item[\textbf{\resqname{c6.2}}] \emph{\resqcontent{c6.2}}
\end{enumerate}
\end{resqbox}

In the following sections, we will address these research questions.

\section{The Universal Transformer }%: A Self-attentive Concurrent-Recurrent Sequence Model}
Here, in this section, we focus on the following research question:
\resq{c6.1}

The Universal Transformer (UT; see Fig.~\ref{fig:universal-transformer-complete}) is based on the popular encoder-decoder architecture commonly used in most neural sequence-to-sequence models \citep{sutskever14,cho2014learning,transformer}. Both the encoder and decoder of the UT operate by applying a recurrent neural network to the representations of each of the positions of the input and output sequence, respectively. However, in contrast to most applications of recurrent neural networks to sequential data, the UT does not recur over positions in the sequence, but over consecutive revisions of the vector representations of each position (i.e., over ``depth''). In other words, the UT is not computationally bound by the number of symbols in the sequence, but only by the number of revisions made to each symbol's representation.

In each recurrent time-step, the representation of every position is concurrently (in parallel) revised in two sub-steps: first, using a self-attention mechanism to exchange information across all positions in the sequence, thereby generating a vector representation for each position that is informed by the representations of all other positions at the previous time-step. Then, by applying a transition function (shared across position and time) to the outputs of the self-attention mechanism, independently at each position. As the recurrent transition function can be applied any number of times, this implies that UTs can have variable depth (number of per-symbol processing steps). Crucially, this is in contrast to most popular neural sequence models, including the Transformer~\citep{transformer} or deep RNNs, which have constant depth as a result of applying a \emph{fixed stack} of layers. We now describe the encoder and decoder in more detail.

\begin{figure}[t]
 \centering
 \includegraphics[width=0.9\textwidth]{04-part-03/chapter-06/figs_and_tables/fig_universal-transformer-complete.pdf}
 \caption{The recurrent blocks of the Universal Transformer encoder and decoder.}
 \label{fig:universal-transformer-complete}
\end{figure}

\textbf{\textsc{Encoder:}} Given an input sequence of length $m$, we start with a matrix whose rows are initialized as the $d$-dimensional embeddings of the symbols at each position of the sequence $H^0 \in \mathbb{R}^{m \times d}$. The UT then iteratively computes representations $H^t$ at step $t$ for all $m$ positions in parallel by applying the multi-headed dot-product self-attention mechanism from \cite{transformer}, followed by a recurrent transition function. We also add residual connections around each of these function blocks and apply dropout and layer normalization \citep{srivastava2014dropout, layernorm2016} (See Figure~\ref{fig:universal-transformer-complete} for the schema of the complete model.).

More specifically, we use the scaled dot-product attention which combines queries $Q$, keys $K$ and values $V$ as follows

\begin{equation}
   \textsc{Attention}(Q, K, V) = \textsc{softmax} \left( \frac{QK^T}{\sqrt{d}} \right) V,
\end{equation}

where $d$ is the number of columns of $Q$, $K$ and $V$. We use the multi-head version with $k$ heads, as introduced in \citep{transformer},

\begin{align}
    \label{MultiheadSelfAttention}
    \textsc{MultiHeadSelfAttention}(H^t) &= \textsc{Concat}(\mathrm{head_1}, ..., \mathrm{head_k})W^O\\
    \text{where}~\mathrm{head_i} &= \textsc{Attention}(H^t W^Q_i, H^t W^K_i, H^t W^V_i)
\end{align}

and we map the state $H^t$ to queries, keys and values with affine projections using learned parameter matrices $W^Q \in \mathbb{R}^{d \times d/k}$, $W^K \in \mathbb{R}^{d \times d/k}$, $W^V \in \mathbb{R}^{d \times d/k}$ and $W^O \in \mathbb{R}^{d \times d}$.

At step $t$, the UT then computes revised representations $H^t \in \mathbb{R}^{m \times d}$ for all $m$ input positions as follows

\begin{align}
    \label{RecurrentTransition}
    H^t &= \textsc{LayerNorm}(A^{t} + \textsc{Transition}(A^t) ) \\
    \mathrm{where}~A^t &= \textsc{LayerNorm}((H^{t-1} + P^{t}) + \textsc{MultiHeadSelfAttention}(H^{t-1} + P^{t})),
\end{align}
where \textsc{LayerNorm()} is defined in \cite{layernorm2016}, and \textsc{Transition()} and $P^t$ are discussed below.

Depending on the task, we use one of two different transition functions: either a separable convolution~\citep{xception2016} or a fully-connected neural network that consists of a single rectified-linear activation function between two affine transformations, applied position-wise, i.e. individually to each row of $A^t$.

$P^t \in \mathbb{R}^{m \times d}$ above are fixed, constant, two-dimensional (position, time) \emph{coordinate embeddings}, obtained by computing the sinusoidal position embedding vectors as defined in \citep{transformer} for the positions $1 \leq i \leq m$ and the time-step $1 \leq t \leq T$ separately for each vector-dimension $1 \leq j \leq d$, and summing:
\begin{align}
\label{eqn:coordinate-embeddings}
    P^t_{i, 2j} &= \sin(i / 10000^{2j / d}) + \sin(t / 10000^{2j / d}) \\
    P^t_{i, 2j+1} &= \cos(i / 10000^{2j / d}) + \cos(t / 10000^{2j / d}).
\end{align}

After $T$ steps (each updating all positions of the input sequence in parallel), the final output of the Universal Transformer encoder is a matrix of $d$-dimensional vector representations $H^T \in \mathbb{R}^{m \times d}$ for the $m$ symbols of the input sequence.

\textbf{\textsc{Decoder:}} The decoder shares the same basic recurrent structure of the encoder. However, after the self-attention function, the decoder additionally also attends to the final encoder representation $H^T$ of each position in the input sequence using the same multihead dot-product attention function from Equation \ref{MultiheadSelfAttention}, but with queries $Q$ obtained from projecting the decoder representations, and keys and values ($K$ and $V$) obtained from projecting the encoder representations (this process is akin to standard attention \citep{bahdanau2014neural}).

Like the Transformer model, the UT is autoregressive \citep{graves2013generating}. Trained using teacher-forcing, at generation time it produces its output one symbol at a time, with the decoder consuming the previously produced output positions. During training, the decoder input is the target output, shifted to the right by one position.
The decoder self-attention distributions are further masked so that the model can only attend to positions to the left of any predicted symbol. Finally, the per-symbol target distributions are obtained by applying an affine transformation $O \in \mathbb{R}^{d \times V}$ from the final decoder state to the output vocabulary size $V$, followed by a softmax which yields an $(m \times V)$-dimensional output matrix normalized over its rows:

\begin{equation}
 p\left(y_{pos} | y_{[1:pos - 1]}, H^T\right) = \textsc{softmax}(OH^T)\footnote{Note that $T$ here denotes time-step $T$ and not the transpose operation.}
\end{equation}

To generate from the model, the encoder is run once for the conditioning input sequence. Then the decoder is run repeatedly, consuming all already-generated symbols, while generating one additional distribution over the vocabulary for the symbol at the next output position per iteration. We then typically sample or select the highest probability symbol as the next symbol.


\subsection{Adaptive Computation by Dynamic Halting}
\label{sec:dynamic-halting}
\begin{figure}
 \centering
 \includegraphics[width=1.0\textwidth]{04-part-03/chapter-06/figs_and_tables/fig_adaptive-universal-transformer.pdf}
 \caption{An unrolled visualization of Universal Transformer with dynamic halting. It illustrates different numbers of recurrent revisions per position (best viewed in colour).}
 \label{fig:adaptive_ut}
\end{figure}
In sequence processing systems, certain symbols (e.g.\ some words or phonemes) are usually more ambiguous than others. It is therefore reasonable to allocate more processing resources to these more ambiguous symbols. Adaptive Computation Time (ACT) \citep{graves2016adaptive} is a mechanism for dynamically modulating the number of computational steps needed to process each input symbol (called the ``ponder time'') in standard recurrent neural networks based on a scalar \emph{halting probability} predicted by the model at each step.

Inspired by the interpretation of Universal Transformers as applying self-attentive RNNs in parallel to all positions in the sequence, we also add a dynamic ACT halting mechanism to each position, i.e.\ to each per-symbol self-attentive RNN. Once the per-symbol recurrent block halts, its state is simply copied to the next step until all blocks halt, or we reach a maximum number of steps. The final output of the encoder is then the final layer of representations produced in this way. Figure~\ref{fig:adaptive_ut} illustrates a Universal Transformer encoder with $T$, number of revisions, dynamically determined for each position.


We implement the dynamic halting based on ACT~\citep{graves2016adaptive} as follows in TensorFlow. In each step of the UT with dynamic halting, we are given the halting probabilities, remainders, number of updates up to that point, and the previous state (all initialized as zeros), as well as a scalar threshold between 0 and 1 (a hyper-parameter). We then compute the new state for each position and calculate the new per-position halting probabilities based on the state for each position~\footnote{The current implementation of adaptive computation time does not allow for a fully ``end-to-end backpropable'' gradient of the proposed training loss. However, the discontinuity of the cost function might not imply that meaningful learning is not possible and in fact, the experiments in the original paper~\citep{graves2016adaptive} as well as here in Universal Transformer with adaptive halting suggest it works fine.}. The UT then decides to halt for some positions that crossed the threshold, and updates the state of other positions until the model halts for all positions or reaches a predefined maximum number of steps:
\begin{lstlisting}[language=Python, caption=UT with dynamic halting.]
# While-loop stops when this predicate is FALSE
# i.e., all ((probability < threshold) & (counter < max_steps)) are false
def should_continue(u0, u1, halting_probability, u2, n_updates, u3):
return tf.reduce_any(
            tf.logical_and(
                tf.less(halting_probability, threshold),
                tf.less(n_updates, max_steps)))
# Do while loop iterations until predicate above is false
(_, _, _, remainder, n_updates, new_state) = tf.while_loop(
    should_continue, ut_with_dynamic_halting, (state, 
    step, halting_probability, remainders, n_updates, previous_state))
\end{lstlisting}


% \begin{lstlisting}[language=Python, caption=]
%  # initializing halting probabilities
% halting_probability = tf.zeros(
%   (
%       batch_size,
%       length,
%   ), name="halting_probability")
% # initializing remainders 
% remainders = tf.zeros(
%   (
%       batch_size,
%       length,
%   ), name="remainder")
% # initializing number of updates performed 
% n_updates = tf.zeros(
%   (
%       batch_size,
%       length,
%   ), name="n_updates")

% # initializing Previous cell states
% previous_state = tf.zeros_like(state, name="previous_state")
% step = tf.constant(0, dtype=tf.int32)

% # While loop stops when this predicate is FALSE.
% # Ie all (probability < 1-eps AND counter < N) are false.
% def should_continue(u0, u1, halting_probability, u2, n_updates, u3):
% return tf.reduce_any(
%     tf.logical_and(
%         tf.less(halting_probability, threshold),
%         tf.less(n_updates, act_max_steps)))

% # Do while loop iterations until predicate above is false.
% (_, _, _, remainder, n_updates, new_state) = tf.while_loop(
%   should_continue, ut_with_dynamic_halting,
%   (state, step, halting_probability, remainders, n_updates, previous_state))
% \end{lstlisting}


The following shows the computations in each step:

\begin{lstlisting}[language=Python, caption=Computations in each step of the UT with dynamic halting.]
def ut_with_dynamic_halting(state, step, halting_probability, 
                            remainders, n_updates, previous_state):
    # Calculate the probabilities based on the state 
    p = common_layers.dense(state, 1, activation=tf.nn.sigmoid, 
        use_bias=True)
    # Mask for inputs which have not halted yet
    still_running = tf.cast(
        tf.less(halting_probability,1.0), tf.float32)
    # Mask of inputs which halted at this step
    new_halted = tf.cast(
        tf.greater(halting_probability + p * still_running, threshold), 
            tf.float32) * still_running
    # Mask of inputs which haven't halted, and didn't halt this step
    still_running = tf.cast(
        tf.less_equal(halting_probability + p * still_running, 
            threshold), tf.float32) * still_running
    # Add the halting probability for this step to the halting
    # probabilities for those inputs which haven't halted yet
    halting_probability += p * still_running
    # Compute remainders for the inputs which halted at this step
    remainders += new_halted * (1 - halting_probability)
    # Add the remainders to those inputs which halted at this step
    halting_probability += new_halted * remainders
    # Increment n_updates for all inputs which are still running
    n_updates += still_running + new_halted
    # Compute the weight to be applied to the new state and output:
    #   0 when the input has already halted,
    #   p when the input hasn't halted yet,
    #   the remainders when it halted this step.
    update_weights = tf.expand_dims(p * still_running +
                                    new_halted * remainders, -1)
    # Apply transformation to the state
    transformed_state = transition_function(self_attention(state))
    # Interpolate transformed and previous states for non-halted inputs
    new_state = ((transformed_state * update_weights) +
                 (previous_state * (1 - update_weights)))
    step += 1
    return (transformed_state, step, halting_probability,
            remainders, n_updates, new_state)
\end{lstlisting}


\section{Universality and Relation to other Models}
\label{sec:related}
When running for a fixed number of steps, the Universal Transformer is equivalent to a multi-layer Transformer with tied parameters across all its layers. This is partly similar to the Recursive Transformer, which ties the weights of its self-attention layers across depth~\citep{gulcehre2018hyperbolic}\footnote{Note that in UT both the self-attention and transition weights are tied across layers.}. However, as the per-symbol recurrent transition functions can be applied any number of times, another and possibly more informative way of characterizing the UT is as a block of parallel RNNs (one for each symbol, with shared parameters) evolving per-symbol hidden states concurrently, generated at each step by attending to the sequence of hidden states at the previous step. In this way, it is related to architectures such as the Neural GPU \citep{neural_gpu} and the Neural Turing Machine \citep{ntm14}. UTs thereby retain the attractive computational efficiency of the original feed-forward Transformer model, but with the added recurrent inductive bias of RNNs. Furthermore, using a dynamic halting mechanism, UTs can choose the number of processing steps based on the input data. %interpolate between the feed-forward, fixed-depth Transformer and a gated, recurrent architecture running for a number of steps dependent on the input data. 

The connection between the Universal Transformer and other sequence models is apparent from the architecture: if we limited the recurrent steps to one, it would be a Transformer. But it is more interesting to consider the relationship between the Universal Transformer and RNNs and other networks where recurrence happens over the time dimension. Superficially these models may seem closely related since they are recurrent as well. But there is a crucial difference: time-recurrent models like RNNs cannot access memory in the recurrent steps. This makes them computationally more similar to automata, since the only memory available in the recurrent part is a fixed-size state vector. UTs, on the other hand, can attend to the whole previous layer, allowing it to access memory in the recurrent step. 

Given sufficient memory the Universal Transformer is computationally universal -- i.e.\ it belongs to the class of models that can be used to simulate any Turing machine, thereby addressing a shortcoming of the standard Transformer model. In addition to being theoretically appealing, our results show that this added expressivity also leads to improved accuracy on several challenging sequence modeling tasks. This closes the gap between practical sequence models competitive on large-scale tasks such as machine translation, and computationally universal models such as the Neural Turing Machine or the Neural GPU \citep{ntm14,neural_gpu}, which can be trained using gradient descent to perform algorithmic tasks.

To show this, we can reduce a Neural GPU to a Universal Transformer. Ignoring the decoder and parameterizing the self-attention module, i.e. self-attention with the residual connection, to be the identity function, we assume the transition function to be a convolution. If we now set the total number of recurrent steps $T$ to be equal to the input length, we obtain exactly a Neural GPU. Note that the last step is where the Universal Transformer crucially differs from the vanilla Transformer whose depth cannot scale dynamically with the size of the input. A similar relationship exists between the Universal Transformer and the Neural Turing Machine, whose single read/write operations per step can be expressed by the global, parallel representation revisions of the Universal Transformer. In contrast to these models, however, which only perform well on algorithmic tasks, the Universal Transformer also achieves competitive results on realistic natural language tasks such as LAMBADA and machine translation.

Another related model architecture is that of end-to-end Memory Networks \citep{sukhbaatar2015}. In contrast to end-to-end memory networks, however, the Universal Transformer uses memory corresponding to states aligned to individual positions of its inputs or outputs. Furthermore, the Universal Transformer follows the encoder-decoder configuration and achieves competitive performance in large-scale sequence-to-sequence tasks.


\subsection{On the Computational Power of UT vs Transformer}
\label{app:univerrality_example}

With respect to their computational power, the key difference between the Transformer and the Universal Transformer lies in the number of sequential steps of computation (i.e.\ in depth). While a standard Transformer executes a total number of operations that scales with the input size, the number of sequential operations is constant, independent of the input size and determined solely by the number of layers. Assuming finite precision, this property implies that the standard Transformer cannot be computationally universal. When choosing a number of steps as a function of the input length, however, the Universal Transformer does not suffer from this limitation. Note that this holds independently of whether or not adaptive computation time is employed but does assume a non-constant, even if possibly deterministic, number of steps. Varying the number of steps dynamically after training is enabled by sharing weights across sequential computation steps in the Universal Transformer.

\begin{figure}[t]
\centering
\includegraphics[width=0.6\textwidth, trim={0.1cm 0.5cm 0.1cm 1.2cm}, clip]{04-part-03/chapter-06/figs_and_tables/fig_universality_example.png}
\end{figure}

An intuitive example are functions whose execution requires the sequential processing of each input element. In this case, for any given choice of depth $T$, one can construct an input sequence of length $N>T$ that cannot be processed correctly by a standard Transformer. With an appropriate, input-length dependent choice of sequential steps, however, a Universal Transformer, RNNs or Neural GPUs can execute such a function.
\section{Experiments and Analysis}
We evaluated the Universal Transformer on a range of algorithmic and language understanding tasks, as well as on machine translation. We describe these tasks and datasets in more detail in Appendix~\ref{app:desc-tasks}. All code and datasets reproducing these results will be released as open source.

\subsection{bAbI Question-Answering}
The bAbi question answering dataset~\citep{weston2015towards} consists of 20 different synthetic tasks\footnote{\url{https://research.fb.com/downloads/babi}}. The aim is that each task tests a unique aspect of language understanding and reasoning, including the ability of: reasoning from supporting facts in a story, answering true/false type questions, counting, understanding negation and indefinite knowledge, understanding coreferences, time reasoning, positional and size reasoning, path-finding, and understanding motivations (to see examples for each of these tasks, please refer to Table 1 in \citep{weston2015towards}).

There are two versions of the dataset, one with 1k training examples and the other with 10k examples. It is important for a model to be data-efficient to achieve good results using only the 1k training examples. Moreover, the original idea is that a single model should be evaluated across all the tasks (not tuning per task), which is the \emph{train joint} setup in Tables~\ref{tab:babi-results} and ~\ref{tbl:babi_details}.

A standard Transformer does not achieve good results on this task\footnote{We experimented with different hyper-parameters and different network sizes, but it always overfits.}. However, we have designed a model based on the Universal Transformer which achieves state-of-the-art results on this task.

To encode the input, similar to~\cite{henaff2016tracking}, we first encode each fact in the story by applying a learned multiplicative positional mask to each word's embedding, and summing up all embeddings.
We embed the question in the same way, and then feed the (Universal) Transformer with these embeddings of the facts and questions. 

As originally proposed, models can either be trained on each task separately (``train single'') or jointly on all tasks (``train joint''). Table~\ref{tab:babi-results} summarizes our results. We conducted 10 runs with different initializations and picked the best model based on performance on the validation set, similar to previous work. Both the UT and UT with dynamic halting achieve state-of-the-art results on all tasks in terms of average error and number of failed tasks\footnote{Defined as $> 5\%$ error.}, in both the 10K and 1K training regime. Tables~\ref{tbl:babi_details} presents the results of best and average results of 10 runs breakdown by task.

\begin{table}[t!]
\centering
\begin{adjustbox}{max width=\textwidth}
\begin{tabular}{lllll}
& & & & \\ \toprule
\multirow{2}{*}{ \bf Model } & \multicolumn{2}{c}{ \bf 10K examples } & \multicolumn{2}{c}{ \bf 1K examples } \\ \cmidrule{2-5}
& train single & train joint & train single & train joint \\ \midrule
\multicolumn{5}{c}{\bf Previous best results:} \\ \midrule
QRNet~\citep{seo2016query} & 0.3 (0/20) & - & - & - \\
Sparse DNC~\citep{rae2016scaling} & - & 2.9 (1/20) & - & - \\
GA+MAGE~\cite{dhingra2017linguistic} & - & - & 8.7 (5/20) & - \\
MemN2N~\cite{sukhbaatar2015} & - & - & -  & 12.4 (11/20) \\\midrule
\multicolumn{5}{c}{\bf Our Results:} \\ \midrule
Transformer~\citep{transformer} & 15.2 (10/20) & 22.1 (12/20) & 21.8 (5/20) & 26.8 (14/20) \\
Universal Transformer (this work) & 0.23 (0/20) & 0.47 (0/20) & 5.31 (5/20) & 8.50 (8/20) \\
UT w/ dynamic halting (this work) & {\bf 0.21 (0/20)} & {\bf 0.29 (0/20)} & {\bf 4.55 (3/20)} & {\bf 7.78 (5/20)} \\ \bottomrule
\end{tabular}
\end{adjustbox}
\caption{Average error and number of failed tasks ($> 5\%$ error) out of 20 (in parentheses; lower is better in both cases) on the bAbI dataset under the different training/evaluation setups. We indicate state-of-the-art where available for each, or `-' otherwise.}
\label{tab:babi-results}
\end{table}
\begin{table}[t!]
\centering
\caption{Detailed results on the bAbI question answering tasks.}
\label{tbl:babi_details}
\begin{subtable}{0.6\textwidth}
\centering
\begin{adjustbox}{max width=\textwidth}
\begin{tabular}{lcccc}
\toprule
\multicolumn{5}{c}{Best seed run for each task (out of 10 runs) } \\ \midrule
\multirow{2}{*}{ Task id } & \multicolumn{2}{c}{ 10K } & \multicolumn{2}{c}{ 1K } \\  \cmidrule{2-5}
& train single & train joint & train single & train joint \\ \midrule
1 & 0.0 & 0.0 & 0.0 & 0.0 \\
2 & 0.0 & 0.0 & 0.0 & 0.5 \\
3 & 0.4 & 1.2 & 3.7 & 5.4 \\
4 & 0.0 & 0.0 & 0.0 & 0.0 \\
5 & 0.0 & 0.0 & 0.0 & 0.5 \\
6 & 0.0 & 0.0 & 0.0 & 0.5 \\
7 & 0.0 & 0.0 & 0.0 & 3.2 \\
8 & 0.0 & 0.0 & 0.0 & 1.6 \\
9 & 0.0 & 0.0 & 0.0 & 0.2 \\
10 & 0.0 & 0.0 & 0.0 & 0.4 \\
11 & 0.0 & 0.0 & 0.0 & 0.1 \\
12 & 0.0 & 0.0 & 0.0 & 0.0 \\
13 & 0.0 & 0.0 & 0.0 & 0.6 \\
14 & 0.0 & 0.0 & 0.0 & 3.8 \\
15 & 0.0 & 0.0 & 0.0 & 5.9 \\
16 & 0.4 & 1.2 & 5.8 & 15.4 \\
17 & 0.6 & 0.2 & 32.0 & 42.9 \\
18 & 0.0 & 0.0 & 0.0 & 4.1 \\
19 & 2.8 & 3.1 & 47.1 & 68.2 \\
20 & 0.0 & 0.0 & 2.4 & 2.4 \\ \midrule
avg err & 0.21 & 0.29 & 4.55 & 7.78 \\ \midrule
failed & 0 & 0 & 3 & 5 \\
\bottomrule
\end{tabular}
\end{adjustbox}
\end{subtable}
\\
\vspace{10pt}
\begin{subtable}{0.6\textwidth}
\centering
\begin{adjustbox}{max width=\textwidth}
\begin{tabular}{lcccc}
\toprule
\multicolumn{5}{c}{Average (\rpm var) over all seeds (for 10 runs)} \\ \midrule
\multirow{2}{*}{ Task id } & \multicolumn{2}{c}{ 10K } & \multicolumn{2}{c}{ 1K } \\  \cmidrule{2-5}
& train single & train joint & train single & train joint \\ \midrule
1 & 0.0 \rpm 0.0 & 0.0 \rpm 0.0 & 0.2 \rpm 0.3 & 0.1 \rpm 0.2 \\ 
2 & 0.2 \rpm 0.4 & 1.7 \rpm 2.6 & 3.2 \rpm 4.1 & 4.3 \rpm 11.6 \\ 
3 & 1.8 \rpm 1.8 & 4.6 \rpm 7.3 & 9.1 \rpm 12.7 & 14.3 \rpm 18.1 \\ 
4 & 0.1 \rpm 0.1 & 0.2 \rpm 0.1 & 0.3 \rpm 0.3 & 0.4 \rpm 0.6 \\ 
5 & 0.2 \rpm 0.3 & 0.8 \rpm 0.5 & 1.1 \rpm 1.3 & 4.3 \rpm 5.6 \\ 
6 & 0.1 \rpm 0.2 & 0.1 \rpm 0.2 & 1.2 \rpm 2.1 & 0.8 \rpm 0.4 \\ 
7 & 0.3 \rpm 0.5 & 1.1 \rpm 1.5 & 0.0 \rpm 0.0 & 4.1 \rpm 2.9 \\ 
8 & 0.3 \rpm 0.2 & 0.5 \rpm 1.1 & 0.1 \rpm 0.2 & 3.9 \rpm 4.2 \\ 
9 & 0.0 \rpm 0.0 & 0.0 \rpm 0.0 & 0.1 \rpm 0.1 & 0.3 \rpm 0.3 \\ 
10 & 0.1 \rpm 0.2 & 0.5 \rpm 0.4 & 0.7 \rpm 0.8 & 1.3 \rpm 1.6 \\ 
11 & 0.0 \rpm 0.0 & 0.1 \rpm 0.1 & 0.4 \rpm 0.8 & 0.3 \rpm 0.9 \\ 
12 & 0.2 \rpm 0.1 & 0.4 \rpm 0.4 & 0.6 \rpm 0.9 & 0.3 \rpm 0.4 \\ 
13 & 0.2 \rpm 0.5 & 0.3 \rpm 0.4 & 0.8 \rpm 0.9 & 1.1 \rpm 0.9 \\ 
14 & 1.8 \rpm 2.6 & 1.3 \rpm 1.6 & 0.1 \rpm 0.2 & 4.7 \rpm 5.2 \\ 
15 & 2.1 \rpm 3.4 & 1.6 \rpm 2.8 & 0.3 \rpm 0.5 & 10.3 \rpm 8.6 \\ 
16 & 1.9 \rpm 2.2 & 0.9 \rpm 1.3 & 9.1 \rpm 8.1 & 34.1 \rpm 22.8 \\ 
17 & 1.6 \rpm 0.8 & 1.4 \rpm 3.4 & 43.7 \rpm 18.6 & 51.1 \rpm 12.9 \\ 
18 & 0.3 \rpm 0.4 & 0.7 \rpm 1.4 & 2.3 \rpm 3.6 & 12.8 \rpm 9.0 \\ 
19 & 3.4 \rpm 4.0 & 6.1 \rpm 7.3 & 50.2 \rpm 8.4 & 73.1 \rpm 23.9 \\ 
20 & 0.0 \rpm 0.0 & 0.0 \rpm 0.0 & 3.2 \rpm 2.5 & 2.6 \rpm 2.8 \\ \midrule
avg & 0.73 \rpm 0.89 & 1.12 \rpm 1.62 & 6.34 \rpm 3.32 & 11.21 \rpm 6.65 \\ 
\bottomrule
\end{tabular}
\end{adjustbox}
\end{subtable}
\end{table}
\afterpage{\clearpage}


To understand the working of the model better, we analyzed both the attention distributions and the average ACT ponder times for this task. First, we observe that the attention distributions start out very uniform, but get progressively sharper in later steps around the correct supporting facts that are required to answer each question, which is indeed very similar to how humans would solve the task. 
%
Second, with dynamic halting we observe that the average ponder time (i.e.\ depth of the per-symbol recurrent processing chain) over all positions in all samples in the test data for tasks requiring three supporting facts is higher ($3.8 \rpm 2.2$) than for tasks requiring only two ($3.1 \rpm 1.1$), which is in turn higher than for tasks requiring only one supporting fact ($2.3 \rpm 0.8$). This indicates that the model adjusts the number of processing steps with the number of supporting facts required to answer the questions. 

\begin{figure}[t]
 \centering
 \includegraphics[width=\textwidth]{04-part-03/chapter-06/figs_and_tables/fig_task3_example_ponder.png}
 \caption{Ponder time of UT with dynamic halting for encoding facts in a story and question in a bAbI task requiring three supporting facts.}
 \label{fig:act_ponder}
\end{figure}
Finally, we observe that the histogram of ponder times at different positions is more uniform in tasks requiring only one supporting fact compared to two and three, and likewise for tasks requiring two compared to three.  Especially for tasks requiring three supporting facts, many positions halt at step 1 or 2 already and only a few get transformed for more steps (see for example Figure~\ref{fig:act_ponder}). This is particularly interesting as the length of stories is indeed much higher in this setting, with more irrelevant facts which the model seems to successfully learn to ignore in this way.

Similar to dynamic memory networks~\citep{kumar2016ask}, there is an iterative attention process in UTs that allows the model to condition its attention over memory on the result of previous iterations. 
%
Figures~\ref{fig:ex1}, \ref{fig:ex2}, \ref{fig:ex3}, and \ref{fig:ex4} present visualizations of the attention distributions on bAbI tasks for some examples from Task 1, 2, and 3. The visualization of attention weights is over different time steps based on different heads over all the facts in the story and a question. Different color bars on the left side indicate attention weights based on different heads (4 heads in total).

\begin{figure}[!h]
\begin{minipage}{\textwidth}
\fontsize{8}{8}\fontfamily{pcr}\selectfont
\begin{tabular}{l l}
\textbf{An example from tasks 1}: & \textbf{(requiring one supportive fact to solve)}\\
\\
\textbf{Story}: & \\
& John travelled to the hallway. \\
& Mary journeyed to the bathroom. \\
& Daniel went back to the bathroom. \\
& John moved to the bedroom \\
\\
\textbf{Question}: & \\
& Where is Mary? \\
\textbf{Model's output}: & \\
& bathroom
\end{tabular}
\end{minipage}
\\ \vfill
\vspace{20pt} % maximize space between the minipages
\begin{minipage}{\textwidth}
    \centering
    \begin{subfigure}[t]{\textwidth}
        \centering
        \includegraphics[height=0.8in]{04-part-03/chapter-06/figs_and_tables/figs_attention_babi/e1-step1.png}
        \caption{Step 1}
    \end{subfigure}%
    \hfill \hfill
    \begin{subfigure}[t]{\textwidth}
        \centering
        \includegraphics[height=0.8in]{04-part-03/chapter-06/figs_and_tables/figs_attention_babi/e1-step2}
        \caption{Step 2}
    \end{subfigure}
    \hfill \hfill
    \begin{subfigure}[t]{\textwidth}
        \centering
        \includegraphics[height=0.8in]{04-part-03/chapter-06/figs_and_tables/figs_attention_babi/e1-step3}
        \caption{Step 3}
    \end{subfigure}
    \hfill \hfill 
    \begin{subfigure}[t]{\textwidth}
        \centering
        \includegraphics[height=0.8in]{04-part-03/chapter-06/figs_and_tables/figs_attention_babi/e1-step4}
        \caption{Step 4}
    \end{subfigure}
\end{minipage}
    \caption{\label{fig:ex1}Visualization of the attention distributions, when encoding the question: \emph{``Where is Mary?''}.}
\end{figure}
\afterpage{\clearpage}




\begin{figure}[!h]
\begin{minipage}{\textwidth}
\fontsize{8}{8}\fontfamily{pcr}\selectfont
\begin{tabular}{l l}
\textbf{An example from tasks 2}: & \textbf{(requiring two supportive facts to solve)}\\
\\
\textbf{Story}: & \\
& Sandra journeyed to the hallway. \\
& Mary went to the bathroom. \\
& Mary took the apple there. \\
& Mary dropped the apple. \\
\\
\textbf{Question}: & \\
& Where is the apple? \\
\textbf{Model's output}: & \\
& bathroom
\end{tabular}
\end{minipage}
\\  \vfill
\vspace{20pt} % maximize space between the minipages
\begin{minipage}{\textwidth}
    \centering
    \begin{subfigure}[t]{\textwidth}
        \centering
        \includegraphics[height=0.8in]{04-part-03/chapter-06/figs_and_tables/figs_attention_babi/e2-step1}
        \caption{Step 1}
    \end{subfigure}%
    \hfill \hfill
    \begin{subfigure}[t]{\textwidth}
        \centering
        \includegraphics[height=0.8in]{04-part-03/chapter-06/figs_and_tables/figs_attention_babi/e2-step2}
        \caption{Step 2}
    \end{subfigure}
    \hfill \hfill
    \begin{subfigure}[t]{\textwidth}
        \centering
        \includegraphics[height=0.8in]{04-part-03/chapter-06/figs_and_tables/figs_attention_babi/e2-step3}
        \caption{Step 3}
    \end{subfigure}
    \hfill \hfill 
    \begin{subfigure}[t]{\textwidth}
        \centering
        \includegraphics[height=0.8in]{04-part-03/chapter-06/figs_and_tables/figs_attention_babi/e2-step4}
        \caption{Step 4}
    \end{subfigure}
    \end{minipage}
    \caption{\label{fig:ex2}Visualization of the attention distributions, when encoding the question: \emph{``Where is the apple?''}.}
\end{figure}

\afterpage{\clearpage}

\begin{figure}[!h]
\begin{minipage}{\textwidth}
\fontsize{8}{8}\fontfamily{pcr}\selectfont
\begin{tabular}{l l}
\textbf{An example from tasks 2}: & \textbf{(requiring two supportive facts to solve)}\\
\\
\textbf{Story}: & \\
& John went to the hallway. \\
& John went back to the bathroom. \\
& John grabbed the milk there. \\
& Sandra went back to the office. \\
& Sandra journeyed to the kitchen. \\
& Sandra got the apple there. \\
& Sandra dropped the apple there. \\
& John dropped the milk. \\
\\
\textbf{Question}: & \\
& Where is the milk? \\
\textbf{Model's output}: & \\
& bathroom
\end{tabular}
\end{minipage}
\\ \vfill
\vspace{20pt} % maximize space between the minipages
\begin{minipage}{\textwidth}
    \centering
    \begin{subfigure}[t]{\textwidth}
        \centering
        \includegraphics[height=1.3in]{04-part-03/chapter-06/figs_and_tables/figs_attention_babi/e3-step1}
        \caption{Step 1}
    \end{subfigure}%
    \hfill \hfill
    \begin{subfigure}[t]{\textwidth}
        \centering
        \includegraphics[height=1.3in]{04-part-03/chapter-06/figs_and_tables/figs_attention_babi/e3-step2}
        \caption{Step 2}
    \end{subfigure}
    \hfill \hfill
    \begin{subfigure}[t]{\textwidth}
        \centering
        \includegraphics[height=1.3in]{04-part-03/chapter-06/figs_and_tables/figs_attention_babi/e3-step3}
        \caption{Step 3}
    \end{subfigure}
    \hfill \hfill 
    \begin{subfigure}[t]{\textwidth}
        \centering
        \includegraphics[height=1.3in]{04-part-03/chapter-06/figs_and_tables/figs_attention_babi/e3-step4}
        \caption{Step 4}
    \end{subfigure}
    \end{minipage}
    \caption{\label{fig:ex3}Visualization of the attention distributions, when encoding the question: \emph{``Where is the milk?''}.}
\end{figure}

\afterpage{\clearpage}

\begin{figure}[!h]
\begin{minipage}{\textwidth}
\fontsize{8}{8}\fontfamily{pcr}\selectfont
\begin{tabular}{l l}
\textbf{An example from tasks 3}: & \textbf{(requiring three supportive facts to solve)}\\
\\

\textbf{Story}: & \\
Mary got the milk. \\
& John moved to the bedroom. \\
& Daniel journeyed to the office. \\
& John grabbed the apple there. \\
& John got the football. \\
& John journeyed to the garden. \\
& Mary left the milk. \\
& John left the football. \\
& Daniel moved to the garden. \\
& Daniel grabbed the football. \\
& Mary moved to the hallway. \\
& Mary went to the kitchen. \\
& John put down the apple there. \\
& John picked up the apple. \\
& Sandra moved to the hallway. \\
& Daniel left the football there. \\
& Daniel took the football. \\
& John travelled to the kitchen. \\
& Daniel dropped the football. \\
& John dropped the apple. \\
& John grabbed the apple. \\
& John went to the office. \\
& Sandra went back to the bedroom. \\
& Sandra took the milk. \\
& John journeyed to the bathroom. \\
& John travelled to the office. \\
& Sandra left the milk. \\
& Mary went to the bedroom. \\
& Mary moved to the office. \\
& John travelled to the hallway. \\
& Sandra moved to the garden. \\
& Mary moved to the kitchen. \\
& Daniel took the football. \\
& Mary journeyed to the bedroom. \\
& Mary grabbed the milk there. \\
& Mary discarded the milk. \\
& John went to the garden. \\
& John discarded the apple there. \\
\\
\textbf{Question}: & \\
& Where was the apple before the bathroom? \\
\textbf{Model's output}: & \\
& office
\end{tabular}
\end{minipage}
\end{figure}
\begin{figure}[h!]\ContinuedFloat
\begin{minipage}{\textwidth}
    \centering
    \begin{subfigure}[t]{\textwidth}
        \centering
        \includegraphics[height=4.2in]{04-part-03/chapter-06/figs_and_tables/figs_attention_babi/e4-step1}
        \caption{Step 1}
    \end{subfigure}%
    \hfill \hfill
    \begin{subfigure}[t]{\textwidth}
        \centering
        \includegraphics[height=4.2in, trim={0 0 0 0.1cm},clip]{04-part-03/chapter-06/figs_and_tables/figs_attention_babi/e4-step2}
        \caption{Step 2}
    \end{subfigure}
\end{minipage}
\end{figure}
\begin{figure}[h!]\ContinuedFloat
\begin{minipage}{\textwidth}\ContinuedFloat
    \begin{subfigure}[t]{\textwidth}
        \centering
        \includegraphics[height=4.2in]{04-part-03/chapter-06/figs_and_tables/figs_attention_babi/e4-step3}
        \caption{Step 3}
    \end{subfigure}
    \hfill \hfill 
    \begin{subfigure}[t]{\textwidth}
        \centering
        \includegraphics[height=4.2in]{04-part-03/chapter-06/figs_and_tables/figs_attention_babi/e4-step4}
        \caption{Step 4}
    \end{subfigure}
    \caption{\label{fig:ex4}Visualization of the attention distributions, when encoding the question: \emph{``Where was the apple before the bathroom?''}.}
\end{minipage}
\end{figure}

The above examples illustrate that there is a notion of temporal states in UT, where the model updates its states (memory) in each step based on the output of previous steps, and this chain of updates can also be viewed as steps in a multi-hop reasoning process.


\section{Conclusion}
In this chapter we introduced the Universal Transformer, a generalization of the Transformer model that extends its theoretical capabilities and produces state-of-the-art results on a wide range of challenging sequence modeling tasks, such as language understanding but also a variety of algorithmic tasks, thereby addressing a key shortcoming of the standard Transformer. The Universal Transformer combines the following key properties into one model:

\textbf{Weight sharing}: Following intuitions behind weight sharing found in CNNs and RNNs, we extend the Transformer with a simple form of weight sharing that strikes an effective balance between inductive bias and model expressivity, which we show extensively on both small and large-scale experiments.

\textbf{Conditional computation}: In our goal to build a computationally universal machine, we equipped the Universal Transformer with the ability to halt or continue computation through a recently introduced mechanism, which shows stronger results compared to the fixed-depth Universal Transformer.

We are enthusiastic about the recent developments on parallel-in-time sequence models. By adding computational capacity and recurrence in processing depth, we hope that further improvements beyond the basic Universal Transformer presented here will help us build learning algorithms that are both more powerful, data efficient, and generalize beyond the current state-of-the-art.
